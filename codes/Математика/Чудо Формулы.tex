
\begingroup
\small

\begin{enumerate}
\item % Число путей без соседних шагов:
$\sum_{0\le k\le n}\binom{n-k}{k}=\mathrm{Fib}_{n+1}$

\item % Альтернирующая сумма биномиалов:
$\sum_{i=0}^{k}(-1)^i\binom{n}{i}=(-1)^k\binom{n-1}{k}$

\item % Сумма верхних диагоналей треугольника Паскаля:
$\sum_{i=0}^{k}\binom{n+i}{i}=\binom{n+k+1}{k}$

\item % (Вандермонд) Свёртка биномиалов:
$\sum_{k=0}^{r}\binom{m}{k}\binom{n}{r-k}=\binom{m+n}{r}$

\item % «Хоккейная клюшка» (сумма биномиалов по строке):
$\sum_{i=r}^{n}\binom{i}{r}=\binom{n+1}{r+1}$

\item % (Куммер) Показатель простого $p$ в биномиале:
Степень $p$ в $\binom{n}{m}$ равна числу переносов при сложении $m$ и $n-m$ в $p$-ичной системе.

\item % Сумма отношений биномиалов:
$\sum_{i=0}^{n}\frac{\binom{k}{i}}{\binom{n}{i}}
=\frac{\binom{n+1}{\,n-k+1\,}}{\binom{n}{k}}$

\item % Рекурсия для дерранжментов (фиксированная позиция):
$d(n)=(n-1)\bigl(d(n-1)+d(n-2)\bigr),\  d(0)=1,\ d(1)=0$

\item % Теорема Лукаса (по модулю простого $p$):
$\binom{m}{n}\equiv\prod_i\binom{m_i}{n_i}\pmod p,$
\quad где $m_i,n_i$ — цифры $m,n$ в $p$-ичной записи.

\item % Стирлинг II рода в разложении степеней:
$\sum_{i=0}^{n}\binom{n}{i}i^k
=\sum_{j=0}^{k}S_2(k,j)\,\frac{n!}{(n-j)!}\,2^{\,n-j}$

\item % Частичная сумма биномиалов с весом $x^i$:
$\sum_{i=0}^{n-1}\binom{i}{j}x^i
= x^j(1-x)^{-j-1}\!\Bigl(1-x^n\sum_{i=0}^j\binom{n}{i}x^{\,j-i}(1-x)^i\Bigr)$

\item % Биномиальная инверсия (следствие):
$P(n)=\sum_{k=0}^{n}\binom{n}{k}Q(k) \implies Q(n)=\sum_{k=0}^{n}(-1)^{\,n-k}\binom{n}{k}P(k)$

\item % Инверсия со знаком (следствие):
$P(n)=\sum_{k=0}^{n}(-1)^k\binom{n}{k}Q(k) \implies Q(n)=\sum_{k=0}^{n}(-1)^k\binom{n}{k}P(k)$

\item % Стирлинговые числа первого рода (рекурсия):
$S_1(n,k)$ — число перестановок длины $n$ с $k$ циклами,
$S_1(n,k)=(n-1)S_1(n-1,k)+S_1(n-1,k-1),\ S_1(0,0)=1$

\item % Стирлинговые числа второго рода (рекурсия и частные случаи):
$S_2(n,k)$ — число разбиений $n$-множества на $k$ непустых блоков,
$S_2(n,k)=kS_2(n-1,k)+S_2(n-1,k-1),\ S_2(0,0)=1$

\item % Частные формулы для $S_2$:
$S_2(n,2)=2^{\,n-1}-1$

$S_2(n,3)=\tfrac{1}{2}\bigl(3^{\,n-1}-2^{\,n-1}\bigr)-\tfrac{1}{2}\bigl(3^{\,n-1}-1\bigr)$

$S_2(n,4)=\tfrac{1}{2!}\Bigl[(4^{\,n-1}-3^{\,n-1})-(4^{\,n-1}-2^{\,n-1})+\tfrac{1}{3}(4^{\,n-1}-1)\Bigr]$

$S_2(n,5)=\tfrac{1}{24}\,5^{\,n-1}-\tfrac{1}{6}\,4^{\,n-1}
+\tfrac{1}{4}\,3^{\,n-1}-\tfrac{1}{6}\,2^{\,n-1}+\tfrac{1}{24}$

\item % Пониженные числа Стирлинга:
$S_2^d(n,k)$ — число разбиений с расстоянием $\ge d$ внутри блока,
$S_2^d(n,k)=S_2(n-d+1,k-d+1)$

\item % Сумма $i a^i$:
$\sum_{i=1}^n i a^i=\frac{a\bigl(n a^{n+1}-(n+1)a^n+1\bigr)}{(a-1)^2}$

\item % Формула Эйлера—Пентовой (пентагональные числа):
$\prod_{n=1}^\infty(1-x^n)=1+\sum_{k=1}^{\infty}(-1)^k\bigl(x^{\,k(3k-1)/2}+x^{\,k(3k+1)/2}\bigr)$

\item % Критерий принадлежности к последовательности Фибоначчи:
$n$ — член Фибоначчи $\iff$ хотя бы одно из чисел $5n^2\pm4$ — точный квадрат.

\item % Периодичность Фибоначчи по модулю $n$:
Период последовательности Фибоначчи по модулю $n$ не превышает $6n$.

\item % Пифагоровы тройки (все целые решения $a^2+b^2=c^2$):
Пифагоровы тройки: пусть $m>n>0,\ \gcd(m,n)=1,$ и не оба нечётные. Тогда
$\;a=k(m^2-n^2),\ b=k(2mn),\ c=k(m^2+n^2)$


\item % Число решений $a^2+b^2=n^2$ с фиксированной гипотенузой:
$\displaystyle \#\{(a,b):a^2+b^2=n^2,a>0,b>0\}
=\Big(\prod_{p^\alpha\parallel n,\ p = 4k+1}2\alpha+1\Big)-1$

\item % Определения функций представления суммой квадратов:
$r_4(n)=\#\{x_1^2+\cdots+x_4^2=n\},\ 
r_8(n)=\#\{x_1^2+\cdots+x_8^2=n\}$

\item % Формулы для $r_4$ и $r_8$:
$r_4(n)=8\sum_{d\mid n, 4\nmid d}d,\ 
r_8(n)=16\sum_{d\mid n}(-1)^{\,n+d}d^3$

\item % Количество неотрицательных решений $ax+by=n$ и обратные:
$\#\{ax+by=n,\ x,y\ge0\}=\frac{n}{ab}-\Bigl\{\tfrac{b'n}{a}\Bigr\}-\Bigl\{\tfrac{a'n}{b}\Bigr\}+1$,

где $a'$ и $b'$ — обратные: $aa'\equiv1\pmod b,\; bb'\equiv1\pmod a$

\item %Формула для $\varphi(mn)$ через $\gcd$:
$\varphi(mn)=\varphi(m)\varphi(n)\frac{d}{\varphi(d)},\  d=\gcd(m,n)$

\item % Сокращение показателя по $\varphi(m)$ при $x\ge\log_2 m$:
$n^x\bmod m = n^{\varphi(m)+\bigl(x\bmod\varphi(m)\bigr)}\bmod m,\ x\ge\log_2 m$

\item % Подсчёт квадратсвободных $\le N$ через функцию Мёбиуса:
$\sum_{n=1}^N\mu^2(n)=\sum_{k=1}^{\lfloor\sqrt N\rfloor}\mu(k)\bigl\lfloor N/k^2\bigr\rfloor$

\item % Инверсия Мёбиуса (аддитивная):
$g(n)=\sum_{d\mid n}f(d) \implies f(n)=\sum_{d\mid n}\mu(d)\,g(n/d)$

\item % Инверсия Мёбиуса (мультипликативная):
$F(n)=\prod_{d\mid n}f(d) \implies f(n)=\prod_{d\mid n}F(n/d)^{\mu(d)}$

\item % Связь $\gcd/\operatorname{lcm}$ №1:
$\gcd\bigl(a,\operatorname{lcm}(b,c)\bigr)=\operatorname{lcm}\bigl(\gcd(a,b),\gcd(a,c)\bigr)$

\item % Связь $\gcd/\operatorname{lcm}$ №2:
$\operatorname{lcm}\bigl(a,\gcd(b,c)\bigr)=\gcd\bigl(\operatorname{lcm}(a,b),\operatorname{lcm}(a,c)\bigr)$

\item % Симметрия для тройки:
$\gcd\bigl(\operatorname{lcm}(a,b),\operatorname{lcm}(b,c),\operatorname{lcm}(a,c)\bigr)
 =\\= \operatorname{lcm}\bigl(\gcd(a,b),\gcd(b,c),\gcd(a,c)\bigr)$

\item % Сумма $\operatorname{lcm}$ по сетке $1\ldots n$:
$\sum_{i,j=1}^n\operatorname{lcm}(i,j)
=\sum_{l=1}^n\Bigl(\frac{(1+\lfloor n/l\rfloor)\lfloor n/l\rfloor}{2}\Bigr)^2\sum_{d\mid l}\mu(d)ld$

\item % Сумма $\operatorname{lcm}(k,n)$ по $k$:
$\sum_{i=1}^n\operatorname{lcm}(i,n)=\frac{n}{2}\Bigl(\sum_{d\mid n}\varphi(d)d+1\Bigr)$

\item % Закон квадратичной взаимности (Лежандр):
$\left(\tfrac{p}{q}\right)\left(\tfrac{q}{p}\right)=(-1)^{\frac{(p-1)(q-1)}{4}}$
 для нечётных простых $p,q$

\end{enumerate}

\endgroup


