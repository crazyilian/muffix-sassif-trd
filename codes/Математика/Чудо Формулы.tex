\small
% (1) Число путей без соседних шагов = Fib_{n+1}
\[
\sum_{0\le k \le n} \binom{n-k}{k} = \mathrm{Fib}_{n+1}
\]

% (9) Альтернирующая сумма биномиалов
\[
\sum_{i=0}^{k} (-1)^i \binom{n}{i} = (-1)^k \binom{n-1}{k}
\]

% (10) Сумма верхних диагоналей треугольника Паскаля
\[
\sum_{i=0}^{k} \binom{n+i}{i} = \binom{n+k+1}{k}
\]

% (13) Вандермонд: свёртка биномиалов
\[
\sum_{k=0}^{r} \binom{m}{k}\binom{n}{r-k} = \binom{m+n}{r}
\]

% (14) «Хоккейная клюшка»
\[
\sum_{i=r}^{n} \binom{i}{r} = \binom{n+1}{r+1}
\]

% (25) Куммер: показатель простого p в биномиале
Степень $p$ в $\binom{n}{m}$ = числу переносов при сложении $m$ и $n-m$ в $p$-СС

% (32) Сумма отношений биномиалов
\[
\sum_{i=0}^{n} \frac{\binom{k}{i}}{\binom{n}{i}} = \frac{\binom{n+1}{\,n-k+1\,}}{\binom{n}{k}}
\]

% (33) Дерранжменты: рекурсия по фиксированной позиции
\[
d(n) = (n-1)(d(n-1)+d(n-2)),\quad d(0)=1,\ d(1)=0
\]

% (36) Теорема Лукаса
\[
\binom{m}{n} \equiv \prod_{i} \binom{m_i}{n_i} \pmod p,\quad
\text{$m_i,n_i$ — цифры в $p$-СС}
\]

% (37) Стирлинг II рода в разложении степеней
\[
\sum_{i=0}^{n} \binom{n}{i} i^k
=\sum_{j=0}^{k} S_2(k,j)\,\tfrac{n!}{(n-j)!}\,2^{\,n-j}
\]

% (38) Частичная сумма биномиалов с весом $x^i$
\[
\sum_{i=0}^{n-1} \binom{i}{j} x^i
= x^j(1-x)^{-j-1}\!\left(1-x^n\sum_{i=0}^j \binom{n}{i} x^{\,j-i}(1-x)^i\right)
\]

% (40) Биномиальная инверсия (следствие)
Если \(P(n)=\sum_{k=0}^{n}\binom{n}{k}Q(k)\), то
\[
Q(n)=\sum_{k=0}^{n}(-1)^{\,n-k}\binom{n}{k}P(k)
\]

% (41) Инверсия со знаком (следствие)
Если \(P(n)=\sum_{k=0}^{n}(-1)^k\binom{n}{k}Q(k)\), то
\[
Q(n)=\sum_{k=0}^{n}(-1)^k\binom{n}{k}P(k)
\]

% (55-56) S_1: число перестановок на k циклов
$S_1(n,k)$ — число перестановок длины $n$ из $k$ циклов
\[
S_1(n,k)=(n-1)S_1(n-1,k)+S_1(n-1,k-1),\ S_1(0,0)=1
\]

% (60-62) S_2: число разбиений на k блоков
$S_2(n,k)$ — число разбиений $n$-множества на $k$ непустых блоков
\[
S_2(n,k)=kS_2(n-1,k)+S_2(n-1,k-1),\ S_2(0,0)=1
\]
\[
S_2(n,2)=2^{n-1}-1
\]
% S_2(3..5) — явные формулы
\[
S_2(n,3)=\tfrac{1}{2}(3^{n-1}-2^{n-1})-\tfrac{1}{2}(3^{n-1}-1)
\]
\[
S_2(n,4)=\tfrac{1}{2!}\left[(4^{n-1}-3^{n-1})-(4^{n-1}-2^{n-1})+\tfrac{1}{3}(4^{n-1}-1)\right]
\]
\[
S_2(n,5)=\tfrac{1}{24}\,5^{\,n-1}-\tfrac{1}{6}\,4^{\,n-1}
+\tfrac{1}{4}\,3^{\,n-1}-\tfrac{1}{6}\,2^{\,n-1}
+\tfrac{1}{24}
\]

% (65) Пониженные числа Стирлинга
$S_2^d(n,k)$ — число разбиений $n$-множества на $k$ непустых блоков c расстоянием хотя бы $d$ между любыми двумя элементами одного блока
\[
S_2^d(n,k)=S_2(n-d+1,k-d+1)
\]

% (80) Сумма $i a^i$
\[
\sum_{i=1}^n i a^i=\frac{a(n a^{n+1}-(n+1)a^n+1)}{(a-1)^2}
\]

% (84) Теорема о пятиугольных числах (как в оригинале Shohag)
\[
\prod_{n=1}^\infty(1-x^n)
=
1+\sum_{k=1}^{\infty}(-1)^k\!\left(x^{\,k(3k-1)/2}+x^{\,k(3k+1)/2}\right)
\]

% (94) Критерий принадлежности к Фибоначчи
\[
n \text{ — Фибоначчи} \iff 5n^2\pm4 \text{ — квадрат хотя бы один из } \pm
\]

% (98) Периодичность Фибоначчи по модулю $n$
\[
\text{Период последовательности Фибоначчи по модулю } n \le 6n
\]

% (101) Пифагоровы тройки: описание всех целых решений $a^2+b^2=c^2$
\[
\text{Пифагоровы тройки: } m>n>0,\ \gcd(m,n)=1,\ \text{не оба нечётные}
\]
\[
a=k(m^2-n^2),\quad b=k(2mn),\quad c=k(m^2+n^2)
\]

% (102) Количество решений $a^2+b^2=n^2$ с фиксированной гипотенузой
\[
\#\{(a,b):a^2+b^2=n^2,a>0,b>0\}
=\Big(\prod_{p^\alpha\parallel n \\ p = 4k+1}(2\alpha+1)-1\Big)
\]

% (104) Представления суммой квадратов: определения и формулы
\[
r_4(n)=\#\{x_1^2+\cdots+x_4^2=n\},\quad
r_8(n)=\#\{x_1^2+\cdots+x_8^2=n\}
\]

% (105) Представления суммой 4 и 8 квадратов
\[
r_4(n)=8\sum_{\substack{d\mid n\\4\not\mid d}}d,\qquad
r_8(n)=16\sum_{d\mid n}(-1)^{n+d}d^3
\]

% (112) Кол-во неотрицательных решений $ax+by=n$ и что такое $a',b'$
\[
\#\{ax+by=n,\ x,y\ge0\}=\frac{n}{ab}-\left\{\tfrac{b'n}{a}\right\}-\left\{\tfrac{a'n}{b}\right\}+1
\]
\[
a'\text{ и }b' \text{ — обратные: } aa'\equiv1\pmod b,\ bb'\equiv1\pmod a
\]

% (131) Формула для $\varphi(mn)$ через $\gcd$
\[
\varphi(mn)=\varphi(m)\varphi(n)\tfrac{d}{\varphi(d)},\ d=\gcd(m,n)
\]

% (139) Сокращение показателя по $\varphi(m)$ при $x\ge\log_2 m$
\[
n^x\bmod m = n^{\varphi(m)+x\bmod \varphi(m)}\bmod m,\quad x\ge\log_2 m
\]

% (149) Подсчёт квадратсвободных $\le N$ через функцию Мёбиуса
\[
\sum_{n=1}^N \mu^2(n)=\sum_{k=1}^{\lfloor\sqrt N\rfloor}\mu(k)\lfloor N/k^2\rfloor
\]

% (150) Инверсия Мёбиуса (аддитивная)
\[
\Bigl(\,g(n)=\sum_{d\mid n} f(d)\,\Bigr)
\;\;\Longrightarrow\;\;
\Bigl(\,f(n)=\sum_{d\mid n}\mu(d)\,g(n/d)\,\Bigr)
\]

% (151) Инверсия Мёбиуса (мультипликативная)
\[
\Bigl(\,F(n)=\prod_{d\mid n} f(d)\,\Bigr)
\;\;\Longrightarrow\;\;
\Bigl(\,f(n)=\prod_{d\mid n}F(n/d)^{\mu(d)}\,\Bigr)
\]

% (160) Связь $\gcd/\operatorname{lcm}$ №1
\[
\gcd(a,\operatorname{lcm}(b,c))=\operatorname{lcm}(\gcd(a,b),\gcd(a,c))
\]

% (161) Связь $\gcd/\operatorname{lcm}$ №2
\[
\operatorname{lcm}(a,\gcd(b,c))=\gcd(\operatorname{lcm}(a,b),\operatorname{lcm}(a,c))
\]

% (174) Симметрия $\gcd/\operatorname{lcm}$ для тройки
\[
\gcd(\operatorname{lcm}(a,b),\operatorname{lcm}(b,c),\operatorname{lcm}(a,c))
=\operatorname{lcm}(\gcd(a,b),\gcd(b,c),\gcd(a,c))
\]

% (173) Сумма $\operatorname{lcm}$ по сетке $1..n$
\[
\sum_{i,j=1}^n \operatorname{lcm}(i,j)
=\sum_{l=1}^n\left(\tfrac{(1+\lfloor n/l\rfloor)\lfloor n/l\rfloor}{2}\right)^2 \sum_{d\mid l}\mu(d)ld
\]

% (176) Сумма $\operatorname{lcm}(k,n)$ по $k$
\[
\sum_{i=1}^n \operatorname{lcm}(i,n)=\tfrac{n}{2}\left(\sum_{d\mid n}\varphi(d)d+1\right)
\]

% (Legendre reciprocity) Закон квадратичной взаимности
\[
\left(\tfrac{p}{q}\right)\left(\tfrac{q}{p}\right)=(-1)^{\frac{(p-1)(q-1)}{4}}
\quad \text{Лежандр } p, q \text{ нечётные простые}
\]

