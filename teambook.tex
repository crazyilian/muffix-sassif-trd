\documentclass[9pt,a4paper,landscape,twosided]{extarticle}

\usepackage[T2A]{fontenc}
\usepackage[T1]{fontenc}
\usepackage[utf8]{inputenc}
\usepackage[english, russian]{babel}
\usepackage{listings}
\usepackage[usenames,dvipsnames]{color}
\usepackage{amsmath}
\usepackage{verbatim}
\usepackage{hyperref}
\usepackage{color}
\usepackage{geometry}
\usepackage{multicol}
\usepackage{graphicx}

\geometry{verbose,landscape,a4paper,tmargin=1.1cm,bmargin=0.7cm,lmargin=0.6cm,rmargin=0.6cm}

\usepackage{listings}
\usepackage{color}

\definecolor{dkgreen}{rgb}{0,0.6,0}
\definecolor{gray}{rgb}{0.5,0.5,0.5}
\definecolor{mauve}{rgb}{0.58,0,0.82}

%% Golang definition for listings
%% http://github.io/julienc91/lstlistings-golang
%%
\lstdefinelanguage{Golang}%
  {morekeywords=[1]{package,import,func,type,struct,return,defer,panic,%
     recover,select,var,const,iota,},%
   morekeywords=[2]{string,uint,uint8,uint16,uint32,uint64,int,int8,int16,%
     int32,int64,bool,float32,float64,complex64,complex128,byte,rune,uintptr,%
     error,interface},%
   morekeywords=[3]{map,slice,make,new,nil,len,cap,copy,close,true,false,%
     delete,append,real,imag,complex,chan,},%
   morekeywords=[4]{for,break,continue,range,go,goto,switch,case,fallthrough,if,%
     else,default,},%
   morekeywords=[5]{Println,Printf,Error,Print,},%
   sensitive=true,%
   morecomment=[l]{//},%
   morecomment=[s]{/*}{*/},%
   morestring=[b]',%
   morestring=[b]",%
   morestring=[s]{`}{`},%
   }


\lstset{frame=tb,
    language=C++,
    aboveskip=1mm,
    belowskip=1mm,
    showstringspaces=true,
    columns=flexible,
    keepspaces=true,
    basicstyle={\fontsize{9pt}{9pt}\ttfamily},
    numbers=none,
    numberstyle=\tiny\color{gray},
    keywordstyle=\color{blue},
    commentstyle=\color{dkgreen},
    stringstyle=\color{mauve},
    escapebegin=\color{dkgreen},
    breaklines=true,
    breakatwhitespace=false,
    tabsize=2,
    inputencoding = utf8,  % Input encoding
    extendedchars = true,  % Extended ASCII
    mathescape    = true   % Mathematical expressions between $
    captionpos    = b,     % Caption position
    literate      =        % Support additional characters
        {á}{{\'a}}1  {é}{{\'e}}1  {í}{{\'i}}1 {ó}{{\'o}}1  {ú}{{\'u}}1
        {Á}{{\'A}}1  {É}{{\'E}}1  {Í}{{\'I}}1 {Ó}{{\'O}}1  {Ú}{{\'U}}1
        {à}{{\`a}}1  {è}{{\`e}}1  {ì}{{\`i}}1 {ò}{{\`o}}1  {ù}{{\`u}}1
        {À}{{\`A}}1  {È}{{\'E}}1  {Ì}{{\`I}}1 {Ò}{{\`O}}1  {Ù}{{\`U}}1
        {ä}{{\"a}}1  {ë}{{\"e}}1  {ï}{{\"i}}1 {ö}{{\"o}}1  {ü}{{\"u}}1
        {Ä}{{\"A}}1  {Ë}{{\"E}}1  {Ï}{{\"I}}1 {Ö}{{\"O}}1  {Ü}{{\"U}}1
        {â}{{\^a}}1  {ê}{{\^e}}1  {î}{{\^i}}1 {ô}{{\^o}}1  {û}{{\^u}}1
        {Â}{{\^A}}1  {Ê}{{\^E}}1  {Î}{{\^I}}1 {Ô}{{\^O}}1  {Û}{{\^U}}1
        {œ}{{\oe}}1  {Œ}{{\OE}}1  {æ}{{\ae}}1 {Æ}{{\AE}}1  {ß}{{\ss}}1
        {ç}{{\c c}}1 {Ç}{{\c C}}1 {ø}{{\o}}1  {å}{{\r a}}1 {Å}{{\r A}}1
        {ñ}{{\~n}}1  {Ñ}{{\~N}}1  {¿}{{?`}}1  {¡}{{!`}}1
        {а}{{\selectfont\char224}}1
        {б}{{\selectfont\char225}}1
        {в}{{\selectfont\char226}}1
        {г}{{\selectfont\char227}}1
        {д}{{\selectfont\char228}}1
        {е}{{\selectfont\char229}}1
        {ё}{{\"e}}1
        {ж}{{\selectfont\char230}}1
        {з}{{\selectfont\char231}}1
        {и}{{\selectfont\char232}}1
        {й}{{\selectfont\char233}}1
        {к}{{\selectfont\char234}}1
        {л}{{\selectfont\char235}}1
        {м}{{\selectfont\char236}}1
        {н}{{\selectfont\char237}}1
        {о}{{\selectfont\char238}}1
        {п}{{\selectfont\char239}}1
        {р}{{\selectfont\char240}}1
        {с}{{\selectfont\char241}}1
        {т}{{\selectfont\char242}}1
        {у}{{\selectfont\char243}}1
        {ф}{{\selectfont\char244}}1
        {х}{{\selectfont\char245}}1
        {ц}{{\selectfont\char246}}1
        {ч}{{\selectfont\char247}}1
        {ш}{{\selectfont\char248}}1
        {щ}{{\selectfont\char249}}1
        {ъ}{{\selectfont\char250}}1
        {ы}{{\selectfont\char251}}1
        {ь}{{\selectfont\char252}}1
        {э}{{\selectfont\char253}}1
        {ю}{{\selectfont\char254}}1
        {я}{{\selectfont\char255}}1
        {А}{{\selectfont\char192}}1
        {Б}{{\selectfont\char193}}1
        {В}{{\selectfont\char194}}1
        {Г}{{\selectfont\char195}}1
        {Д}{{\selectfont\char196}}1
        {Е}{{\selectfont\char197}}1
        {Ё}{{\"E}}1
        {Ж}{{\selectfont\char198}}1
        {З}{{\selectfont\char199}}1
        {И}{{\selectfont\char200}}1
        {Й}{{\selectfont\char201}}1
        {К}{{\selectfont\char202}}1
        {Л}{{\selectfont\char203}}1
        {М}{{\selectfont\char204}}1
        {Н}{{\selectfont\char205}}1
        {О}{{\selectfont\char206}}1
        {П}{{\selectfont\char207}}1
        {Р}{{\selectfont\char208}}1
        {С}{{\selectfont\char209}}1
        {Т}{{\selectfont\char210}}1
        {У}{{\selectfont\char211}}1
        {Ф}{{\selectfont\char212}}1
        {Х}{{\selectfont\char213}}1
        {Ц}{{\selectfont\char214}}1
        {Ч}{{\selectfont\char215}}1
        {Ш}{{\selectfont\char216}}1
        {Щ}{{\selectfont\char217}}1
        {Ъ}{{\selectfont\char218}}1
        {Ы}{{\selectfont\char219}}1
        {Ь}{{\selectfont\char220}}1
        {Э}{{\selectfont\char221}}1
        {Ю}{{\selectfont\char222}}1
        {Я}{{\selectfont\char223}}1
        {і}{{\selectfont\char105}}1
        {ї}{{\selectfont\char168}}1
        {є}{{\selectfont\char185}}1
        {ґ}{{\selectfont\char160}}1
        {І}{{\selectfont\char73}}1
        {Ї}{{\selectfont\char136}}1
        {Є}{{\selectfont\char153}}1
        {Ґ}{{\selectfont\char128}}1 
}

\newcommand{\dollar}{\mbox{\textdollar}}

\setlength{\columnsep}{0.1in}
\setlength{\columnseprule}{1px}

\usepackage{fancyhdr}
\pagestyle{fancyplain}
\fancyhf{}
\fancyhead[R]{\fontsize{12}{12}\selectfont\bf{\thepage}}
\fancyhead[L]{\fontsize{8}{8}\selectfont\bf{NRU HSE (Andrianov, Lepeshov, Shulyatev)}}
\renewcommand{\headrulewidth}{0pt}
\renewcommand\headrule{\vspace{-0.2cm}\hspace{-0.4cm}\rule{\paperwidth}{0.4pt}\vspace{-0.8cm}}

\begin{document}

\title{\bf{Muffix Sassif -- TRD}}
\author{Andrianov, Lepeshov, Shulyatev}
\date{\selectlanguage{english}\today}
\maketitle
\begin{multicols*}{3}
\begin{center}{\includegraphics[width=5cm]{/home/ilian/github/muffix-sassif-trd/notebook-generator/picture.jpg}\end{center}
\tableofcontents
\end{multicols*}
\pagebreak
\begin{multicols*}{3}
\lstloadlanguages{C++,Java}


\section{Геометрия}

\subsection{3D}
\begin{lstlisting}[language=C++]
double eps = 1e-7;

struct Pt {
  double x;
  double y;
  double z;

  Pt(double x_, double y_, double z_) : x(x_), y(y_), z(z_) {}

  Pt operator-(const Pt& other) const {
    return {x - other.x, y - other.y, z - other.z};
  }

  Pt operator+(const Pt& other) const {
    return {x + other.x, y + other.y, z + other.z};
  }

  Pt operator/(const double& a) const {
    return {x / a, y / a, z / a};
  }

  Pt operator*(const double& a) const {
    return {x * a, y * a, z * a};
  }

  Pt cross(const Pt& p2) const {
    double nx = y * p2.z - z * p2.y;
    double ny = z * p2.x - x * p2.z;
    double nz = x * p2.y - y * p2.x;
    return {nx, ny, nz};
  }

  bool operator==(const Pt& pt) const {
    return abs(x - pt.x) < eps && abs(y - pt.y) < eps && abs(z - pt.z) < eps;
  }

  double dist() {
    return sqrtl(x * x + y * y + z * z);
  }
};

struct Plane {
  double a, b, c, d;

  Plane(double a_, double b_, double c_, double d_) : a(a_), b(b_), c(c_), d(d_) {
    double kek = sqrtl(a * a + b * b + c * c);
    if (kek < eps) return;
    a /= kek;
    b /= kek;
    c /= kek;
    d /= kek;
  }

  double get_val(Pt p) {
    // НЕ СТАВИТЬ МОДУЛЬ
    return a * p.x + b * p.y + c * p.z + d;
  }
  
  double dist(Pt p) {
    return abs(get_val(p));
  }

  bool on_plane(Pt p) {
    return abs(get_val(p)) / sqrtl(a * a + b * b + c * c) < eps;
  }

  Pt proj(Pt p) {
    double t = (a * p.x + b * p.y + c * p.z + d) / (a * a + b * b + c * c);
    return p - Pt(a, b, c) * t;
  }
};

bool on_line(Pt p1, Pt p2, Pt p3) {
  return (p2 - p1).cross(p3 - p1) == Pt(0, 0, 0);
}

Plane get_plane(Pt p1, Pt p2, Pt p3) {
  Pt norm = (p2 - p1).cross(p3 - p1);
  Plane pl(norm.x, norm.y, norm.z, 0);
  pl.d = -pl.get_val(p1);
  return pl;
}

pair<pair<double, double>, pair<double, double>> get_xy(double a, double b, double c) {
  if (abs(a) > eps) {
    double y1 = 0, y2 = 10;
    return {{(-c - b * y1) / a, y1}, {(-c - b * y2) / a, y2}};
  }
  double x1 = 0, x2 = 10;
  return {{x1, (-c - a * x1) / b}, {x2, (-c - a * x2) / b}};
}

pair<Pt, Pt> intersect(Plane pl1, Plane pl2) {
  if (abs(pl2.a) < eps && abs(pl2.b) < eps && abs(pl2.c) < eps) {
    assert(false);
  }
  if (abs(pl2.a) > eps) {
    double nd = pl1.d - pl1.a * pl2.d / pl2.a;
    double nc = pl1.c - pl1.a * pl2.c / pl2.a;
    double nb = pl1.b - pl1.a * pl2.b / pl2.a;
    if (abs(nc) < eps && abs(nb) < eps) {
      // плоскости параллельны (могут совпадать)
      return {Pt(0, 0, 0), Pt(0, 0, 0)};
    }
    auto [yz1, yz2] = get_xy(nb, nc, nd);
    double x1 = (-pl2.d - pl2.c * yz1.second - pl2.b * yz1.first) / pl2.a;
    double x2 = (-pl2.d - pl2.c * yz2.second - pl2.b * yz2.first) / pl2.a;
    return {Pt(x1, yz1.first, yz1.second), Pt(x2, yz2.first, yz2.second)};
  }
  Plane copy_pl1(pl1.c, pl1.a, pl1.b, pl1.d);
  Plane copy_pl2(pl2.c, pl2.a, pl2.b, pl2.d);
  auto [p1, p2] = intersect(copy_pl1, copy_pl2);
  return {Pt(p1.y, p1.z, p1.x), Pt(p2.y, p2.z, p2.x)};
}
\end{lstlisting}

\subsection{SVG}
\begin{lstlisting}[language=C++]
struct SVG {
    FILE *out;
    ld sc = 50;

    void open() {
      out = fopen("image.svg", "w");
      fprintf(out, "<svg xmlns='http://www.w3.org/2000/svg' viewBox='-1000 -1000 2000 2000'>\n");
    }

    void line(pt a, pt b, string col = "green") {
      a = a * sc, b = b * sc;
      fprintf(out, "<line x1='%Lf' y1='%Lf' x2='%Lf' y2='%Lf' stroke='%s'/>\n", a.x, -a.y, b.x, -b.y, col.c_str());
    }

    void circle(pt a, ld r = -1, string col = "red") {
      r = (r == -1 ? 10 : sc * r);
      a = a * sc;
      fprintf(out, "<circle cx='%Lf' cy='%Lf' r='%Lf' fill='%s' fill-opacity='0.5'/>\n", a.x, -a.y, r, col.c_str());
    }

    void text(pt a, string s) {
      a = a * sc;
      fprintf(out, "<text x='%Lf' y='%Lf' font-size='10px'>%s</text>\n", a.x, -a.y, s.c_str());
    }

    void close() {
      fprintf(out, "</svg>\n");
      fclose(out);
      out = 0;
    }

    ~SVG() {
      if (out)
        close();
    }
};
\end{lstlisting}

\subsection{Вектор, прямая, окружность}
\begin{lstlisting}[language=C++]
//// Вектор ////

struct vctr {
  dbl x, y;
  vctr() {}
  vctr(dbl x, dbl y) : x(x), y(y) {}

  dbl operator%(const vctr &o) const { return x * o.x + y * o.y; }
  dbl operator*(const vctr &o) const { return x * o.y - y * o.x; }
  vctr operator+(const vctr &o) const { return {x + o.x, y + o.y}; }
  vctr operator-(const vctr &o) const { return {x - o.x, y - o.y}; }
  vctr operator-() const { return {-x, -y}; }
  vctr operator*(const dbl d) const { return {x * d, y * d}; }
  vctr operator/(const dbl d) const { return {x / d, y / d}; }
  void operator+=(const vctr &o) { x += o.x, y += o.y; }
  void operator-=(const vctr &o) { x -= o.x, y -= o.y; }
  dbl dist2() const { return x * x + y * y; }
  dbl dist() const { return sqrtl(dist2()); }
  vctr norm() const { return *this / dist(); }
};

dbl angle_between(const vctr &a, const vctr &b) {
  return atan2(b * a, b % a);
}

// y > 0 ? 0 : 1
bool is2plane(const vctr &a) {
  return sign(a.y) < 0 || (sign(a.y) == 0 && sign(a.x) < 0);
}

bool cmp_angle(const vctr &a, const vctr &b) {
  bool pla = is2plane(a);
  bool plb = is2plane(b);
  if (pla != plb)
    return pla < plb;
  return sign(a * b) > 0;
}

//// Прямая ////

struct line {
  dbl a, b, c;

  line() {}
  line(dbl a, dbl b, dbl c) : a(a), b(b), c(c) {}
  line(const vctr A, const vctr B) {
    a = A.y - B.y;
    b = B.x - A.x;
    c = A * B;
    assert(a != 0 || b != 0);
  }

  void operator*=(dbl x) { a *= x, b *= x, c *= x; }
  void operator/=(dbl x) { a /= x, b /= x, c /= x; }
  dbl get(const vctr P) const { return a * P.x + b * P.y + c; }
  vctr anyPoint() const {
    dbl x = -a * c / (a * a + b * b);
    dbl y = -b * c / (a * a + b * b);
    return vctr(x, y);
  }
  void normalize() {
    dbl d = sqrtl(a * a + b * b);
    a /= d;
    b /= d;
    c /= d;
  }
};

bool isparallel(line l1, line l2) {
  return vctr(l1.a, l1.b) * vctr(l2.a, l2.b) == 0;
}

vctr intersection(const line &l1, const line &l2) {
  dbl x = (l1.c * l2.b - l2.c * l1.b) / (l2.a * l1.b - l2.b * l1.a);
  dbl y = -(l1.c * l2.a - l2.c * l1.a) / (l2.a * l1.b - l2.b * l1.a);
  return vctr(x, y);
}

// Серединный перпендикуляр (не биссектриса!)
line bisection(const vctr A, const vctr B) {
  vctr M = (A + B) / 2;
  vctr AB = B - A;
  vctr norm = vctr(AB.y, -AB.x);
  return line(M, M + norm);
}

//// Окружность ////

struct circle {
  dbl x, y, r;

  circle() {}
  circle(dbl x, dbl y, dbl r) : x(x), y(y), r(r) {}
  circle(vctr P, dbl r) : x(P.x), y(P.y), r(r) {}
  circle(const vctr A, const vctr B) {
    vctr C = (A + B) / 2;
    x = C.x, y = C.y;
    r = (A - B).dist() / 2;
  }
  circle(const vctr A, const vctr B, const vctr C) {
    line l1 = bisection(A, B);
    line l2 = bisection(B, C);
    vctr P = intersection(l1, l2);
    x = P.x, y = P.y;
    r = (P - A).dist();
  }

  bool isin(const vctr P) const {
    return (vctr(x, y) - P).dist2() <= r * r;
  }
  vctr cent() const { return vctr(x, y); }
};


\end{lstlisting}

\subsection{Выпуклая оболочка}
\begin{lstlisting}[language=C++]
vctr minvctr(INF, INF);

bool cmp_convex_hull(const vctr &a, const vctr &b) {
  vctr A = a - minvctr;
  vctr B = b - minvctr;
  auto sign_prod = sign(A * B);
  if (sign_prod != 0)
    return sign_prod > 0;
  return A.dist2() < B.dist2();
}

// minvctr updates here
vector<vctr> get_convex_hull(vector<vctr> arr) {
  minvctr = {INF, INF};
  for (auto v : pts) {
    if (is2plane(v - minvctr))
      minvctr = v;
  }
  vector<vctr> hull;
  sort(arr.begin(), arr.end(), cmp_convex_hull);
  for (vctr &el : arr) {
    while (hull.size() > 1 && sign((hull.back() - hull[hull.size() - 2]) * (el - hull.back())) <= 0)
      hull.pop_back();
    hull.push_back(el);
  }
  return hull;
}

\end{lstlisting}

\subsection{Задача 16}
\begin{lstlisting}[language=C++]
bool isInSameHalf(vctr p, vctr r1, vctr r2) {
  return sign((r2 - r1) % (p - r1)) >= 0;
}

dbl distPointPoint(vctr a, vctr b) {
  return (a - b).dist();
}

dbl distPointLine(vctr a, vctr l1, vctr l2) {
  line l(l1, l2);
  l.normalize();
  return abs(l.get(a));
}

dbl distPointRay(vctr a, vctr r1, vctr r2) {
  if (!isInSameHalf(a, r1, r2))
    return distPointPoint(a, r1);
  return distPointLine(a, r1, r2);
}

dbl distPointSeg(vctr a, vctr s1, vctr s2) {
  return max(distPointRay(a, s1, s2),
             distPointRay(a, s2, s1));
}

bool isIntersectionLineLine(line l1, line l2) {
  dbl znam = l1.b * l2.a - l1.a * l2.b;
  return sign(znam) != 0;
}

vctr intersectionLineLine(line l1, line l2) {
  dbl znam = l1.b * l2.a - l1.a * l2.b;
  dbl y = -(l1.c * l2.a - l2.c * l1.a) / znam;
  dbl x = -(l1.c * l2.b - l2.c * l1.b) / -znam;
  return vctr(x, y);
}

vctr getPointOnLine(line l) {
  if (sign(l.b) != 0)
    return vctr(0, -l.c / l.b);
  return vctr(-l.c / l.a, 0);
}

dbl distLineLine(vctr l1a, vctr l1b, vctr l2a, vctr l2b) {
  line l1(l1a, l1b);
  line l2(l2a, l2b);
  if (isIntersectionLineLine(l1, l2))
    return 0;
  vctr p = getPointOnLine(l1);
  l2.normalize();
  return abs(l2.get(p));
}

dbl distRayLine(vctr r1, vctr r2, vctr l1, vctr l2) {
  line r(r1, r2);
  line l(l1, l2);
  if (!isIntersectionLineLine(l, r))
    return distLineLine(r1, r2, l1, l2);
  vctr p = intersectionLineLine(l, r);
  if (isInSameHalf(p, r1, r2))
    return 0;
  return distPointLine(r1, l1, l2);
}

dbl distSegLine(vctr s1, vctr s2, vctr l1, vctr l2) {
  return max(distRayLine(s1, s2, l1, l2),
             distRayLine(s2, s1, l1, l2));
}

dbl distRayRay(vctr r1a, vctr r1b, vctr r2a, vctr r2b) {
  line r1(r1a, r1b);
  line r2(r2a, r2b);
  if (!isIntersectionLineLine(r1, r2)) {
    if (isInSameHalf(r1a, r2a, r2b) || isInSameHalf(r2a, r1a, r1b))
      return distLineLine(r1a, r1b, r2a, r2b);
    else
      return distPointPoint(r1a, r2a);
  }
  vctr p = intersectionLineLine(r1, r2);
  if (isInSameHalf(p, r1a, r1b) && isInSameHalf(p, r2a, r2b))
    return 0;
  return min(distPointRay(r1a, r2a, r2b),
             distPointRay(r2a, r1a, r1b));
}

dbl distSegRay(vctr s1, vctr s2, vctr r1, vctr r2) {
  return max(distRayRay(s1, s2, r1, r2),
             distRayRay(s2, s1, r1, r2));
}

dbl distSegSeg(vctr s1a, vctr s1b, vctr s2a, vctr s2b) {
  return max(distSegRay(s1a, s1b, s2a, s2b),
             distSegRay(s1a, s1b, s2b, s2a));
}

\end{lstlisting}

\subsection{Касательные из точки}
\begin{lstlisting}[language=C++]
pair<int, int> tangents_from_point(vector<vctr> &p, vctr &a) {
  int n = p.size();
  int logn = 31 - __builtin_clz(n);
  auto findWithSign = [&](int val) {
    int i = 0;
    for (int k = logn; k >= 0; --k) {
      int i1 = (i - (1 << k) + n) % n;
      int i2 = (i + (1 << k)) % n;
      if (sign((p[i1] - a) * (p[i] - a)) == val)
        i = i1;
      if (sign((p[i2] - a) * (p[i] - a)) == val)
        i = i2;
    }
    return i;
  };
  return {findWithSign(1), findWithSign(-1)};
}

\end{lstlisting}

\subsection{Касательные параллельные прямой}
\begin{lstlisting}[language=C++]
// find point with max signed distance to line
int tangent_parallel_line(const vector<vctr> &p, line l) {
  int n = p.size();
  int i = 0;
  int logn = 31 - __builtin_clz(n);
  for (int k = logn; k >= 0; --k) {
    int i1 = (i - (1 << k) + n) % n;
    int i2 = (i + (1 << k)) % n;
    if (l.get(p[i1]) > l.get(p[i]))
      i = i1;
    if (l.get(p[i2]) > l.get(p[i]))
      i = i2;
  }
  return i;
}

\end{lstlisting}

\subsection{Лежит ли точка в многоугольнике}
\begin{lstlisting}[language=C++]
// P starts with minvctr
bool is_point_in_poly(vctr A, vector<vctr> &P) {
  int n = P.size();
  int ind = lower_bound(P.begin(), P.end(), A, cmp_convex_hull) - P.begin();
  if (ind == n || ind == 0)
    return false;
  if (ind == 0)
    ind++;
  vctr B = A - P[ind - 1];
  vctr C = P[ind] - P[ind - 1];
  return sign(C * B) >= 0;
}

\end{lstlisting}

\subsection{Минимальная покрывающая окружность}
\begin{lstlisting}[language=C++]
mt19937 rnd(179);

circle MinDisk2(vector<vctr> &p, vctr A, vctr B, int sz) {
  circle w(A, B);
  for (int i = 0; i < sz; ++i) {
    if (w.isin(p[i]))
      continue;
    w = circle(A, B, p[i]);
  }
  return w;
}

circle MinDisk1(vector<vctr> &p, vctr A, int sz) {
  shuffle(p.begin(), p.begin() + sz, rnd);
  circle w(A, p[0]);
  for (int i = 1; i < sz; ++i) {
    if (w.isin(p[i]))
      continue;
    w = MinDisk2(p, A, p[i], i);
  }
  return w;
}

circle MinDisk(vector<vctr> &p) {
  int sz = p.size();
  if (sz == 1)
    return circle(p[0], 0);
  shuffle(p.begin(), p.end(), rnd);
  circle w(p[0], p[1]);
  for (int i = 2; i < sz; ++i) {
    if (w.isin(p[i]))
      continue;
    w = MinDisk1(p, p[i], i);
  }
  return w;
}

\end{lstlisting}

\subsection{Пересечение полуплоскостей}
\begin{lstlisting}[language=C++]
// half plane: ax+by+c > 0
// bounding box MUST have
vector<int> intersection_half_planes_inds(const vector<line> &ls) {
  int n = (int)ls.size();
  vector<int> lsi(n);
  iota(lsi.begin(), lsi.end(), 0);
  sort(lsi.begin(), lsi.end(), [&](int i, int j) {
    vctr aa(ls[i].a, ls[i].b);
    vctr bb(ls[j].a, ls[j].b);
    bool pla = is2plane(aa);
    bool plb = is2plane(bb);
    if (pla != plb)
      return pla < plb;
    return aa * bb > 0;
  });

  vector<line> st;
  vector<int> inds;
  for (int ii = 0; ii < 2 * n; ++ii) {
    int i = lsi[ii % n];
    if (st.empty()) {
      st.push_back(ls[i]);
      inds.push_back(i);
      continue;
    }
    vctr p = intersection(ls[i], st.back());
    bool pp = isparallel(ls[i], st.back());
    bool bad = false;
    while (st.size() >= 2) {
      if (!pp && sign(st[st.size() - 2].get(p)) >= 0)
        break;
      else if (pp && sign(st.back().get(ls[i].anyPoint())) <= 0) {
        bad = true;
        break;
      }
      st.pop_back();
      inds.pop_back();
      p = intersection(ls[i], st.back());
      pp = isparallel(ls[i], st.back());
    }
    if (!bad) {
      st.push_back(ls[i]);
      inds.push_back(i);
    }
  }
  vector<int> cnt(n, 0);
  for (int i : inds)
    cnt[i]++;
  vector<int> good;
  for (int i : inds) {
    if (cnt[i]-- == 2)
      good.push_back(i);
  }
  return good;
}

vector<vctr> intersection_half_planes(vector<line> &ls) {
  vector<int> inter = intersection_half_planes_inds(ls);
  int n = inter.size();
  vector<vctr> pts;
  for (int i = 0; i < n; ++i) {
    int j = (i + 1) % n;
    vctr P = intersection(ls[inter[i]], ls[inter[j]]);
    if (pts.empty() || sign(pts.back().x - P.x) != 0
         || sign(pts.back().y - P.y) != 0)
      pts.push_back(P);
  }
  return pts;
}

\end{lstlisting}

\subsection{Пересечение с окружностью}
\begin{lstlisting}[language=C++]
bool is_intersection_line_circ(line l, circle c) {
  l.normalize();
  dbl d = abs(l.get(c.cent()));
  return d < c.r - EPS;
}

vector<vctr> intersection_line_circ(line l, circle c) {
  l.normalize();
  dbl d = abs(l.get(c.cent()));
  vctr per = vctr(l.a, l.b).norm() * d;
  vctr a = c.cent() + per;
  if (sign(d - c.r) > 0)
      return {};
  if (sign(l.get(a)) != 0)
      a = c.cent() - per;
  if (sign(c.r - d) == 0)
    return {a};
  dbl k = sqrtl(c.r * c.r - d * d);
  vctr par = vctr(-l.b, l.a).norm() * k;
  return {a + par, a - par};
}

vector<vctr> intersection_circ_circ(circle a, circle b) {
  line l(2 * (b.x - a.x),
         2 * (b.y - a.y),
         b.r * b.r - a.r * a.r
             + (a.x * a.x + a.y * a.y)
             - (b.x * b.x + b.y * b.y));
  if (sign(l.a) == 0 && sign(l.b) == 0)
    return {};
  return intersection_line_circ(l, a);
}

vector<vctr> tangent_vctr_circ(vctr v, circle c) {
  dbl d = (c.cent() - v).dist();
  dbl k = sqrtl(d * d - c.r * c.r);
  circle c2(v.x, v.y, k);
  return intersection_circ_circ(c, c2);
}

\end{lstlisting}

\subsection{Проверка на пересечение отрезков}
\begin{lstlisting}[language=C++]
bool is_intersection_seg(vctr A, vctr B, vctr C, vctr D) {
  for (int i = 0; i < 2; ++i) {
    auto l1 = A.x, r1 = B.x, l2 = C.x, r2 = D.x;
    if (l1 > r1) swap(l1, r1);
    if (l2 > r2) swap(l2, r2);
    if (max(l1, l2) > min(r1, r2))
      return false;
    swap(A.x, A.y);
    swap(B.x, B.y);
    swap(C.x, C.y);
    swap(D.x, D.y);
  }
  for (int _ = 0; _ < 2; ++_) {
    auto v1 = (B - A) * (C - A);
    auto v2 = (B - A) * (D - A);
    if (sign(v1) * sign(v2) == 1)
      return false;
    swap(A, C);
    swap(B, D);
  }
  return true;
}

\end{lstlisting}

\subsection{Сумма Минковского}
\begin{lstlisting}[language=C++]
// Список вершин -> список рёбер
vector<vctr> poly_to_edges(const vector<vctr> &A) {
  vector<vctr> edg(A.size());
  for (int i = 0; i < A.size(); ++i)
    edg[i] = A[(i + 1) % A.size()] - A[i];
  return edg;
}

// A и B начинаются с минимальных вершин
vector<vctr> minkowski_sum(const vector<vctr> &A, const vector<vctr> &B) {
  auto edgA = poly_to_edges(A);
  auto edgB = poly_to_edges(B);
  vector<vctr> edgC(A.size() + B.size());
  merge(edgA.begin(), edgA.end(), edgB.begin(), edgB.end(), edgC.begin(), cmp_angle);
  vector<vctr> C(edgC.size());
  C[0] = A[0] + B[0];
  for (int i = 0; i + 1 < C.size(); ++i)
    C[i + 1] = C[i] + edgC[i];
  return C;
}

\end{lstlisting}

\subsection{Формула Эйлера}
\noindent\rule{\linewidth}{0.15mm}
\begin{itemize}
    \item $V$ -- число вершин выпуклого многогранника (планарного графа)
    \item $E$ -- число рёбер
    \item $F$ -- число граней (если планарный граф, то включая внешнюю)
\end{itemize}
Тогда $V-E+F=2$

\noindent\rule{\linewidth}{0.15mm}

\section{Графы}

\subsection{2-SAT}
\begin{lstlisting}[language=C++]
for (int i = 1; i <= n; ++i) {
  not_v[i] = i + n;
  not_v[i + n] = i;
}
for (int i = 0; i < m; ++i) {
  cin >> u >> v;
  g[not_v[v]].push_back(u);
  g[not_v[u]].push_back(v);
  rg[u].push_back(not_v[v]);
  rg[v].push_back(not_v[u]);
}
// делаем КСС, получаем comp
for (int v = 1; v <= n; ++v) {
  if (comp[v] == comp[not_v[v]]) {
    cout << "UNSATISFIABLE\n";
    return 0;
  }
}
for (int v = 1; v <= n; ++v)
  cout << (comp[v] > comp[not_v[v]] ? v : not_v[v]);
\end{lstlisting}

\subsection{l-r-capacity-maxflow}
\noindent\rule{\linewidth}{0.15mm}
\section*{Maximum flow problem with minimum capacities}

We describe how to find the maximum flow from $S'$ to $T'$ when the edges also constrain the minimum bound of the flow amount (edges have ``minimum capacities''). It can be boiled down to an ordinary max-flow problem.

Consider an edge from $u$ to $v$ whose capacity is $R$ and minimum capacity is $L$. To deal with the minimum capacity, create a new vertex $S'$ to $T'$, remove the original edge, and add edges with the following capacities:

\begin{center}
    \includegraphics[width=5cm]{/home/ilian/github/muffix-sassif-trd/assets/flow.png} \\
    \caption{Flow network example with minimum capacities}
\end{center}

Add such edges for all edges with the minimum capacities. On the resulting graph, accumulate maximum flow in the following order:

\begin{itemize}
    \item from $S'$ to $T'$
    \item from $S'$ to $T$
    \item from $S$ to $T'$
    \item from $S$ to $T$
\end{itemize}

An $S$--$T$ flow that satisfies the minimum capacities exists if and only if, for all outgoing edges from $S'$ and incoming edges to $T'$, the flow and capacity are equal. (This can be understood by corresponding the flows from $S'$ and $T'$ to the original edges.)

Alternatively, if you just want to know the existence of a flow satisfying the minimum capacities, one can add an edge from $T'$ to $S'$ with infinite capacity and consider the flow from $S'$ to $T'$ once, instead of accumulating flows four times.
\noindent\rule{\linewidth}{0.15mm}

\subsection{Венгерский алгоритм}
\begin{lstlisting}[language=C++]
pair<int, vector<int>> venger(vector<vector<int>> a) {
// ищет минимальное по стоимости
// работает только при n <= m
// a - массив весов $(n+1) \times (m+1)$
// a[0][..] = a[..][0] = 0
// возвращает ans[i] = j если взяли ребро a[i][j]
  int n = (int) a.size() - 1;
  int m = (int) a[0].size() - 1;
  vector<int> u(n + 1), v(m + 1), p(m + 1), way(m + 1);
  for (int i = 1; i <= n; ++i) {
    p[0] = i;
    int j0 = 0;
    vector<int> minv(m + 1, INF);
    vector<char> used(m + 1, false);
    do {
      used[j0] = true;
      int i0 = p[j0], delta = INF, j1;
      for (int j = 1; j <= m; ++j)
        if (!used[j]) {
          int cur = a[i0][j] - u[i0] - v[j];
          if (cur < minv[j])
            minv[j] = cur, way[j] = j0;
          if (minv[j] < delta)
            delta = minv[j], j1 = j;
        }
      for (int j = 0; j <= m; ++j)
        if (used[j])
          u[p[j]] += delta, v[j] -= delta;
        else
          minv[j] -= delta;
      j0 = j1;
    } while (p[j0] != 0);
    do {
      int j1 = way[j0];
      p[j0] = p[j1];
      j0 = j1;
    } while (j0);
  }
  int cost = -v[0];
  vector<int> ans(n + 1);
  for (int j = 1; j <= m; ++j)
    ans[p[j]] = j;
  return {cost, ans};
}
\end{lstlisting}

\subsection{Вершинная двусвязность}
\begin{lstlisting}[language=C++]
struct edge {
  int u, ind;

  bool operator<(const edge &other) const {
    return u < other.u;
  }
};

vector<int> stack_;

void paint(int v, int pr = -1) {
  used[v] = pr;
  up[v] = tin[v] = ++timer;
  for (auto e: g[v]) {
    if (e.u == pr) {
      continue;
    }
    if (!used[e.u]) {
      stack_.push_back(e.ind);
      paint(e.u, v);
      if (up[e.u] >= tin[v]) {
        ++mx_col;
        while (true) {
          int cur_edge = stack_.back();
          col[cur_edge] = mx_col;
          stack_.pop_back();
          if (cur_edge == e.ind) {
            break;
          }
        }
      }
      up[v] = min(up[v], up[e.u]);
    } else if (tin[e.u] < tin[v]) {
      stack_.push_back(e.ind);
      up[v] = min(up[v], tin[e.u]);
    } else if (up[v] > tin[e.u]) {
      up[v] = up[e.u];
    }
  }
}

signed main() {
  int n, m;
  cin >> n >> m;
  for (int i = 0; i < m; ++i) {
    int u, v;
    cin >> u >> v;
    g[u].push_back({v, i});
    g[v].push_back({u, i});
  }
  for (int v = 1; v <= n; ++v) {
    sort(all(g[v]));
  }
  for (int v = 1; v <= n; ++v) {
    if (!used[v]) {
      paint(v);
    }
  }
  for (int v = 1; v <= n; ++v) {
    int len = g[v].size();
    for (int i = 1; i < len; ++i) {
      if (col[g[v][i].ind] == 0) {
        col[g[v][i].ind] = col[g[v][i - 1].ind];
      }
    }
  }
}
\end{lstlisting}

\subsection{Диниц}
\begin{lstlisting}[language=C++]
struct edge {
  int v, f, c, ind;
};

vector<edge> g[MAXN];
pair<int, int> pred[MAXN];
int d[MAXN];
int inds[MAXN];

bool dfs(int v, int final, int W) {
  if (v == final) {
    return true;
  }
  for (int i = inds[v]; i < (int) g[v].size(); i++) {
    auto e = g[v][i];
    if (e.f + W <= e.c && d[v] + 1 == d[e.v]) {
      pred[e.v] = {v, i};
      bool flag = dfs(e.v, final, W);
      if (flag) {
        return true;
      }
      inds[v]++;
    } else {
      inds[v]++;
    }
  }
  return false;
}

bool bfs(int start, int final, int W) {
  fill(d, d + MAXN, INF);
  d[start] = 0;
  deque<int> q = {start};
  while (!q.empty()) {
    int v = q.front();
    q.pop_front();
    for (auto e : g[v]) {
      if (e.f + W <= e.c && d[e.v] > d[v] + 1) {
        d[e.v] = d[v] + 1;
        q.push_back(e.v);
      }
    }
  }
  if (d[final] == INF) {
    return false;
  }
  fill(inds, inds + MAXN, 0);
  while (dfs(start, final, W)) {
    int v = final;
    int x = INF;
    while (v != start) {
      int ind = pred[v].second;
      v = pred[v].first;
      x = min(x, g[v][ind].c - g[v][ind].f);
    }
    v = final;
    while (v != start) {
      int ind = pred[v].second;
      v = pred[v].first;
      g[v][ind].f += x;
      g[g[v][ind].v][g[v][ind].ind].f -= x;
    }
  }
  return true;
}

void Dinic(int start, int final) {
  int W = (1LL << 30);
  do {
    while (bfs(start, final, W));
    W /= 2;
  } while (W >= 1);
}

signed main() {
  int n, m;
  vector<pair<int, int>> edges;
  for (int i = 0; i < m; i++) {
    int u, v, c;
    cin >> u >> v >> c;
    edges.emplace_back(u, v);
    g[u].push_back({v, 0, c, (int) g[v].size()});
    // если ребро - ориентированно, 
    // то обратная capacity = 0
    g[v].push_back({u, 0, c, (int) g[u].size() - 1});
  }
  int start = 1, target = n;
  Dinic(start, target);
  int res = 0;
  for (auto e : g[start]) {
    res += e.f;
  }
  vector<int> cut;
  for (int i = 0; i < m; i++) {
    int u = edges[i].first, v = edges[i].second;
    if ((d[u] != INF && d[v] == INF) ||
        (d[u] == INF && d[v] != INF)) {
      cut.push_back(i + 1);
    }
  }
}
\end{lstlisting}

\subsection{КСС}
\begin{lstlisting}[language=C++]
void dfs1(int v, vector<int> &topsort) {
  used[v] = 1;
  for (auto u : g[v]) {
    if (!used[u]) {
      dfs1(u, topsort);
    }
  }
  topsort.push_back(v);
}

void dfs2(int v, int col) {
  comp[v] = col;
  for (auto u : rg[v]) {
    if (!comp[u]) {
      dfs2(u, col);
    }
  }
}

signed main() {
  vector<int> topsort;
  for (int v = 1; v <= n; ++v)
    if (!used[v])
      dfs1(v, topsort);
  reverse(all(topsort));
  for (int j = 1; j <= n; ++j)
    if (!comp[topsort[j - 1]])
      dfs2(topsort[j - 1], j);
}
\end{lstlisting}

\subsection{Минкост (Джонсон)}
\begin{lstlisting}[language=C++]

using cost_t = ll;
using flow_t = int;

const int MAXN = 10000;
const int MAXM = 25000 * 2;
const cost_t INFw = 1e12;
const flow_t INFf = 10;

struct Edge {
  int v, u;
  flow_t f, c;
  cost_t w;
};

Edge edg[MAXM];
int esz = 0;
vector<int> graph[MAXN];
ll dist[MAXN];
ll pot[MAXN];
int S, T;
int NUMV;
int pre[MAXN];
bitset<MAXN> inQ;

flow_t get_flow() {
  int v = T;
  if (pre[v] == -1)
    return 0;
  flow_t f = INFf;
  do {
    int ei = pre[v];
    Edge &e = edg[ei];
    f = min(f, e.c - e.f);
    if (f == 0)
      return 0;
    v = e.v;
  } while (v != S);
  v = T;
  do {
    int ei = pre[v];
    edg[ei].f += f;
    edg[ei ^ 1].f -= f;
    v = edg[ei].v;
  } while (v != S);
  return f;
}

void spfa() {
  fill(dist, dist + NUMV, INFw);
  dist[S] = 0;
  deque<int> Q = {S};
  inQ[S] = true;
  while (!Q.empty()) {
    int v = Q.front();
    Q.pop_front();
    inQ[v] = false;
    cost_t d = dist[v];
    for (int ei : graph[v]) {
      Edge &e = edg[ei];
      if (e.f == e.c)
        continue;
      cost_t w = e.w + pot[v] - pot[e.u];
      if (dist[e.u] <= d + w)
        continue;
      pre[e.u] = ei;
      dist[e.u] = d + w;
      if (!inQ[e.u]) {
        inQ[e.u] = true;
        Q.push_back(e.u);
      }
    }
  }
  for (int i = 0; i < NUMV; ++i)
    pot[i] += dist[i];
}

cost_t mincost() {
  spfa(); // pot[i] = 0 // or ford_bellman
  flow_t f = 0;
  while (true) {
    flow_t ff = get_flow();
    if (ff == 0)
      break;
    f += ff;
    spfa(); // or dijkstra
  }
  cost_t res = 0;
  for (int i = 0; i < esz; ++i)
    res += edg[i].f * edg[i].w;
  res /= 2;
  return res;
}

void add_edge(int v, int u, int c, int w) {
  edg[esz] = {v, u, 0, c, w};
  edg[esz + 1] = {u, v, 0, 0, -w};
  graph[v].push_back(esz);
  graph[u].push_back(esz + 1);
  esz += 2;
}

signed main() {
  ios_base::sync_with_stdio(false);
  cin.tie(nullptr);
  int n, m;
  cin >> n >> m;
  S = 0;
  T = n - 1;
  NUMV = n;
  for (int i = 0; i < m; ++i) {
    int v, u, c, w;
    cin >> v >> u >> c >> w;
    v--, u--;
    add_edge(v, u, c, w);
  }
  cost_t ans = mincost();
  cout << ans;
}


\end{lstlisting}

\subsection{Мосты}
\begin{lstlisting}[language=C++]
void dfs(int v, int par) {
  vis[v] = 1;
  up[v] = tin[v] = timer++;
  for (auto u : g[v]) {
    if (!vis[u]) {
      dfs(u, v);
      up[v] = min(up[v], up[u]);
    } else if (u != par) {
      up[v] = min(up[v], tin[u]);
    }
    if (up[u] > tin[v]) {
      bridges.emplace_back(v, u);
    }
  }
}

\end{lstlisting}

\subsection{Паросочетания}
\begin{lstlisting}[language=C++]
int dfs(int v, int c) {
  if (used[v] == c) return 0;
  used[v] = c;
  for (auto u : g[v]) {
    if (res[u] == -1) {
      res[u] = v;
      return 1;
    }
  }
  for (auto u : g[v]) {
    if (dfs(res[u], c)) {
      res[u] = v;
      return 1;
    }
  }
  return 0;
}

signed main() {
  // n - в левой доле, m - в правой
  fill(res, res + m, -1);
  for (int i = 0; i < n; ++i) {
    ans += dfs(i, i + 1);
  }
}
\end{lstlisting}

\subsection{Точки сочленения}
\begin{lstlisting}[language=C++]
void dfs(int v, int par) {
  vis[v] = 1;
  up[v] = tin[v] = timer++;
  int child = 0;
  for (auto u : g[v]) {
    if (!vis[u]) {
      dfs(u, v);
      up[v] = min(up[v], up[u]);
      if (up[u] >= tin[v] && par != -1) {
        points.insert(v);
      }
      child++;
    } else if (u != par) {
      up[v] = min(up[v], tin[u]);
    }
  }
  if (par == -1 && child >= 2) {
    points.insert(v);
  }
}

\end{lstlisting}

\subsection{Эдмондс-Карп}
\begin{lstlisting}[language=C++]
struct edge {
  int v, f, c, ind;
};

vector<edge> g[MAXN];

bool bfs(int start, int final, int W) {
  vector<int> d(MAXN, INF);
  vector<pair<int, int>> pred(MAXN);
  d[start] = 0;
  deque<int> q = {start};
  while (!q.empty()) {
    int v = q.front();
    q.pop_front();
    for (int i = 0; i < (int) g[v].size(); i++) {
      auto e = g[v][i];
      if (e.f + W <= e.c && d[e.v] > d[v] + 1) {
        d[e.v] = d[v] + 1;
        pred[e.v] = {v, i};
        q.push_back(e.v);
      }
    }
  }
  if (d[final] == INF) {
    return false;
  }
  int v = final;
  int x = INF;
  while (v != start) {
    int ind = pred[v].second;
    v = pred[v].first;
    x = min(x, g[v][ind].c - g[v][ind].f);
  }
  v = final;
  while (v != start) {
    int ind = pred[v].second;
    v = pred[v].first;
    g[v][ind].f += x;
    g[g[v][ind].v][g[v][ind].ind].f -= x;
  }
  return true;
}

signed main() {
  int n, m;
  for (int i = 0; i < m; i++) {
    int u, v, c;
    cin >> u >> v >> c;
    g[u].push_back({v, 0, c, (int) g[v].size()});
    g[v].push_back({u, 0, 0, (int) g[u].size() - 1});
  }
  int start = 1, final = n;
  int W = (1 << 30);
  do {
    while (bfs(start, final, W));
    W /= 2;
  } while (W >= 1);
  int res = 0;
  for (auto e : g[start]) {
    res += e.f;
  }
}
\end{lstlisting}

\subsection{Эйлеров цикл}
\begin{lstlisting}[language=C++]
// unconnected graph, deleting edges, set<int> g[N];
for (int v = 0; v < n; v++) {
  if (!g[v].empty()) {
    vector<int> ccl;
    vector<int> s = {v};
    while (!s.empty()) {
      int u = s.back();
      if (g[u].empty()) {
        ccl.pb(u);
        s.pop_back();
      } else {
        int u2 = *g[u].begin();
        g[u].erase(u2);
        g[u2].erase(u);
        s.pb(u2);
      }
    }
    // ccl[0] = ccl.back()
    // i.e for graph with edges $(1, 2), (1, 3), (2, 3) \to ccl = [1, 2, 3, 1]$
  }
}
\end{lstlisting}

\section{ДП}

\subsection{CHT}
\begin{lstlisting}[language=C++]
struct Line {
  ld k, b;
};

pair<ld, ld> inter(Line a, Line b) {
  ld x = (b.b - a.b) / (a.k - b.k);
  ld y = a.k * x + a.b;
  return {x, y};
}

void add_line(ld k, ld b, vector<Line> &s, vector<pair<ld, ld>> &pts) {
  while (s.size() >= 2) {
    pair<ld, ld> x1 = inter(s.back(), s[s.size() - 2]);
    pair<ld, ld> x2 = inter(s[s.size() - 2], {k, b});
    if (x1 > x2) {
      break;
    }
    pts.pop_back();
    s.pop_back();
  }
  if (!s.empty()) {
    pts.push_back(inter(s.back(), {k, b}));
  }
  s.push_back({k, b});
}

ld bin_search(vector<Line> &s, ld x) {
  int l = 0, r = s.size();
  while (l + 1 < r) {
    int m = (r + l) / 2;
    auto kek = inter(s[m - 1], s[m]);
    if (kek.first >= x) {
      l = m;
    } else {
      r = m;
    }
  }
  return s[l].k * x + s[l].b;
}
\end{lstlisting}

\subsection{Li Chao}
\begin{lstlisting}[language=C++]
// MAXIMUM
struct Line {
  int k, b;

  int f(int x) {
    return k * x + b;
  }
};

struct ST {
  vector<Line> st;

  ST(int n) {
    Line ln = {0LL, -INF};
    st.resize(4 * n, ln);
  }

  void upd(int i, int l, int r, Line ln) {
    int child = 1;
    Line ln1 = ln;
    int m = (l + r) / 2;
    if (ln.f(m) > st[i].f(m)) {
      if (ln.k < st[i].k) {
        child = 2;
      }
      ln1 = st[i];
      st[i] = ln;
    } else {
      if (st[i].k < ln.k) {
        child = 2;
      }
    }
    if (l + 1 < r) {
      if (child == 1) {
        upd(i * 2 + 1, l, m, ln1);
      } else {
        upd(i * 2 + 2, m, r, ln1);
      }
    }
  }

  int res(int i, int l, int r, int x) {
    if (l + 1 == r) {
      return st[i].f(x);
    }
    int m = (l + r) / 2;
    int val = st[i].f(x);
    if (x < m) {
      val = max(val, res(i * 2 + 1, l, m, x));
    } else {
      val = max(val, res(i * 2 + 2, m, r, x));
    }
    return val;
  }
};
\end{lstlisting}

\subsection{SOS-dp}
\begin{lstlisting}[language=C++]
// dp initial fill, a[] is given array, mb extra zeros
for (int i = 0; i < (1 << N); i++) {
  dp[i] = a[i];
}

// Classic SOS-dp, goal: dp[mask] = \sum a[submasks of mask]
for (int i = 0; i < N; i++) {
  for (int mask = 0; mask < (1 << N); mask++) {
    if ((mask >> i) & 1) {
      dp[mask] += dp[mask ^ (1 << i)];
    }
  }
}

// Overmasks SOS-dp, goal: dp[mask] = \sum a[overmasks of mask]
for (int i = 0; i < N; i++) {
  for (int mask = (1 << N) - 1; mask >= 0; mask--) {
    if (((mask >> i) & 1) == 0) {
      dp[mask] += dp[mask ^ (1 << i)];
    }
  }
}

// to inverse SOS-dp (restore original array by SOS-dp array):
// use same code, but -= instead of += in dp transitions
\end{lstlisting}

\subsection{НВП}
\begin{lstlisting}[language=C++]
// 0-indexation ({$a_0, ..., a_{n-1}$})
vector<int> lis(vector<int> a) {
  int n = (int) a.size();
  vector<int> dp(n + 1, INF), ind(n + 1), par(n + 1); // INF > all a[i] required
  ind[0] = -INF;
  dp[0] = -INF;
  for (int i = 0; i < n; i++) {
    int l = upper_bound(dp.begin(), dp.end(), a[i]) - dp.begin();
    if (dp[l - 1] < a[i] && a[i] < dp[l]) {
      dp[l] = a[i];
      ind[l] = i;
      par[i] = ind[l - 1];
    }
  }
  vector<int> ans; // exact values
  for (int l = n; l >= 0; l--) {
    if (dp[l] < INF) {
      int pi = ind[l];
      ans.resize(l);
      for (int i = 0; i < l; i++) {
        ans[i] = a[pi]; // =pi if need indices
        pi = par[pi];
      }
      reverse(ans.begin(), ans.end());
      return ans;
    }
  }
  return {};
}
\end{lstlisting}

\subsection{НОВП}
\begin{lstlisting}[language=C++]
// 1-indexation ({$0, a_1, ..., a_n$}, {$0, b_1, ..., b_m$})
vector<int> lcis(vector<int> a, vector<int> b) {
  int n = (int) a.size() - 1, m = (int) b.size() - 1;
  vector<int> dp(m + 1), dp2(m + 1), par(m + 1);
  for (int i = 1; i <= n; i++) {
    int best = 0, best_idx = 0;
    for (int j = 1; j <= m; j++) {
      dp2[j] = dp[j];
      if (a[i] == b[j]) {
        dp2[j] = max(dp2[j], best + 1);
        par[j] = best_idx;
      }
      if (a[i] > b[j] && best < dp[j]) {
        best = dp[j];
        best_idx = j;
      }
    }
    swap(dp, dp2);
  }
  int pj = 0;
  for (int j = 1; j <= m; j++) {
    if (dp[pj] < dp[j]) {
      pj = j;
    }
  }
  vector<int> ans; // exact values
  while (pj > 0) {
    ans.push_back(b[pj]);
    pj = par[pj];
  }
  reverse(ans.begin(), ans.end());
  return ans;
}
\end{lstlisting}

\section{Деревья}

\subsection{Centroid}
\begin{lstlisting}[language=C++]
int levels[MAXN];
int szs[MAXN];
int cent_par[MAXN];

int calcsizes(int v, int p) {
  int sz = 1;
  for (int u : graph[v]) {
    if (u != p && levels[u] == 0)
      sz += calcsizes(u, v);
  }
  return szs[v] = sz;
}

void centroid(int v, int lvl=1, int p=-1) {
  int sz = calcsizes(v, -1);
  int nxt = v, prv;
  while (nxt != -1) {
    prv = v, v = nxt, nxt = -1;
    for (int u : graph[v]) {
      if (u != prv && levels[u] == 0 && szs[u] * 2 >= sz)
        nxt = u;
    }
  }
  levels[v] = lvl;
  cent_par[v] = p;
  for (int u : graph[v]) {
    if (levels[u] == 0)
      centroid(u, lvl + 1, v);
  }
  // calc smth for centroid v
}
\end{lstlisting}

\subsection{HLD}
\begin{lstlisting}[language=C++]
int par[MAXN], sizes[MAXN];
int pathup[MAXN];
int tin[MAXN], tout[MAXN];
int timer;

int dfs1_hld(int v, int p) {
  par[v] = p;
  int sz = 1;
  for (int i = 0; i < graph[v].size(); ++i) {
    int u = graph[v][i];
    if (u == p) {
      swap(graph[v][i--], graph[v].back());
      graph[v].pop_back();
      continue;
    }
    sz += dfs1_hld(u, v);
  }
  return sizes[v] = sz;
}

void dfs2_hld(int v, int up) {
  tin[v] = timer++;
  pathup[v] = up;
  if (graph[v].empty()) {
    tout[v] = timer;
    return;
  }
  for (int i = 1; i < graph[v].size(); ++i) {
    if (sizes[graph[v][i]] > sizes[graph[v][0]])
      swap(graph[v][i], graph[v][0]);
  }
  dfs2_hld(graph[v][0], up);
  for (int i = 1; i < graph[v].size(); ++i)
    dfs2_hld(graph[v][i], graph[v][i]);
  tout[v] = timer;
}

bool is_ancestor(int v, int p) {
  return tin[p] <= tin[v] && tout[v] <= tout[p];
}

// get_hld полностью аналогичный
void update_hld(int v, int u, int ARG) {
  for (int _ = 0; _ < 2; ++_) {
    while (!is_ancestor(u, pathup[v])) {
      int vup = pathup[v];
      ST.update(0, 0, timer, tin[vup], tin[v] + 1, ARG);
      v = par[vup];
    }
    swap(v, u);
  }
  if (tin[v] > tin[u])
    swap(v, u);
  // v = lca
  ST.update(0, 0, timer, tin[v], tin[u] + 1, ARG);
}

signed main() {
  dfs1_hld(0, -1);
  dfs2_hld(0, 0);
  ST.build();
  // your code here
}
\end{lstlisting}

\subsection{Link-cut}
\begin{lstlisting}[language=C++]
struct Node {
  Node *ch[2];
  Node *p;
  bool rev;
  int sz;

  Node() {
    ch[0] = nullptr;
    ch[1] = nullptr;
    p = nullptr;
    rev = false;
    sz = 1;
  }
};

int size(Node *v) {
  return (v ? v->sz : 0);
}

int chnum(Node *v) {
  return v->p->ch[1] == v;
}

bool isroot(Node *v) {
  return v->p == nullptr || v->p->ch[chnum(v)] != v;
}

void push(Node *v) {
  if (v->rev) {
    if (v->ch[0])
      v->ch[0]->rev ^= 1;
    if (v->ch[1])
      v->ch[1]->rev ^= 1;
    swap(v->ch[0], v->ch[1]);
    v->rev = false;
  }
}

void pull(Node *v) {
  v->sz = size(v->ch[1]) + size(v->ch[0]) + 1;
}

void attach(Node *v, Node *p, int num) {
  if (p)
    p->ch[num] = v;
  if (v)
    v->p = p;
}

void rotate(Node *v) {
  Node *p = v->p;
  push(p);
  push(v);
  int num = chnum(v);
  Node *u = v->ch[1 - num];
  if (!isroot(v->p))
    attach(v, p->p, chnum(p));
  else
    v->p = p->p;
  attach(u, p, num);
  attach(p, v, 1 - num);
  pull(p);
  pull(v);
}

void splay(Node *v) {
  push(v);
  while (!isroot(v)) {
    if (!isroot(v->p)) {
      if (chnum(v) == chnum(v->p))
        rotate(v->p);
      else
        rotate(v);
    }
    rotate(v);
  }
}

void expose(Node *v) {
  splay(v);
  v->ch[1] = nullptr;
  pull(v);
  while (v->p != nullptr) {
    Node *p = v->p;
    splay(p);
    attach(v, p, 1);
    pull(p);
    splay(v);
  }
}

void makeroot(Node *v) {
  expose(v);
  v->rev ^= 1;
  push(v);
}

void link(Node *v, Node *u) {
  makeroot(v);
  makeroot(u);
  u->p = v;
}

void cut(Node *v, Node *u) {
  makeroot(u);
  makeroot(v);
  v->ch[1] = nullptr;
  u->p = nullptr;
}

int get(Node *v, Node *u) {
  makeroot(u);
  makeroot(v);
  Node *w = u;
  while (!isroot(w))
    w = w->p;
  return (w == v ? size(v) - 1 : -1);
}

const int MAXN = 100010;
Node *nodes[MAXN];

int main() {
  int n, q;
  cin >> n >> q;
  for (int i = 0; i < n; ++i)
    nodes[i] = new Node();
  while (q--) {
    string s;
    int a, b;
    cin >> s >> a >> b;
    a--, b--;
    if (s[0] == 'g')
      cout << get(nodes[a], nodes[b]) << '\n';
    else if (s[0] == 'l')
      link(nodes[a], nodes[b]);
    else
      cut(nodes[a], nodes[b]);
  }
}

\end{lstlisting}

\section{Другое}

\subsection{Slope trick}
\begin{lstlisting}[language=C++]
// Дан массив $a_n$. Сделать минимальное кол-во $\pm 1$, чтобы $a_n$ стал неубывающим.

void solve() {
    int n;
    cin >> n;
    vector<int> a(n);
    for (int i = 0; i < n; i++) {
        cin >> a[i];
    }
    int ans = 0;
    multiset<int> now;
    for (int i = 0; i < n; i++) {
        now.insert(a[i]);
        ans += (*now.rbegin() - a[i]);
        now.erase(now.find(*now.rbegin()));
        now.insert(a[i]);
    }
    cout << ans << '\n';
}
\end{lstlisting}

\subsection{attribute\_packed}
\begin{lstlisting}[language=C++]
struct Kek {
  int a;
  char b;
  // char[3]
  int c;
} __attribute__((packed));
// sizeof = 9 (instead of 12)
\end{lstlisting}

\subsection{custom\_bitset}
\begin{lstlisting}[language=C++]
struct custom_bitset {
    vector<uint64_t> bits;
    int64_t b, n;

    custom_bitset(int64_t _b = 0) {
        init(_b);
    }

    void init(int64_t _b) {
        b = _b;
        n = (b + 63) / 64;
        bits.assign(n, 0);
    }

    void clear() {
        b = n = 0;
        bits.clear();
    }

    void reset() {
        bits.assign(n, 0);
    }

    void _clean() {
        // Reset all bits after `b`.
        if (b != 64 * n)
            bits.back() &= (1LLU << (b - 64 * (n - 1))) - 1;
    }

    bool get(int64_t index) const {
        return bits[index / 64] >> (index % 64) & 1;
    }

    void set(int64_t index, bool value) {
        assert(0 <= index && index < b);
        bits[index / 64] &= ~(1LLU << (index % 64));
        bits[index / 64] |= uint64_t(value) << (index % 64);
    }

    // Simulates `bs |= bs << shift;`
    void or_shift(int64_t shift) {
        int64_t div = shift / 64, mod = shift % 64;

        if (mod == 0) {
            for (int64_t i = n - 1; i >= div; i--)
                bits[i] |= bits[i - div];

            return;
        }

        for (int64_t i = n - 1; i >= div + 1; i--)
            bits[i] |= bits[i - (div + 1)] >> (64 - mod) | bits[i - div] << mod;

        if (div < n)
            bits[div] |= bits[0] << mod;

        _clean();
    }

    // Simulates `bs |= bs >> shift;`
    void or_shift_down(int64_t shift) {
        int64_t div = shift / 64, mod = shift % 64;

        if (mod == 0) {
            for (int64_t i = div; i < n; i++)
                bits[i - div] |= bits[i];

            return;
        }

        for (int64_t i = 0; i < n - (div + 1); i++)
            bits[i] |= bits[i + (div + 1)] << (64 - mod) | bits[i + div] >> mod;

        if (div < n)
            bits[n - div - 1] |= bits[n - 1] >> mod;

        _clean();
    }

    int64_t find_first() const {
        for (int i = 0; i < n; i++)
            if (bits[i] != 0)
                return 64 * i + __builtin_ctzll(bits[i]);

        return -1;
    }

    custom_bitset &operator&=(const custom_bitset &other) {
        assert(b == other.b);

        for (int i = 0; i < n; i++)
            bits[i] &= other.bits[i];

        return *this;
    }

    custom_bitset &operator|=(const custom_bitset &other) {
        assert(b == other.b);

        for (int i = 0; i < n; i++)
            bits[i] |= other.bits[i];

        return *this;
    }

    custom_bitset &operator^=(const custom_bitset &other) {
        assert(b == other.b);

        for (int i = 0; i < n; i++)
            bits[i] ^= other.bits[i];

        return *this;
    }
};
\end{lstlisting}

\subsection{ordered\_set}
\begin{lstlisting}[language=C++]
#include <ext/pb_ds/assoc_container.hpp>
#include <ext/pb_ds/tree_policy.hpp>

using namespace __gnu_pbds;

typedef tree<int, null_type, less<>, rb_tree_tag, tree_order_statistics_node_update> ordered_set;

//st.find_by_order(index);
//st.order_of_key(key);

\end{lstlisting}

\subsection{pragma}
\begin{lstlisting}[language=C++]
#pragma GCC optimize("Ofast,fast-math,unroll-loops,no-stack-protector,inline")
#pragma GCC target("sse,sse2,sse3,ssse3,sse4,sse4.1,sse4.2,avx,avx2,abm,mmx,popcnt")

\end{lstlisting}

\subsection{Аллокатор Копелиовича}
\begin{lstlisting}[language=C++]
// Код вставить до инклюдов

#include <cassert>

const int MAX_MEM = 1e8; // ~100mb
int mpos = 0;
char mem[MAX_MEM];

inline void *operator new(std::size_t n) {
  assert((mpos += n) <= MAX_MEM);
  return (void *)(mem + mpos - n);
}

inline void operator delete(void *) noexcept {} // must have!
inline void operator delete(void *, std::size_t) noexcept {} // fix!!
\end{lstlisting}

\section{Математика}

\subsection{A div B < C div D}
\begin{lstlisting}[language=C++]
char sign(ll x) {
  return x < 0 ? -1 : x > 0;
}

// -1 = less, 0 = equal, 1 = greater
char compare(ll a, ll b, ll c, ll d) {
  if (a / b != c / d)
    return sign(a / b - c / d);
  a = a % b;
  c = c % d;
  if (a == 0)
    return -sign(c) * sign(d);
  if (c == 0)
    return sign(a) * sign(b);
  return is_less(d, c, b, a) * sign(a) * sign(b) * sign(c) * sign(d);
}

\end{lstlisting}

\subsection{FFT mod}
\begin{lstlisting}[language=C++]
const int MOD = 998244353; // $7 \cdot 17 \cdot 2^{23} + 1$
const int GEN = 3;
//const int MOD = 7340033; // $7 \cdot 2^{20} + 1$
//const int GEN = 5;
//const int MOD = 469762049; // $7 \cdot  2^{26} + 1$
//const int GEN = 30;

const int LOG = 20;
const int MAXN = 1 << LOG;
int tail[MAXN + 1];
int OMEGA[MAXN + 1];

int binpow(int x, int p) {
  int res = 1;
  while (p > 0) {
    if (p & 1)
      res = res * 1ll * x % MOD;
    x = x * 1ll * x % MOD;
    p >>= 1;
  }
  return res;
}

int omega(int n, int k) {
  return OMEGA[MAXN / n * k];
}

int gettail(int x, int lg) {
  return tail[x] >> (LOG - lg);
}

void calcomega() {
  long long one = binpow(GEN, (MOD - 1) / MAXN);
  OMEGA[0] = 1;
  for (int i = 1; i < MAXN; ++i) {
    OMEGA[i] = OMEGA[i - 1] * one % MOD;
  }
}

void calctail() {
  int n = MAXN;
  for (int x = 0; x < n; ++x) {
    int res = 0;
    for (int i = 0; i < LOG; ++i) {
      res += ((x >> i) & 1) << (LOG - i - 1);
    }
    tail[x] = res;
  }
}

// Without precalc, tail[], OMEGA[]
//
//long long omega(int n, int k) {
//    return binpow(GEN, (MOD - 1) / n * k);
//}
//
//int gettail(int x, int lg) {
//    int res = 0;
//    for (int i = 0; i < lg; ++i)
//        res += ((x >> i) & 1) << (lg - i - 1);
//    return res;
//}

void fft(vector<int> &A, int lg) {
  int n = 1 << lg;
  for (int i = 0; i < n; ++i) {
    int j = gettail(i, lg);
    if (i < j)
      swap(A[i], A[j]);
  }
  for (int len = 2; len <= n; len *= 2) {
    for (int i = 0; i < n; i += len) {
      for (int j = 0; j < len / 2; ++j) {
        auto v = A[i + j];
        auto u = A[i + j + len / 2] * 1ll * omega(len, j) % MOD;
        A[i + j] = (v + u) % MOD;
        A[i + j + len / 2] = (v - u + MOD) % MOD;
      }
    }
  }
}

int inverse(int x) {
  return binpow(x, MOD - 2);
}

void invfft(vector<int> &A, int lg) {
  int n = 1 << lg;
  fft(A, lg);
  for (auto &el : A)
    el = el * 1ll * inverse(n % MOD) % MOD;
  reverse(A.begin() + 1, A.end());
}

vector<int> mul(vector<int> A, vector<int> B) {
  if (A.empty() || B.empty())
    return {};
  int lg = 32 - __builtin_clz(A.size() + B.size() - 1);
  int n = 1 << lg;
  A.resize(n, 0);
  B.resize(n, 0);
  fft(A, lg);
  fft(B, lg);
  for (int i = 0; i < n; ++i)
    A[i] = A[i] * 1ll * B[i] % MOD;
  invfft(A, lg);
  return A;
}

signed main() {
  calctail(); // НЕ ЗАБЫТЬ
  calcomega(); // НЕ ЗАБЫТЬ
  // your code here
}

\end{lstlisting}

\subsection{FFT}
\begin{lstlisting}[language=C++]
const double PI = acos(-1);
const int LOG = 20;
const int MAXN = 1 << LOG;

//using comp = complex<double>;
struct comp {
  double x, y;
  comp() : x(0), y(0) {}
  comp(double x, double y) : x(x), y(y) {}
  comp(int x) : x(x), y(0) {}
  comp operator+(const comp &o) const { return {x + o.x, y + o.y}; }
  comp operator-(const comp &o) const { return {x - o.x, y - o.y}; }
  comp operator*(const comp &o) const { return {x * o.x - y * o.y, x * o.y + y * o.x}; }
  comp operator/(const int k) const { return {x / k, y / k}; }
  comp conj() const { return {x, -y}; }
};

comp OMEGA[MAXN + 10];
int tail[MAXN + 10];

comp omega(int n, int k) {
  return OMEGA[MAXN / n * k];
}

int gettail(int x, int lg) {
  return tail[x] >> (LOG - lg);
}

void calcomega() {
  for (int i = 0; i < MAXN; ++i) {
    double x = 2 * PI * i / MAXN;
    OMEGA[i] = {cos(x), sin(x)};
  }
}

void calctail() {
  tail[0] = 0;
  for (int i = 1; i < MAXN; ++i)
    tail[i] = (tail[i >> 1] >> 1) | ((i & 1) << (LOG - 1));
}

void fft(vector<comp> &A, int lg) {
  int n = A.size();
  for (int i = 0; i < n; ++i) {
    int j = gettail(i, lg);
    if (i < j)
      swap(A[i], A[j]);
  }
  for (int len = 2; len <= n; len *= 2) {
    for (int i = 0; i < n; i += len) {
      for (int j = 0; j < len / 2; ++j) {
        auto v = A[i + j];
        auto u = A[i + j + len / 2] * omega(len, j);
        A[i + j] = v + u;
        A[i + j + len / 2] = v - u;
      }
    }
  }
}

void fft2(vector<comp> &A, vector<comp> &B, int lg) {
  int n = A.size();
  vector<comp> C(n);
  for (int i = 0; i < n; ++i) {
    C[i].x = A[i].x;
    C[i].y = B[i].x;
  }
  fft(C, lg);
  C.push_back(C[0]);
  for (int i = 0; i < n; ++i) {
    A[i] = (C[i] + C[n - i].conj()) / 2;
    B[i] = (C[i] - C[n - i].conj()) / 2 * comp(0, -1);
  }
}

void invfft(vector<comp> &A, int lg) {
  int n = 1 << lg;
  fft(A, lg);
  for (auto &el : A)
    el = el / n;
  reverse(A.begin() + 1, A.end());
}

vector<int> mul(vector<int> &a, vector<int> &b) {
  if (a.empty() || b.empty())
    return {};
  int lg = 32 - __builtin_clz(a.size() + b.size() - 1);
  int n = 1 << lg;
  vector<comp> A(n, 0), B(n, 0);
  for (int i = 0; i < a.size(); ++i)
    A[i] = a[i];
  for (int i = 0; i < b.size(); ++i)
    B[i] = b[i];
//  fft2(A, B, lg);
  fft(A, lg);
  fft(B, lg);
  for (int i = 0; i < n; ++i)
    A[i] = A[i] * B[i];
  invfft(A, lg);
  vector<int> c(n);
  for (int i = 0; i < n; ++i)
    c[i] = round(A[i].x);
  while (!c.empty() && c.back() == 0)
    c.pop_back();
  return c;
}

signed main() {
  calcomega(); // НЕ ЗАБЫТЬ
  calctail(); // НЕ ЗАБЫТЬ
  // your code here
}

\end{lstlisting}

\subsection{Floor Sum}
\begin{lstlisting}[language=C++]
int floor_sum(int n, int div, int mul, int add) {
    // sum_{i=0}^{n-1} floor((add + i*mul)/div)
    int ans = 0;
    ans += (n * (n - 1) / 2) * (mul / div);
    mul %= div;
    ans += n * (add / div);
    add %= div;
    int l = mul * n + add;
    if (l >= div)
        ans += floor_sum(l / div, mul, div, l % div);
    return ans;
}
\end{lstlisting}

\subsection{Гаусс}
\begin{lstlisting}[language=C++]
vector<vector<int>> gauss(vector<vector<int>> &a) {
  int n = a.size();
  int m = a[0].size();
//  int det = 1;
  for (int col = 0, row = 0; col < m && row < n; ++col) {
    for (int i = row; i < n; ++i) {
      if (a[i][col]) {
        swap(a[i], a[row]);
        if (i != row) {
//          det *= -1;
        }
        break;
      }
    }
    if (!a[row][col])
      continue;
    for (int i = 0; i < n; ++i) {
      if (i != row && a[i][col]) {
        int val = a[i][col] * inv(a[row][col]) % mod;
        for (int j = col; j < m; ++j) {
          a[i][j] -= val * a[row][j];
          a[i][j] %= mod;
        }
      }
    }
    ++row;
  }
//  for (int i = 0; i < n; ++i) det = (det * a[i][i]) % mod;
//  det = (det % mod + mod) % mod;
// result in (-mod, mod)
  return a;
}

pair<int, vector<int>> sle(vector<vector<int>> a, vector<int> b) {
  int n = a.size();
  int m = a[0].size();
  assert(n == b.size());
  for (int i = 0; i < n; ++i) {
    a[i].push_back(b[i]);
  }
  a = gauss(a);
  vector<int> x(m, 0);
  for (int i = n - 1; i >= 0; --i) {
    int leftmost = m;
    for (int j = 0; j < m; ++j) {
      if (a[i][j] != 0) {
        leftmost = j;
        break;
      }
    }
    if (leftmost == m && a[i].back() != 0) return {-1, {}};
    if (leftmost == m) continue;
    int val = a[i].back();
    for (int j = m - 1; j > leftmost; --j) {
      val -= a[i][j] * x[j];
      val %= mod;
    }
    x[leftmost] = (val * inv(a[i][leftmost]) % mod + mod) % mod;
  }
  return {1, x};
}

vector<bitset<N>> gauss_bit(vector<bitset<N>> a, int m) {
  int n = a.size();
  for (int col = 0, row = 0; col < m && row < n; ++col) {
    for (int i = row; i < n; ++i) {
      if (a[i][col]) {
        swap(a[i], a[row]);
        break;
      }
    }
    if (!a[row][col])
      continue;
    for (int i = 0; i < n; ++i)
      if (i != row && a[i][col])
        a[i] ^= a[row];
    ++row;
  }
  return a;
}
\end{lstlisting}

\subsection{Диофантовы уравнения}
\begin{lstlisting}[language=C++]
pair<int, int> ext_gcd(int a, int b) {
  int x1 = 1, y1 = 0, x2 = 0, y2 = 1;
  while (b) {
    int k = a / b;
    x1 = x1 - x2 * k;
    y1 = y1 - y2 * k;
    swap(x1, x2);
    swap(y1, y2);
    a %= b;
    swap(a, b);
  }
  return {x1, y1};
}

bool cool_ext_gcd(int a, int b, int c, int &x, int &y) {
  if (b == 0) {
    y = 0;
    if (a == 0) {
      x = 0;
      return c == 0;
    } else {
      x = c / a;
      return c % a == 0;
    }
  }
  auto [x0, y0] = ext_gcd(a, b);
  int g = x0 * a + y0 * b;
  if (c % g != 0)
    return false;
  x0 *= c / g;
  y0 *= c / g;
  int t = b / g;
  int k = (-x0) / t;
  if (x0 + t * k < 0)
    k += t / abs(t);
  x = x0 + t * k;
  y = y0 - (a / g) * k;
  return true;
}
\end{lstlisting}

\subsection{КТО}
\begin{lstlisting}[language=C++]
// x = a_i % p_i
vector<vector<int>> r(k, vector<int>(k));
for (int i = 0; i < k; ++i)
  for (int j = 0; j < k; ++j)
    if (i != j)
      r[i][j] = binpow(p[i] % p[j], p[j] - 2, p[j]);
vector<int> x(k);
for (int i = 0; i < k; ++i) {
  x[i] = a[i];
  for (int j = 0; j < i; ++j) {
    x[i] = r[j][i] * (x[i] - x[j]);
    x[i] = x[i] % p[i];
    if (x[i] < 0) x[i] += p[i];
  }
}
int ans = 0;
for (int i = 0; i < k; ++i) {
  int val = x[i];
  for (int j = 0; j < i; ++j) val *= p[j];
  ans += val;
}
\end{lstlisting}

\subsection{Код Грея}
\begin{lstlisting}[language=C++]
for (int i = 0; i < (1 << n); i++) {
  gray[i] = i ^ (i >> 1);
}
\end{lstlisting}

\subsection{Линейное решето}
\begin{lstlisting}[language=C++]
const int N = 10000000;
int lp[N + 1];
vector<int> pr;
for (int i = 2; i <= N; ++i) {
  if (lp[i] == 0) {
    lp[i] = i;
    pr.push_back(i);
  }
  for (int j = 0; j < (int) pr.size() && pr[j] <= lp[i] && i * pr[j] <= N; ++j)
    lp[i * pr[j]] = pr[j];
}
\end{lstlisting}

\subsection{Миллер Рабин}
\begin{lstlisting}[language=C++]
// assuming '#define int long long' is ON (replace 'int' with 'long long' if not)
// works for all n < 2^64
const int MAGIC[7] = {2, 325, 9375, 28178, 450775, 9780504, 1795265022};

int bpow(__int128 a, int x, int mod) {
  a %= mod;
  __int128 ans = 1;
  while (x) {
    if (x % 2) {
      ans *= a;
      ans %= mod;
    }
    a *= a;
    a %= mod;
    x /= 2;
  }
  return (int) ans;
}

bool is_prime(int n) {
  if (n == 1) return false;
  if (n <= 3) return true;
  if (n % 2 == 0 || n % 3 == 0) return false;
  int s = __builtin_ctzll(n - 1), d = n >> s; // $n-1 = 2^s \cdot d$
  for (auto a : MAGIC) {
    if (a % n == 0) {
      continue;
    }
    int x = bpow(a, d, n);
    for (int _ = 0; _ < s; _++) {
      int y = bpow(x, 2, n);
      if (y == 1 && x != 1 && x != n - 1) {
        return false;
      }
      x = y;
    }
    if (x != 1) {
      return false;
    }
  }
  return true;
}
\end{lstlisting}

\subsection{Ро-Поллард}
\begin{lstlisting}[language=C++]
typedef long long ll;

ll mult(ll a, ll b, ll mod) {
  return (__int128)a * b % mod;
}

ll f(ll x, ll c, ll mod) {
  return (mult(x, x, mod) + c) % mod;
}

ll rho(ll n, ll x0=2, ll c=1) {
  ll x = x0;
  ll y = x0;
  ll g = 1;
  while (g == 1) {
    x = f(x, c, n);
    y = f(y, c, n);
    y = f(y, c, n);
    g = gcd(abs(x - y), n);
  }
  return g;
}

mt19937_64 rnd(time(nullptr));

void factor(int n, vector<int> &pr) {
  if (n == 4) {
    factor(2, pr);
    factor(2, pr);
    return;
  }
  if (n == 1) {
    return;
  }
  if (is_prime(n)) {
    pr.push_back(n);
    return;
  }
  int d = rho(n, rnd() % (n - 2) + 2, rnd() % 3 + 1);
  factor(n / d, pr);
  factor(d, pr);
}
\end{lstlisting}

\section{Строки}

\subsection{Z-функция}
\begin{lstlisting}[language=C++]
vector<int> z_func(string s) {
  int n = s.size();
  vector<int> z(n, 0);
  z[0] = n;
  int l = 0, r = 0;
  for (int i = 1; i < n; i++) {
    if (i < r) {
      z[i] = min(z[i - l], r - i);
    }
    while (i + z[i] < n && s[z[i]] == s[i + z[i]]) {
      z[i]++;
    }
    if (i + z[i] > r) {
      l = i;
      r = i + z[i];
    }
  }
  return z;
}
\end{lstlisting}

\subsection{eertree}
\begin{lstlisting}[language=C++]
int len[MAXN], suf[MAXN];
int go[MAXN][ALPH];
char s[MAXN];

int n, last, sz;

void init() {
  n = 0, last = 0;
  s[n++] = -1;
  suf[0] = 1; // root of suflink tree = 1
  len[1] = -1;
  sz = 2;
}

int get_link(int v) {
  while (s[n - len[v] - 2] != s[n - 1])
    v = suf[v];
  return v;
}

void add_char(char c) {
  c -= 'a';
  s[n++] = c;
  last = get_link(last);
  if (!go[last][c]) {
    len[sz] = len[last] + 2;
    suf[sz] = go[get_link(suf[last])][c];
    go[last][c] = sz++;
  }
  last = go[last][c]; // cur v = last
}
\end{lstlisting}

\subsection{Ахо-Корасик}
\begin{lstlisting}[language=C++]
int go[MAXN][ALPH];
vector<int> term[MAXN];
int par[MAXN], suf[MAXN];
char par_c[MAXN];
vector<int> g[MAXN];

int cntv = 1;

void add(string &s) {
  static int cnt_s = 1;
  int v = 0;
  for (char el: s) {
    if (go[v][el - 'a'] == 0) {
      go[v][el - 'a'] = cntv;
      par[cntv] = v;
      par_c[cntv] = el;
      cntv++;
    }
    v = go[v][el - 'a'];
  }
  term[v].push_back(cnt_s++);
}

void bfs() {
  deque<int> q = {0};
  while (!q.empty()) {
    int v = q.front();
    q.pop_front();
    if (v > 0) {
      if (par[v] == 0) {
        suf[v] = 0;
      } else {
        suf[v] = go[suf[par[v]]][par_c[v] - 'a'];
      }
      g[suf[v]].push_back(v);
    }
    for (int c = 0; c < 26; c++) {
      if (go[v][c] == 0) {
        go[v][c] = go[suf[v]][c];
      } else {
        q.push_back(go[v][c]);
      }
    }
  }
}
\end{lstlisting}

\subsection{Муффиксный Сассив}
\begin{lstlisting}[language=C++]
vector<int> build_suff_arr(string &s) {
  // Remove, if you want to sort cyclic shifts
  s += (char) (1);
  int n = s.size();
  vector<int> a(n);
  iota(all(a), 0);
  stable_sort(all(a), [&](int i, int j) {
      return s[i] < s[j];
  });
  vector<int> c(n);
  int cc = 0;
  for (int i = 0; i < n; i++) {
    if (i == 0 || s[a[i]] != s[a[i - 1]])
      c[a[i]] = cc++;
    else
      c[a[i]] = c[a[i - 1]];
  }
  for (int L = 1; L < n; L *= 2) {
    vector<int> cnt(n);
    for (auto i: c) cnt[i]++;
    if (*min_element(all(cnt)) > 0) break;
    vector<int> pref(n);
    for (int i = 1; i < n; i++)
      pref[i] = pref[i - 1] + cnt[i - 1];
    vector<int> na(n);
    for (int i = 0; i < n; i++) {
      int pos = (a[i] - L + n) % n;
      na[pref[c[pos]]++] = pos;
    }
    a = na;
    vector<int> nc(n);
    cc = 0;
    for (int i = 0; i < n; i++) {
      if (i == 0 || c[a[i]] != c[a[i - 1]] ||
          c[(a[i] + L) % n] != c[(a[i - 1] + L) % n])
        nc[a[i]] = cc++;
      else
        nc[a[i]] = nc[a[i - 1]];
    }
    c = nc;
  }
  // Remove, if you want to sort cyclic shifts
  a.erase(a.begin());
  s.pop_back();
  return a;
}

vector<int> kasai(string s, vector<int> sa) {
  // lcp[i] = lcp(sa[i], sa[i + 1])
  int n = s.size(), k = 0;
  vector<int> lcp(n, 0);
  vector<int> rank(n, 0);
  for (int i = 0; i < n; i++) rank[sa[i]] = i;
  for (int i = 0; i < n; i++, k ? k-- : 0) {
    if (rank[i] == n - 1) {
      k = 0;
      continue;
    }
    int j = sa[rank[i] + 1];
    while (i + k < n && j + k < n && s[i + k] == s[j + k]) k++;
    lcp[rank[i]] = k;
  }
  return lcp;
}
\end{lstlisting}

\subsection{Префикс-функция}
\begin{lstlisting}[language=C++]
vector<int> prefix_func(string s) {
  int n = s.size();
  vector<int> pref(n, 0);
  int ans = 0;
  for (int i = 1; i < n; i++) {
    while (ans > 0 && s[ans] != s[i]) {
      ans = pref[ans - 1];
    }
    if (s[i] == s[ans]) {
      ans++;
    }
    pref[i] = ans;
  }
  return pref;
}
\end{lstlisting}

\subsection{Суффиксный автомат}
\begin{lstlisting}[language=C++]
// Суфавтомат с подсчётом кол-ва различных подстрок

const int SIGMA = 26;
int ans = 0;

struct Node {
  int go[SIGMA];
  int s, p;
  int len;

  Node() {
    fill(go, go + SIGMA, -1);
    s = -1, p = -1;
    len = 0;
  }
};

int add(int A, int ch, vector<Node> &sa) {
  int B = sa.size();
  sa.emplace_back();
  sa[B].p = A;
  sa[B].s = 0;
  sa[B].len = sa[A].len + 1;
  for (; A != -1; A = sa[A].s) {
    if (sa[A].go[ch] == -1) {
      sa[A].go[ch] = B;
      continue;
    }
    int C = sa[A].go[ch];
    if (sa[C].p == A) {
      sa[B].s = C;
      break;
    }
    int D = sa.size();
    sa.emplace_back();
    sa[D].s = sa[C].s;
    sa[D].p = A;
    sa[D].len = sa[A].len + 1;
    sa[C].s = D;
    sa[B].s = D;
    copy(sa[C].go, sa[C].go + SIGMA, sa[D].go);
    for (; A != -1 && sa[A].go[ch] == C; A = sa[A].s)
      sa[A].go[ch] = D;
    break;
  }
  ans += sa[B].len - sa[sa[B].s].len;
  return B;
}

signed main() {
  string s;
  cin >> s;
  vector<Node> sa(1);
  int A = 0;
  for (char c : s)
    A = add(A, c - 'a', sa);
  cout << ans;
}
\end{lstlisting}

\section{Структуры данных}

\subsection{Disjoint Sparse Table}
\begin{lstlisting}[language=C++]
int tree[LOG][MAXN];
int floorlog2[MAXN]; // i ? (31 - __builtin_clz(i)) : 0

void build(vector<int> &a) {
  int n = a.size();
  copy(a.begin(), a.end(), tree[0]);
  for (int lg = 1; lg < LOG; ++lg) {
    int len = 1 << lg;
    auto &lvl = tree[lg];
    for (int m = len; m < n; m += len * 2) {
      lvl[m - 1] = a[m - 1];
      lvl[m] = a[m];
      for (int i = m - 2; i >= m - len; --i)
        lvl[i] = min(lvl[i + 1], a[i]);
      for (int i = m + 1; i < m + len && i < n; ++i)
        lvl[i] = min(lvl[i - 1], a[i]);
    }
  }
  for (int i = 2; i <= n; ++i)
    floorlog2[i] = floorlog2[i / 2] + 1;
}

// a[l..r)
int get(int l, int r) {
  r--;
  int i = floorlog2[l ^ r];
  return min(tree[i][l], tree[i][r]);
}

\end{lstlisting}

\subsection{Segment Tree Beats}
\begin{lstlisting}[language=C++]
// min=, sum
struct ST {
  vector<int> st, mx, mx_cnt, sec_mx;

  ST(int n) {
    st.resize(n * 4, 0);
    mx.resize(n * 4, 0);
    mx_cnt.resize(n * 4, 0);
    sec_mx.resize(n * 4, 0);
    build(0, 0, n);
  }

  void upd_from_children(int v) {
    st[v] = st[v * 2 + 1] + st[v * 2 + 2];
    mx[v] = max(mx[v * 2 + 1], mx[v * 2 + 2]);
    mx_cnt[v] = 0;
    sec_mx[v] = max(sec_mx[v * 2 + 1], sec_mx[v * 2 + 2]);
    if (mx[v * 2 + 1] == mx[v]) {
      mx_cnt[v] += mx_cnt[v * 2 + 1];
    } else {
      sec_mx[v] = max(sec_mx[v], mx[v * 2 + 1]);
    }
    if (mx[v * 2 + 2] == mx[v]) {
      mx_cnt[v] += mx_cnt[v * 2 + 2];
    } else {
      sec_mx[v] = max(sec_mx[v], mx[v * 2 + 2]);
    }
  }

  void build(int i, int l, int r) {
    if (l + 1 == r) {
      st[i] = mx[i] = 0;
      mx_cnt[i] = 1;
      sec_mx[i] = -INF;
      return;
    }
    int m = (r + l) / 2;
    build(i * 2 + 1, l, m);
    build(i * 2 + 2, m, r);
    upd_from_children(i);
  }

  void push_min_eq(int v, int val) {
    if (mx[v] > val) {
      st[v] -= (mx[v] - val) * mx_cnt[v];
      mx[v] = val;
    }
  }

  void push(int i) {
    push_min_eq(i * 2 + 1, mx[i]);
    push_min_eq(i * 2 + 2, mx[i]);
  }

  void update(int i, int l, int r, int ql, int qr, int val) {
    if (mx[i] <= val) {
      return;
    }
    if (ql == l && qr == r && sec_mx[i] < val) {
      push_min_eq(i, val);
      return;
    }
    push(i);
    int m = (r + l) / 2;
    if (qr <= m) {
      update(i * 2 + 1, l, m, ql, qr, val);
    } else if (ql >= m) {
      update(i * 2 + 2, m, r, ql, qr, val);
    } else {
      update(i * 2 + 1, l, m, ql, m, val);
      update(i * 2 + 2, m, r, m, qr, val);
    }
    upd_from_children(i);
  }

  int sum(int i, int l, int r, int ql, int qr) {
    if (l == ql && r == qr) {
      return st[i];
    }
    push(i);
    int m = (r + l) / 2;
    if (qr <= m) {
      return sum(i * 2 + 1, l, m, ql, qr);
    }
    if (ql >= m) {
      return sum(i * 2 + 2, m, r, ql, qr);
    }
    return sum(i * 2 + 1, l, m, ql, m) + sum(i * 2 + 2, m, r, m, qr);
  }
};
\end{lstlisting}

\subsection{ДД по неявному}
\begin{lstlisting}[language=C++]
pair<Node *, Node *> split(Node *t, int k) {
  if (!now)
    return {nullptr, nullptr};
  int szl = size(t->l);
  if (k <= szl) {
    auto [l, r] = split(t->l, k);
    t->l = r;
    pull(t);
    return {l, t};
  } else {
    auto [l, r] = split(t->r, k - szl - 1);
    t->r = l;
    pull(t);
    return {t, r};
  }
}

Node *merge(Node *l, Node *r) {
  if (!l)
    return r;
  if (!r)
    return l;
  if (l->y < r->y) {
    l->r = merge(l->r, r);
    pull(l);
    return l;
  } else {
    r->l = merge(l, r->l);
    pull(r);
    return r;
  }
}

void insert(Node *&root, int pos, int val) {
  Node *new_v = new Node(val);
  auto [l, r] = split(root, pos);
  root = merge(merge(l, new_v), r);
}

void erase(Node *&root, int pos) {
  auto [lm, r] = split(root, pos + 1);
  auto [l, m] = split(lm, pos);
  root = merge(l, r);
}


int sum(Node *v) {
  return v ? v->sm : 0;
}

// query [l, r)
int query(Node *&root, int ql, int qr) {
  auto [lm, r] = split(root, qr);
  auto [l, m] = split(lm, ql);
  int res = sum(m);
  root = merge(merge(l, m), r);
  return res;
}
\end{lstlisting}

\subsection{ДД}
\begin{lstlisting}[language=C++]
pair<Node *, Node *> split(Node *t, int x) {
  if (!t)
    return {nullptr, nullptr};
  if (x <= t->x) {
    auto [l, r] = split(t->l, x);
    t->l = r;
    pull(t);
    return {l, t};
  } else {
    auto [l, r] = split(t->r, x);
    t->r = l;
    pull(t);
    return {t, r};
  }
}

Node *merge(Node *l, Node *r) {
  if (!l)
    return r;
  if (!r)
    return l;
  if (l->y < r->y) {
    l->r = merge(l->r, r);
    pull(l);
    return l;
  } else {
    r->l = merge(l, r->l);
    pull(r);
    return r;
  }
}

void insert(Node *&root, int val) {
  Node *new_v = new Node(val);
  auto [l, r] = split(root, val);
  root = merge(merge(l, new_v), r);
}

void erase(Node *&root, int val) {
  auto [lm, r] = split(root, val + 1);
  auto [l, m] = split(lm, val);
  root = merge(l, r);
}

int sum(Node *v) {
  return v ? v->sm : 0;
}

// query [l, r)
int query(Node *&root, int ql, int qr) {
  auto [lm, r] = split(root, qr);
  auto [l, m] = split(lm, ql);
  int res = sum(m);
  root = merge(merge(l, m), r);
  return res;
}
\end{lstlisting}

\subsection{Персистентное ДД по неявному}
\begin{lstlisting}[language=C++]
mt19937 rnd(228);

struct Node;
int size(Node *);
int sum(Node *);

struct Node {
  Node *l, *r;
  int val, sz, sm;

  Node(int val) : val(val), sz(1), sm(val) {
    l = r = nullptr;
  }
  Node(int val, Node *l, Node *r) : val(val), l(l), r(r) {
    sz = 1 + size(l) + size(r);
    sm = val + sum(l) + sum(r);
  }
};

int size(Node *v) {
  return v ? v->sz : 0;
}

int sum(Node *v) {
  return v ? v->sm : 0;
}

pair<Node *, Node *> split(Node *t, int x) {
  if (!t)
    return {nullptr, nullptr};
  int lsz = size(t->l);
  if (lsz >= x) {
    auto [l, r] = split(t->l, x);
    auto v = new Node(t->val, r, t->r);
    return {l, v};
  } else {
    auto [l, r] = split(t->r, x - lsz - 1);
    auto v = new Node(t->val, t->l, l);
    return {v, r};
  }
}

bool chooseleft(int lsz, int rsz) {
  return rnd() % (lsz + rsz) < lsz;
}

Node *merge(Node *l, Node *r) {
  if (!l)
    return r;
  if (!r)
    return l;
  if (chooseleft(l->sz, r->sz)) {
    auto rr = merge(l->r, r);
    auto v = new Node(l->val, l->l, rr);
    return v;
  } else {
    auto ll = merge(l, r->l);
    auto v = new Node(r->val, ll, r->r);
    return v;
  }
}

Node *insert(Node *root, int pos, int val) {
  Node *new_v = new Node(val);
  auto [l, r] = split(root, pos);
  return merge(merge(l, new_v), r);
}

Node *erase(Node *root, int pos) {
  auto [lm, r] = split(root, pos + 1);
  auto [l, m] = split(lm, pos);
  return merge(l, r);
}

// query [l, r)
pair<int, Node *> query(Node *root, int ql, int qr) {
  auto [lm, r] = split(root, qr);
  auto [l, m] = split(lm, ql);
  int res = sum(m);
  auto new_root = merge(merge(l, m), r);
  return {res, new_root};
}

\end{lstlisting}

\subsection{Персистентное ДО}
\begin{lstlisting}[language=C++]
// left: v ? v->l : nullptr (same for right)
// sum: v ? v->sm : 0

// v can be nullptr. returns new root of subtree
Node *update(Node *v, int l, int r, int qi, int qx) {
  if (qi < l || r <= qi)
    return v;
  if (l + 1 == r)
    return new Node(qx);
  int m = (l + r) / 2;
  Node *u = new Node();
  u->l = update(left(v), l, m, qi, qx);
  u->r = update(right(v), m, r, qi, qx);
  u->sm = sum(u->l) + sum(u->r);
  return u;
}

int get(Node *v, int l, int r, int ql, int qr) {
  if (!v || qr <= l || r <= ql)
    return 0;
  if (ql <= l && r <= qr)
    return v->sm;
  int m = (l + r) / 2;
  auto a = get(v->l, l, m, ql, qr);
  auto b = get(v->r, m, r, ql, qr);
  return a + b;
}

\end{lstlisting}

\subsection{Спарсы}
\begin{lstlisting}[language=C++]
int tree[LOG][MAXN];
int floorlog2[MAXN]; // i ? (31 - __builtin_clz(i)) : 0

void build(vector<int> &a) {
  int n = a.size();
  copy(a.begin(), a.end(), tree[0]);
  for (int i = 1; i < LOG; ++i) {
    int len = 1 << (i - 1);
    for (int j = 0; j + len < n; ++j)
      tree[i][j] = min(tree[i - 1][j], tree[i - 1][j + len]);
  }
  for (int i = 2; i <= n; ++i)
    floorlog2[i] = floorlog2[i / 2] + 1;
}

// min a[l..r)
int get(int l, int r) {
  int i = floorlog2[r - l];
  return min(tree[i][l], tree[i][r - (1 << i)]);
}

\end{lstlisting}

\subsection{Фенвик (+ на отрезке)}
\begin{lstlisting}[language=C++]
// a[l..r) += x
void update(int l, int r, int x) {
  T1.add(l, x);
  T1.add(r, -x);
  T2.add(l, -x * l);
  T2.add(r, x * r);
}

// sum a[0..pos)
int rsq(int pos) {
  return T1.rsq(pos) * pos + T2.rsq(pos);
}

// sum a[l..r)
int sum(int l, int r) {
  return rsq(r) - rsq(l);
}

\end{lstlisting}

\subsection{Фенвик}
\begin{lstlisting}[language=C++]
// Нумерация с 0

struct Fenwick {
  int n;
  vector<int> f;
 
  Fenwick(int n) : n(n) {
    f.resize(n + 1);
  }
 
  // a[i] += x
  void add(int i, int x) {
    for (++i; i <= n; i += i & -i)
      f[i] += x;
  }
 
  // sum a[0..i)
  int get(int i) {
    int ans = 0;
    for (; i > 0; i -= i & -i)
      ans += f[i];
    return ans;
  }
 
  // a[..] > 0; find max k: sum a[0..k) <= x
  int max_not_more(int x) {
    int cur = 0;
    for (int i = 20; i >= 0; --i) {
      int len = 1 << i;
      if (cur + len <= n && f[cur + len] <= x) {
        cur += len;
        x -= f[cur];
      }
    }
    return cur;
  }
};

// sum a[x1..x2)[y1..y2)[z1..x2)
int sum_3d(int x1, int x2, int y1, int y2, int z1, int z2) {
  int ans = get(x2, y2, z2);
  ans -= get(x1, y2, z2) + get(x2, y1, z2) + get(x2, y2, z1);
  ans += get(x1, y1, z2) + get(x1, y2, z1) + get(x2, y1, z1);
  ans -= get(x1, y1, z1);
  return ans;
}
\end{lstlisting}


\end{multicols*}
\end{document}


