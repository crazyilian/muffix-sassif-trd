\documentclass[9pt,a4paper,landscape,twosided]{extarticle}

\usepackage[T2A]{fontenc}
\usepackage[T1]{fontenc}
\usepackage[utf8]{inputenc}
\usepackage[english, russian]{babel}
\usepackage{listings}
\usepackage[usenames,dvipsnames]{color}
\usepackage{amsmath}
\usepackage{verbatim}
\usepackage{hyperref}
\usepackage{color}
\usepackage{geometry}
\usepackage{multicol}
\usepackage{graphicx}
\usepackage{amssymb}

\geometry{verbose,landscape,a4paper,tmargin=1.1cm,bmargin=0.7cm,lmargin=0.6cm,rmargin=0.6cm}

\usepackage{listings}
\usepackage{color}

\definecolor{dkgreen}{rgb}{0,0.6,0}
\definecolor{gray}{rgb}{0.5,0.5,0.5}
\definecolor{mauve}{rgb}{0.58,0,0.82}

%% Golang definition for listings
%% http://github.io/julienc91/lstlistings-golang
%%
\lstdefinelanguage{Golang}%
  {morekeywords=[1]{package,import,func,type,struct,return,defer,panic,%
     recover,select,var,const,iota,},%
   morekeywords=[2]{string,uint,uint8,uint16,uint32,uint64,int,int8,int16,%
     int32,int64,bool,float32,float64,complex64,complex128,byte,rune,uintptr,%
     error,interface},%
   morekeywords=[3]{map,slice,make,new,nil,len,cap,copy,close,true,false,%
     delete,append,real,imag,complex,chan,},%
   morekeywords=[4]{for,break,continue,range,go,goto,switch,case,fallthrough,if,%
     else,default,},%
   morekeywords=[5]{Println,Printf,Error,Print,},%
   sensitive=true,%
   morecomment=[l]{//},%
   morecomment=[s]{/*}{*/},%
   morestring=[b]',%
   morestring=[b]",%
   morestring=[s]{`}{`},%
   }


\lstset{frame=tb,
    language=C++,
    aboveskip=1mm,
    belowskip=1mm,
    showstringspaces=true,
    columns=flexible,
    keepspaces=true,
    basicstyle={\fontsize{9pt}{9pt}\ttfamily},
    numbers=none,
    numberstyle=\tiny\color{gray},
    keywordstyle=\color{blue},
    commentstyle=\color{dkgreen},
    stringstyle=\color{mauve},
    escapebegin=\color{dkgreen},
    breaklines=true,
    breakatwhitespace=false,
    tabsize=2,
    inputencoding = utf8,  % Input encoding
    extendedchars = true,  % Extended ASCII
    mathescape    = true   % Mathematical expressions between $
    captionpos    = b,     % Caption position
    literate      =        % Support additional characters
        {á}{{\'a}}1  {é}{{\'e}}1  {í}{{\'i}}1 {ó}{{\'o}}1  {ú}{{\'u}}1
        {Á}{{\'A}}1  {É}{{\'E}}1  {Í}{{\'I}}1 {Ó}{{\'O}}1  {Ú}{{\'U}}1
        {à}{{\`a}}1  {è}{{\`e}}1  {ì}{{\`i}}1 {ò}{{\`o}}1  {ù}{{\`u}}1
        {À}{{\`A}}1  {È}{{\'E}}1  {Ì}{{\`I}}1 {Ò}{{\`O}}1  {Ù}{{\`U}}1
        {ä}{{\"a}}1  {ë}{{\"e}}1  {ï}{{\"i}}1 {ö}{{\"o}}1  {ü}{{\"u}}1
        {Ä}{{\"A}}1  {Ë}{{\"E}}1  {Ï}{{\"I}}1 {Ö}{{\"O}}1  {Ü}{{\"U}}1
        {â}{{\^a}}1  {ê}{{\^e}}1  {î}{{\^i}}1 {ô}{{\^o}}1  {û}{{\^u}}1
        {Â}{{\^A}}1  {Ê}{{\^E}}1  {Î}{{\^I}}1 {Ô}{{\^O}}1  {Û}{{\^U}}1
        {œ}{{\oe}}1  {Œ}{{\OE}}1  {æ}{{\ae}}1 {Æ}{{\AE}}1  {ß}{{\ss}}1
        {ç}{{\c c}}1 {Ç}{{\c C}}1 {ø}{{\o}}1  {å}{{\r a}}1 {Å}{{\r A}}1
        {ñ}{{\~n}}1  {Ñ}{{\~N}}1  {¿}{{?`}}1  {¡}{{!`}}1
        {а}{{\selectfont\char224}}1
        {б}{{\selectfont\char225}}1
        {в}{{\selectfont\char226}}1
        {г}{{\selectfont\char227}}1
        {д}{{\selectfont\char228}}1
        {е}{{\selectfont\char229}}1
        {ё}{{\"e}}1
        {ж}{{\selectfont\char230}}1
        {з}{{\selectfont\char231}}1
        {и}{{\selectfont\char232}}1
        {й}{{\selectfont\char233}}1
        {к}{{\selectfont\char234}}1
        {л}{{\selectfont\char235}}1
        {м}{{\selectfont\char236}}1
        {н}{{\selectfont\char237}}1
        {о}{{\selectfont\char238}}1
        {п}{{\selectfont\char239}}1
        {р}{{\selectfont\char240}}1
        {с}{{\selectfont\char241}}1
        {т}{{\selectfont\char242}}1
        {у}{{\selectfont\char243}}1
        {ф}{{\selectfont\char244}}1
        {х}{{\selectfont\char245}}1
        {ц}{{\selectfont\char246}}1
        {ч}{{\selectfont\char247}}1
        {ш}{{\selectfont\char248}}1
        {щ}{{\selectfont\char249}}1
        {ъ}{{\selectfont\char250}}1
        {ы}{{\selectfont\char251}}1
        {ь}{{\selectfont\char252}}1
        {э}{{\selectfont\char253}}1
        {ю}{{\selectfont\char254}}1
        {я}{{\selectfont\char255}}1
        {А}{{\selectfont\char192}}1
        {Б}{{\selectfont\char193}}1
        {В}{{\selectfont\char194}}1
        {Г}{{\selectfont\char195}}1
        {Д}{{\selectfont\char196}}1
        {Е}{{\selectfont\char197}}1
        {Ё}{{\"E}}1
        {Ж}{{\selectfont\char198}}1
        {З}{{\selectfont\char199}}1
        {И}{{\selectfont\char200}}1
        {Й}{{\selectfont\char201}}1
        {К}{{\selectfont\char202}}1
        {Л}{{\selectfont\char203}}1
        {М}{{\selectfont\char204}}1
        {Н}{{\selectfont\char205}}1
        {О}{{\selectfont\char206}}1
        {П}{{\selectfont\char207}}1
        {Р}{{\selectfont\char208}}1
        {С}{{\selectfont\char209}}1
        {Т}{{\selectfont\char210}}1
        {У}{{\selectfont\char211}}1
        {Ф}{{\selectfont\char212}}1
        {Х}{{\selectfont\char213}}1
        {Ц}{{\selectfont\char214}}1
        {Ч}{{\selectfont\char215}}1
        {Ш}{{\selectfont\char216}}1
        {Щ}{{\selectfont\char217}}1
        {Ъ}{{\selectfont\char218}}1
        {Ы}{{\selectfont\char219}}1
        {Ь}{{\selectfont\char220}}1
        {Э}{{\selectfont\char221}}1
        {Ю}{{\selectfont\char222}}1
        {Я}{{\selectfont\char223}}1
        {і}{{\selectfont\char105}}1
        {ї}{{\selectfont\char168}}1
        {є}{{\selectfont\char185}}1
        {ґ}{{\selectfont\char160}}1
        {І}{{\selectfont\char73}}1
        {Ї}{{\selectfont\char136}}1
        {Є}{{\selectfont\char153}}1
        {Ґ}{{\selectfont\char128}}1 
}

\newcommand{\dollar}{\mbox{\textdollar}}

\setlength{\columnsep}{0.1in}
\setlength{\columnseprule}{1px}

\usepackage{fancyhdr}
\pagestyle{fancyplain}
\fancyhf{}
\fancyhead[R]{\fontsize{12}{12}\selectfont\bf{\thepage}}
\fancyhead[L]{\fontsize{8}{8}\selectfont\bf{Nizhny Novgorod Branch of HSE University (Andrianov, Lepeshov, Shulyatev)}}
\renewcommand{\headrulewidth}{0pt}
\renewcommand\headrule{\vspace{-0.2cm}\hspace{-0.4cm}\rule{\paperwidth}{0.4pt}\vspace{-0.8cm}}

\begin{document}

\title{\bf{Muffix Sassif -- TRD}}
\author{Andrianov, Lepeshov, Shulyatev}
\date{\selectlanguage{english}\today}
\maketitle
\begin{multicols*}{3}
\begin{center}\includegraphics[width=5cm]{/home/ilian/github/muffix-sassif-trd/notebook-generator/picture.jpg}\end{center}
\tableofcontents
\end{multicols*}
\pagebreak
\begin{multicols*}{3}
\lstloadlanguages{C++,Java}


\section{Геометрия}

\subsection{3D}
\begin{lstlisting}[language=C++]
struct Pt {
  dbl x, y, z;

  Pt() : x(0), y(0), z(0) {}
  Pt(dbl x_, dbl y_, dbl z_) : x(x_), y(y_), z(z_) {}

  Pt operator-(const Pt& o) const {
    return {x - o.x, y - o.y, z - o.z};
  }
  Pt operator+(const Pt& o) const {
    return {x + o.x, y + o.y, z + o.z};
  }
  Pt operator/(const dbl& a) const {
    return {x / a, y / a, z / a};
  }
  Pt operator*(const dbl& a) const {
    return {x * a, y * a, z * a};
  }
  Pt cross(const Pt& o) const {
    dbl nx = y * o.z - z * o.y;
    dbl ny = z * o.x - x * o.z;
    dbl nz = x * o.y - y * o.x;
    return {nx, ny, nz};
  }
  dbl dot(const Pt &o) const {
    return x * o.x + y * o.y + z * o.z;
  }
  bool operator==(const Pt& o) const {
    return abs(x - o.x) < EPS && abs(y - o.y) < EPS && abs(z - o.z) < EPS;
  }
  dbl dist() {
    return sqrtl(x * x + y * y + z * z);
  }
};

struct Plane {
  dbl a, b, c, d;

  Plane(dbl a_, dbl b_, dbl c_, dbl d_) : a(a_), b(b_), c(c_), d(d_) {
    dbl z = sqrtl(a * a + b * b + c * c);
    if (z < EPS) return;
    a /= z, b /= z, c /= z, d /= z;
  }
  dbl get_val(const Pt &p) const {
    // НЕ СТАВИТЬ МОДУЛЬ
    return a * p.x + b * p.y + c * p.z + d;
  }
  dbl dist(const Pt &p) const {
    return abs(get_val(p));
  }
  bool on_plane(const Pt &p) const {
    return dist(p) / sqrtl(a * a + b * b + c * c) < EPS;
  }
  Pt proj(const Pt &p) const {
    dbl t = get_val(p) / (a * a + b * b + c * c);
    return p - Pt(a, b, c) * t;
  }
};

bool on_line(Pt p1, Pt p2, Pt p3) {
  return (p2 - p1).cross(p3 - p1) == Pt(0, 0, 0);
}

Plane get_plane(Pt p1, Pt p2, Pt p3) {
  Pt norm = (p2 - p1).cross(p3 - p1);
  Plane pl(norm.x, norm.y, norm.z, 0);
  pl.d = -pl.get_val(p1);
  return pl;
}

pair<pair<dbl, dbl>, pair<dbl, dbl>> get_xy(dbl a, dbl b, dbl c) {
  if (abs(a) > EPS) {
    dbl y1 = 0, y2 = 10;
    return {{(-c - b * y1) / a, y1}, {(-c - b * y2) / a, y2}};
  }
  dbl x1 = 0, x2 = 10;
  return {{x1, (-c - a * x1) / b}, {x2, (-c - a * x2) / b}};
}

pair<Pt, Pt> intersect(Plane pl1, Plane pl2) {
  if (abs(pl2.a) < EPS && abs(pl2.b) < EPS && abs(pl2.c) < EPS) {
    assert(false);
  }
  if (abs(pl2.a) > EPS) {
    dbl nd = pl1.d - pl1.a * pl2.d / pl2.a;
    dbl nc = pl1.c - pl1.a * pl2.c / pl2.a;
    dbl nb = pl1.b - pl1.a * pl2.b / pl2.a;
    if (abs(nc) < EPS && abs(nb) < EPS) {
      // плоскости параллельны (могут совпадать)
      return {Pt(0, 0, 0), Pt(0, 0, 0)};
    }
    auto [yz1, yz2] = get_xy(nb, nc, nd);
    dbl x1 = (-pl2.d - pl2.c * yz1.second - pl2.b * yz1.first) / pl2.a;
    dbl x2 = (-pl2.d - pl2.c * yz2.second - pl2.b * yz2.first) / pl2.a;
    return {Pt(x1, yz1.first, yz1.second), Pt(x2, yz2.first, yz2.second)};
  }
  Plane copy_pl1(pl1.c, pl1.a, pl1.b, pl1.d);
  Plane copy_pl2(pl2.c, pl2.a, pl2.b, pl2.d);
  auto [p1, p2] = intersect(copy_pl1, copy_pl2);
  return {Pt(p1.y, p1.z, p1.x), Pt(p2.y, p2.z, p2.x)};
}

// угол между двумя векторами
dbl get_ang(Pt p1, Pt p2) {
  return acosl(p1.dot(p2) / p1.dist() / p2.dist());
}

// любой перпендикулярный вектор
Pt vector_perp(Pt v) {
  if (abs(v.x) > EPS || abs(v.y) > EPS)
    return {v.y, -v.x, 0};
  return {v.z, 0, -v.x};
}

// плоскость через точку p перпендикулярная вектору v
Plane plane_perp(Pt p, Pt v) {
  Pt v1 = vector_perp(v);
  Pt v2 = v.cross(v1);
  return get_plane(p, v1 + p, v2 + p);
}
\end{lstlisting}

\subsection{База 1 - вектор}
\begin{lstlisting}[language=C++]
char sign(dbl x) {
  return x < -EPS ? -1 : x > EPS;
}

struct vctr {
  dbl x, y;
  vctr() {}
  vctr(dbl x, dbl y) : x(x), y(y) {}

  dbl operator%(const vctr &o) const { return x * o.x + y * o.y; }
  dbl operator*(const vctr &o) const { return x * o.y - y * o.x; }
  vctr operator+(const vctr &o) const { return {x + o.x, y + o.y}; }
  vctr operator-(const vctr &o) const { return {x - o.x, y - o.y}; }
  vctr operator-() const { return {-x, -y}; }
  vctr operator*(const dbl d) const { return {x * d, y * d}; }
  vctr operator/(const dbl d) const { return {x / d, y / d}; }
  void operator+=(const vctr &o) { x += o.x, y += o.y; }
  void operator-=(const vctr &o) { x -= o.x, y -= o.y; }
  dbl dist2() const { return x * x + y * y; }
  dbl dist() const { return sqrtl(dist2()); }
  vctr norm() const { return *this / dist(); }
};

dbl angle_between(const vctr &a, const vctr &b) {
  return atan2(b * a, b % a);
}

// y > 0 ? 0 : 1
bool is2plane(const vctr &a) {
  return sign(a.y) < 0 || (sign(a.y) == 0 && sign(a.x) < 0);
}

bool cmp_angle(const vctr &a, const vctr &b) {
  bool pla = is2plane(a);
  bool plb = is2plane(b);
  if (pla != plb)
    return pla < plb;
  return sign(a * b) > 0;
}

vctr rotate_ccw(const vctr &a, dbl phi) {
  dbl cs = cos(phi);
  dbl sn = sin(phi);
  return {a.x * cs - a.y * sn, a.y * cs + a.x * sn};
}

vctr rotate_ccw_90(const vctr &a) {
  return {-a.y, a.x};
}

\end{lstlisting}

\subsection{База 2 - прямая}
\begin{lstlisting}[language=C++]
struct line {
  dbl a, b, c;

  line() {}
  line(dbl a, dbl b, dbl c) : a(a), b(b), c(c) {}
  line(const vctr A, const vctr B) {
    a = A.y - B.y;
    b = B.x - A.x;
    c = A * B;
    // left halfplane of A->B is positive
    // assert(a != 0 || b != 0);
  }
  void operator*=(dbl x) { a *= x, b *= x, c *= x; }
  void operator/=(dbl x) { a /= x, b /= x, c /= x; }
  dbl get(const vctr P) const { return a * P.x + b * P.y + c; }
  vctr anyPoint() const {
    dbl x = -a * c / (a * a + b * b);
    dbl y = -b * c / (a * a + b * b);
    return {x, y};
  }
  void normalize() {
    dbl d = sqrtl(a * a + b * b);
    a /= d, b /= d, c /= d;
  }
};

bool isparallel(line l1, line l2) {
  return sign(l2.a * l1.b - l2.b * l1.a) == 0;
}

vctr intersection(const line &l1, const line &l2) {
  dbl z = l2.a * l1.b - l2.b * l1.a;
  dbl x = (l1.c * l2.b - l2.c * l1.b) / z;
  dbl y = -(l1.c * l2.a - l2.c * l1.a) / z;
  return {x, y};
}

// Серединный перпендикуляр (не биссектриса!)
line bisection(const vctr A, const vctr B) {
  vctr M = (A + B) / 2;
  return line(M, M + rotate_ccw_90(B - A));
}

\end{lstlisting}

\subsection{База 3 - окружность}
\begin{lstlisting}[language=C++]
struct circle {
  vctr C;
  dbl r;

  circle() {}
  circle(dbl x, dbl y, dbl r) : C(x, y), r(r) {}
  circle(vctr C, dbl r) : C(C), r(r) {}
  circle(const vctr A, const vctr B) {
    C = (A + B) / 2;
    r = (A - B).dist() / 2;
  }
  circle(const vctr A, const vctr B, const vctr D) {
    line l1 = bisection(A, B);
    line l2 = bisection(B, D);
    C = intersection(l1, l2);
    r = (C - A).dist();
  }

  bool isin(const vctr P) const {
    return sign((C - P).dist2() - r * r) <= 0;
  }
};

vector<vctr> intersection_line_circ(line l, circle c) {
  l.normalize();
  dbl d = abs(l.get(c.C));
  vctr per = vctr(l.a, l.b).norm() * d;
  vctr a = c.C + per;
  if (sign(d - c.r) > 0)
    return {};
  if (sign(l.get(a)) != 0)
    a = c.C - per;
  if (sign(c.r - d) == 0)
    return {a};
  dbl k = sqrtl(c.r * c.r - d * d);
  vctr v = vctr(-l.b, l.a).norm() * k;
  return {a + v, a - v};
}

vector<vctr> intersection_circ_circ(circle A, circle B) {
  vctr a = A.C, b = B.C;
  line l(2 * (b.x - a.x),
         2 * (b.y - a.y),
         B.r * B.r - A.r * A.r
             + (a.x * a.x + a.y * a.y)
             - (b.x * b.x + b.y * b.y));
  if (sign(l.a) == 0 && sign(l.b) == 0)
    return {};
  return intersection_line_circ(l, A);
}

vector<vctr> tangent_vctr_circ(vctr v, circle c) {
  dbl d = (c.C - v).dist();
  dbl k = sqrtl(d * d - c.r * c.r);
  circle c2(v.x, v.y, k);
  return intersection_circ_circ(c, c2);
}

\end{lstlisting}

\subsection{Выпуклая оболочка}
\begin{lstlisting}[language=C++]
vctr minvctr(INF, INF);

bool cmp_convex_hull(const vctr &a, const vctr &b) {
  vctr A = a - minvctr;
  vctr B = b - minvctr;
  auto sign_prod = sign(A * B);
  if (sign_prod != 0)
    return sign_prod > 0;
  return A.dist2() < B.dist2();
}

// minvctr updates here
vector<vctr> get_convex_hull(vector<vctr> arr) {
  minvctr = {INF, INF};
  for (auto v : arr) {
    auto tmp = v - minvctr;
    if (sign(tmp.y) < 0 || (sign(tmp.y) == 0 && sign(tmp.x) < 0))
      minvctr = v;
  }
  vector<vctr> hull;
  sort(arr.begin(), arr.end(), cmp_convex_hull);
  for (vctr &el : arr) {
    while (hull.size() > 1 && sign((hull.back() - hull[hull.size() - 2]) * (el - hull.back())) <= 0)
      hull.pop_back();
    hull.push_back(el);
  }
  return hull;
}

\end{lstlisting}

\subsection{Задача 16}
\begin{lstlisting}[language=C++]
bool isInSameHalf(vctr p, vctr r1, vctr r2) {
  return sign((r2 - r1) % (p - r1)) >= 0;
}

dbl distPointPoint(vctr a, vctr b) {
  return (a - b).dist();
}

dbl distPointLine(vctr a, vctr l1, vctr l2) {
  line l(l1, l2);
  l.normalize();
  return abs(l.get(a));
}

dbl distPointRay(vctr a, vctr r1, vctr r2) {
  if (!isInSameHalf(a, r1, r2))
    return distPointPoint(a, r1);
  return distPointLine(a, r1, r2);
}

dbl distPointSeg(vctr a, vctr s1, vctr s2) {
  return max(distPointRay(a, s1, s2),
             distPointRay(a, s2, s1));
}

bool isIntersectionLineLine(line l1, line l2) {
  dbl znam = l1.b * l2.a - l1.a * l2.b;
  return sign(znam) != 0;
}

vctr intersectionLineLine(line l1, line l2) {
  dbl znam = l1.b * l2.a - l1.a * l2.b;
  dbl y = -(l1.c * l2.a - l2.c * l1.a) / znam;
  dbl x = -(l1.c * l2.b - l2.c * l1.b) / -znam;
  return vctr(x, y);
}

vctr getPointOnLine(line l) {
  if (sign(l.b) != 0)
    return vctr(0, -l.c / l.b);
  return vctr(-l.c / l.a, 0);
}

dbl distLineLine(vctr l1a, vctr l1b, vctr l2a, vctr l2b) {
  line l1(l1a, l1b);
  line l2(l2a, l2b);
  if (isIntersectionLineLine(l1, l2))
    return 0;
  vctr p = getPointOnLine(l1);
  l2.normalize();
  return abs(l2.get(p));
}

dbl distRayLine(vctr r1, vctr r2, vctr l1, vctr l2) {
  line r(r1, r2);
  line l(l1, l2);
  if (!isIntersectionLineLine(l, r))
    return distLineLine(r1, r2, l1, l2);
  vctr p = intersectionLineLine(l, r);
  if (isInSameHalf(p, r1, r2))
    return 0;
  return distPointLine(r1, l1, l2);
}

dbl distSegLine(vctr s1, vctr s2, vctr l1, vctr l2) {
  return max(distRayLine(s1, s2, l1, l2),
             distRayLine(s2, s1, l1, l2));
}

dbl distRayRay(vctr r1a, vctr r1b, vctr r2a, vctr r2b) {
  line r1(r1a, r1b);
  line r2(r2a, r2b);
  if (!isIntersectionLineLine(r1, r2)) {
    if (isInSameHalf(r1a, r2a, r2b) || isInSameHalf(r2a, r1a, r1b))
      return distLineLine(r1a, r1b, r2a, r2b);
    else
      return distPointPoint(r1a, r2a);
  }
  vctr p = intersectionLineLine(r1, r2);
  if (isInSameHalf(p, r1a, r1b) && isInSameHalf(p, r2a, r2b))
    return 0;
  return min(distPointRay(r1a, r2a, r2b),
             distPointRay(r2a, r1a, r1b));
}

dbl distSegRay(vctr s1, vctr s2, vctr r1, vctr r2) {
  return max(distRayRay(s1, s2, r1, r2),
             distRayRay(s2, s1, r1, r2));
}

dbl distSegSeg(vctr s1a, vctr s1b, vctr s2a, vctr s2b) {
  return max(distSegRay(s1a, s1b, s2a, s2b),
             distSegRay(s1a, s1b, s2b, s2a));
}

\end{lstlisting}

\subsection{Калиперы}
\begin{lstlisting}[language=C++]
// Диаметр выпуклого многоугольника
int calipers(vector<vctr> &pts) {
  int n = pts.size();
  int a = 0, b = 0;
  for (int i = 1; i < n; ++i) {
    auto &v = pts[i];
    if (tie(v.y, v.x) < tie(pts[a].y, pts[a].x))
      a = i;
    if (tie(v.y, v.x) > tie(pts[b].y, pts[b].x))
      b = i;
  }
  int aa = (a + 1) % n, bb = (b + 1) % n;
  int dist2 = 0;
  for (int i = 0; i < n; ++i) {
    while (sign((pts[aa] - pts[a]) * (pts[bb] - pts[b])) > 0)
      b = bb, bb = (b + 1) % n;
    dist2 = max(dist2, (pts[a] - pts[b]).dist2());
    a = aa, aa = (a + 1) % n;
  }
  return dist2;
}
\end{lstlisting}

\subsection{Касательные из точки}
\begin{lstlisting}[language=C++]
pair<int, int> tangents_from_point(vector<vctr> &p, vctr &a) {
  int n = p.size();
  int logn = 31 - __builtin_clz(n);
  auto findWithSign = [&](int sgn) {
    int i = 0;
    for (int k = logn; k >= 0; --k) {
      int i1 = (i - (1 << k) + n) % n;
      int i2 = (i + (1 << k)) % n;
      if (sign((p[i1] - a) * (p[i] - a)) == sgn)
        i = i1;
      if (sign((p[i2] - a) * (p[i] - a)) == sgn)
        i = i2;
    }
    return i;
  };
  return {findWithSign(1), findWithSign(-1)};
}

\end{lstlisting}

\subsection{Касательные параллельные прямой}
\begin{lstlisting}[language=C++]
// find point with max (sgn=1) or min (sgn=-1) signed distance to line
int tangent_parallel_line(const vector<vctr> &p, line l, int sgn) {
  l *= sgn;
  int n = p.size();
  int i = 0;
  int logn = 31 - __builtin_clz(n);
  for (int k = logn; k >= 0; --k) {
    int i1 = (i - (1 << k) + n) % n;
    int i2 = (i + (1 << k)) % n;
    if (l.get(p[i1]) > l.get(p[i]))
      i = i1;
    if (l.get(p[i2]) > l.get(p[i]))
      i = i2;
  }
  return i;
}

\end{lstlisting}

\subsection{Лежит ли точка в многоугольнике}
\begin{lstlisting}[language=C++]
// Выпуклый многоугольник, P[0] = minvctr
bool is_point_in_poly(vctr A, vector<vctr> &P) {
  auto tmp = A - P[0];
  if (sign(tmp.y) < 0 || (sign(tmp.y) == 0 && sign(tmp.x) < 0))
    return false;
  if (sign(tmp.y) == 0 && sign(tmp.x) == 0)
    return true;
  int ind = lower_bound(P.begin(), P.end(), A, cmp_convex_hull) - P.begin();
  assert(ind != 0);
  if (ind == P.size())
    return false;
  vctr B = A - P[ind - 1];
  vctr C = P[ind] - P[ind - 1];
  return sign(C * B) >= 0;
}

bool is_point_in_poly_strict(vctr A, vector<vctr> &P) {
  if (sign(A.y - P[0].y) <= 0 || sign((A - P[0]) * (P.back() - P[0])) <= 0)
      return false;
  int ind = lower_bound(P.begin(), P.end(), A, cmp_convex_hull) - P.begin();
  assert(ind != 0 && ind != P.size());
  vctr B = A - P[ind - 1];
  vctr C = P[ind] - P[ind - 1];
  return sign(C * B) > 0;
}

\end{lstlisting}

\subsection{Минимальная покрывающая окружность}
\begin{lstlisting}[language=C++]
mt19937 rnd(179);

circle MinDisk2(vector<vctr> &p, vctr A, vctr B, int sz) {
  circle w(A, B);
  for (int i = 0; i < sz; ++i) {
    if (w.isin(p[i]))
      continue;
    w = circle(A, B, p[i]);
  }
  return w;
}

circle MinDisk1(vector<vctr> &p, vctr A, int sz) {
  shuffle(p.begin(), p.begin() + sz, rnd);
  circle w(A, p[0]);
  for (int i = 1; i < sz; ++i) {
    if (w.isin(p[i]))
      continue;
    w = MinDisk2(p, A, p[i], i);
  }
  return w;
}

circle MinDisk(vector<vctr> &p) {
  int sz = p.size();
  if (sz == 1)
    return circle(p[0], 0);
  shuffle(p.begin(), p.end(), rnd);
  circle w(p[0], p[1]);
  for (int i = 2; i < sz; ++i) {
    if (w.isin(p[i]))
      continue;
    w = MinDisk1(p, p[i], i);
  }
  return w;
}

\end{lstlisting}

\subsection{Пересечение полуплоскостей}
\begin{lstlisting}[language=C++]
// half plane: ax+by+c > 0
// bounding box MUST have
vector<int> intersection_half_planes_inds(const vector<line> &ls) {
  int n = (int)ls.size();
  vector<int> lsi(n);
  iota(lsi.begin(), lsi.end(), 0);
  sort(lsi.begin(), lsi.end(), [&](int i, int j) {
    vctr aa(ls[i].a, ls[i].b);
    vctr bb(ls[j].a, ls[j].b);
    bool pla = is2plane(aa);
    bool plb = is2plane(bb);
    if (pla != plb)
      return pla < plb;
    return aa * bb > 0;
  });

  vector<line> st;
  vector<int> inds;
  for (int ii = 0; ii < 2 * n; ++ii) {
    int i = lsi[ii % n];
    if (st.empty()) {
      st.push_back(ls[i]);
      inds.push_back(i);
      continue;
    }
    vctr p = intersection(ls[i], st.back());
    bool pp = isparallel(ls[i], st.back());
    bool bad = false;
    while (st.size() >= 2) {
      if (!pp && sign(st[st.size() - 2].get(p)) >= 0)
        break;
      else if (pp && sign(st.back().get(ls[i].anyPoint())) <= 0) {
        bad = true;
        break;
      }
      st.pop_back();
      inds.pop_back();
      p = intersection(ls[i], st.back());
      pp = isparallel(ls[i], st.back());
    }
    if (!bad) {
      st.push_back(ls[i]);
      inds.push_back(i);
    }
  }
  vector<int> cnt(n, 0);
  for (int i : inds)
    cnt[i]++;
  vector<int> good;
  for (int i : inds) {
    if (cnt[i]-- == 2)
      good.push_back(i);
  }
  return good;
}

vector<vctr> intersection_half_planes(vector<line> &ls) {
  vector<int> inter = intersection_half_planes_inds(ls);
  int n = inter.size();
  vector<vctr> pts;
  for (int i = 0; i < n; ++i) {
    int j = (i + 1) % n;
    vctr P = intersection(ls[inter[i]], ls[inter[j]]);
    if (pts.empty() || sign(pts.back().x - P.x) != 0
         || sign(pts.back().y - P.y) != 0)
      pts.push_back(P);
  }
  // pts против часовой стрелки, но pts[0] != minvctr
  return pts;
}

\end{lstlisting}

\subsection{Проверка на пересечение отрезков}
\begin{lstlisting}[language=C++]
bool is_intersection_seg(vctr A, vctr B, vctr C, vctr D) {
  for (int i = 0; i < 2; ++i) {
    auto l1 = A.x, r1 = B.x, l2 = C.x, r2 = D.x;
    if (l1 > r1) swap(l1, r1);
    if (l2 > r2) swap(l2, r2);
    if (max(l1, l2) > min(r1, r2))
      return false;
    swap(A.x, A.y);
    swap(B.x, B.y);
    swap(C.x, C.y);
    swap(D.x, D.y);
  }
  for (int _ = 0; _ < 2; ++_) {
    auto v1 = (B - A) * (C - A);
    auto v2 = (B - A) * (D - A);
    if (sign(v1) * sign(v2) == 1)
      return false;
    swap(A, C);
    swap(B, D);
  }
  return true;
}

\end{lstlisting}

\subsection{Сумма Минковского}
\begin{lstlisting}[language=C++]
// Список вершин -> список рёбер
vector<vctr> poly_to_edges(const vector<vctr> &A) {
  vector<vctr> edg(A.size());
  for (int i = 0; i < A.size(); ++i)
    edg[i] = A[(i + 1) % A.size()] - A[i];
  return edg;
}

// A и B начинаются с минимальных вершин
vector<vctr> minkowski_sum(const vector<vctr> &A, const vector<vctr> &B) {
  auto edgA = poly_to_edges(A);
  auto edgB = poly_to_edges(B);
  vector<vctr> edgC(A.size() + B.size());
  merge(edgA.begin(), edgA.end(), edgB.begin(), edgB.end(), edgC.begin(), cmp_angle);
  // cmp_angle из шаблона вектора
  vector<vctr> C(edgC.size());
  C[0] = A[0] + B[0];
  for (int i = 0; i + 1 < C.size(); ++i)
    C[i + 1] = C[i] + edgC[i];
  return C;
}

\end{lstlisting}

\subsection{Формула Эйлера}
\noindent\rule{\linewidth}{0.15mm}
\begin{itemize}
    \item $V$ -- число вершин выпуклого многогранника (планарного графа)
    \item $E$ -- число рёбер
    \item $F$ -- число граней (если планарный граф, то включая внешнюю)
\end{itemize}
Тогда $V-E+F=2$

\noindent\rule{\linewidth}{0.15mm}

\section{Графы}

\subsection{2-SAT}
\begin{lstlisting}[language=C++]
struct TwoSat {
  int n;
  vector<vector<int>> g, rg;
  vector<int> comp, topsort;
  vector<char> used;

  TwoSat(int n_) : n(n_) {
    g.resize(2 * n);
    rg.resize(2 * n);
    comp.assign(2 * n, -1);
    topsort.reserve(2 * n);
    used.assign(2 * n, 0);
  }

  int neg(int v) {
    return 2 * n - 1 - v;
  }

  void add(int v, int u) {
    g[v].pb(u);
    rg[u].pb(v);
  }

  void add_OR(int v, int u) { // v | u
    add(neg(v), u), add(neg(u), v);
  }

  void add_IMPL(int v, int u) { // v -> u
    add_OR(neg(v), u);
  }

  void dfs1(int v) {
    used[v] = 1;
    for (auto u: g[v]) {
      if (!used[u])
        dfs1(u);
    }
    topsort.push_back(v);
  }

  void dfs2(int v, int col) {
    comp[v] = col;
    for (auto u: rg[v]) {
      if (comp[u] == -1)
        dfs2(u, col);
    }
  }

  void SCC() {
    for (int v = 0; v < 2 * n; ++v) {
      if (!used[v])
        dfs1(v);
    }
    reverse(all(topsort));
    int cc = 0;
    for (int i = 0; i < 2 * n; ++i) {
      if (comp[topsort[i]] == -1)
        dfs2(topsort[i], cc++);
    }
  }

  vector<int> solution() {
    SCC();
    vector<int> ans(n);
    for (int v = 0; v < n; ++v) {
      if (comp[v] == comp[neg(v)]) {
        return {-1}; // no solution
      }
      ans[v] = comp[v] > comp[neg(v)];
    }
    return ans;
  }

};
\end{lstlisting}

\subsection{L-R Flow}
\begin{lstlisting}[language=C++]
struct LRFlow {
  Dinic dinic;
  int S, T; // исток и сток
  int Sx, Tx; // вспомогательные вершины, любые неиспользуемые индексы

  LRFlow(int S, int T, int Sx, int Tx) : S(S), T(T), Sx(Sx), Tx(Tx) {}

  void addedge(int v, int u, int mincap, int maxcap) {
    // все рёбра ориентированные
    dinic.addedge(v, u, maxcap - mincap);
    dinic.addedge(Sx, u, mincap);
    dinic.addedge(v, Tx, mincap);
  }

  bool inner_check() {
    for (auto edge: dinic.graph[Sx]) {
      if (edge.f != edge.c) {
        return false;
      }
    }
    for (auto edge: dinic.graph[Tx]) {
      auto iedge = dinic.graph[edge.u][edge.r];
      if (iedge.f != iedge.c) {
        return false;
      }
    }
    return true;
  }

  bool only_existence() {
    dinic.addedge(T, S, INF);
    dinic.run(Sx, Tx);
    return inner_check();
  }

  bool with_flows() {
    dinic.run(Sx, Tx);
    dinic.run(Sx, T);
    dinic.run(S, Tx);
    dinic.run(S, T);
    // real (v, u, mincap, maxcap) flow = flow on (v, u, maxcap - mincap) edge + mincap
    return inner_check();
  }
};
\end{lstlisting}

\subsection{WeightedMatching}
\begin{lstlisting}[language=C++]
// НЕ ЗАБЫТЬ вызвать init(n)
// вершины нумераются от 1 до n
namespace weighted_matching{
  const int INF = (int)1e9 + 7;
  const int MAXN = 1050; //double of possible N
  struct E{
    int x, y, w;
  };
  int n, m;
  E G[MAXN][MAXN];
  int lab[MAXN], match[MAXN], slack[MAXN], st[MAXN], pa[MAXN];
  int flo_from[MAXN][MAXN], S[MAXN], vis[MAXN];
  vector<int> flo[MAXN];
  queue<int> Q;
  void init(int _n) {
    n = _n;
    for(int x = 1; x <= n; ++x)
      for(int y = 1; y <= n; ++y)
        G[x][y] = E{x, y, 0};
  }
  void add_edge(int x, int y, int w) {
    G[x][y].w = G[y][x].w = w;
  }
  int e_delta(E e) {
    return lab[e.x] + lab[e.y] - G[e.x][e.y].w * 2;
  }
  void update_slack(int u, int x) {
    if(!slack[x] || e_delta(G[u][x]) < e_delta(G[slack[x]][x]))
      slack[x] = u;
  }
  void set_slack(int x) {
    slack[x] = 0;
    for(int u = 1; u <= n; ++u)
      if(G[u][x].w > 0 && st[u] != x && S[st[u]] == 0)
        update_slack(u, x);
  }
  void q_push(int x) {
    if(x <= n) Q.push(x);
    else for(int i = 0; i < (int)flo[x].size(); ++i)
        q_push(flo[x][i]);
  }
  void set_st(int x, int b) {
    st[x] = b;
    if(x > n) for(int i = 0; i < (int)flo[x].size(); ++i)
        set_st(flo[x][i], b);
  }
  int get_pr(int b, int xr) {
    int pr = find(flo[b].begin(), flo[b].end(), xr) - flo[b].begin();
    if(pr & 1) {
      reverse(flo[b].begin() + 1, flo[b].end());
      return (int)flo[b].size() - pr;
    }
    else return pr;
  }
  void set_match(int x, int y) {
    match[x] = G[x][y].y;
    if(x <= n) return;
    E e = G[x][y];
    int xr = flo_from[x][e.x], pr = get_pr(x, xr);
    for(int i = 0; i < pr; ++i) set_match(flo[x][i], flo[x][i^1]);
    set_match(xr, y);
    rotate(flo[x].begin(), flo[x].begin() + pr, flo[x].end());
  }
  void augment(int x, int y) {
    while(1) {
      int ny = st[match[x]];
      set_match(x, y);
      if(!ny) return;
      set_match(ny, st[pa[ny]]);
      x = st[pa[ny]], y = ny;
    }
  }
  int get_lca(int x, int y) {
    static int t = 0;
    for(++t; x || y; swap(x, y)) {
      if(x == 0) continue;
      if(vis[x] == t) return x;
      vis[x] = t;
      x = st[match[x]];
      if(x) x = st[pa[x]];
    }
    return 0;
  }
  void add_blossom(int x, int l, int y) {
    int b = n + 1;
    while(b <= m && st[b]) ++b;
    if(b > m) ++m;
    lab[b] = 0, S[b] = 0;
    match[b] = match[l];
    flo[b].clear();
    flo[b].push_back(l);
    for(int u = x, v; u != l; u = st[pa[v]])
      flo[b].push_back(u), flo[b].push_back(v = st[match[u]]), q_push(v);
    reverse(flo[b].begin() + 1, flo[b].end());
    for(int u = y, v; u != l; u = st[pa[v]])
      flo[b].push_back(u), flo[b].push_back(v = st[match[u]]), q_push(v);
    set_st(b, b);
    for(int i = 1; i <= m; ++i) G[b][i].w = G[i][b].w = 0;
    for(int i = 1; i <= n; ++i) flo_from[b][i] = 0;
    for(int i = 0; i < (int)flo[b].size(); ++i) {
      int us = flo[b][i];
      for(int u = 1; u <= m; ++u)
        if(G[b][u].w == 0 || e_delta(G[us][u]) < e_delta(G[b][u]))
          G[b][u] = G[us][u], G[u][b] = G[u][us];
      for(int u = 1; u <= n; ++u)
        if(flo_from[us][u])
          flo_from[b][u] = us;
    }
    set_slack(b);
  }
  void expand_blossom(int b) {
    for(int i = 0; i < (int)flo[b].size(); ++i)
      set_st(flo[b][i], flo[b][i]);
    int xr = flo_from[b][G[b][pa[b]].x], pr = get_pr(b, xr);
    for(int i = 0; i < pr; i += 2) {
      int xs = flo[b][i], xns = flo[b][i + 1];
      pa[xs] = G[xns][xs].x;
      S[xs] = 1, S[xns] = 0;
      slack[xs] = 0, set_slack(xns);
      q_push(xns);
    }
    S[xr] = 1, pa[xr] = pa[b];
    for(int i = pr + 1; i < (int)flo[b].size(); ++i) {
      int xs = flo[b][i];
      S[xs] = -1, set_slack(xs);
    }
    st[b] = 0;
  }
  bool on_found_edge(E e) {
    int x = st[e.x], y = st[e.y];
    if(S[y] == -1) {
      pa[y] = e.x, S[y] = 1;
      int ny = st[match[y]];
      slack[y] = slack[ny] = 0;
      S[ny] = 0, q_push(ny);
    }
    else if(S[y] == 0) {
      int l = get_lca(x, y);
      if(!l) return augment(x, y), augment(y, x), true;
      else add_blossom(x, l, y);
    }
    return false;
  }
  bool matching() {
    fill(S + 1, S + m + 1, -1);
    fill(slack + 1, slack + m + 1, 0);
    Q = queue<int>();
    for(int x = 1; x <= m; ++x)
      if(st[x] == x && !match[x]) pa[x] = 0, S[x] = 0, q_push(x);
    if(Q.empty()) return false;
    while(1) {
      while(Q.size()) {
        int x = Q.front(); Q.pop();
        if(S[st[x]] == 1) continue;
        for(int y = 1; y <= n; ++y) {
          if(G[x][y].w > 0 && st[x] != st[y]) {
            if(e_delta(G[x][y]) == 0) {
              if(on_found_edge(G[x][y])) return true;
            }
            else update_slack(x, st[y]);
          }
        }
      }
      int d = INF;
      for(int b = n + 1; b <= m; ++b)
        if(st[b] == b && S[b] == 1) d = min(d, lab[b] / 2);
      for(int x = 1; x <= m; ++x)
        if(st[x] == x && slack[x]) {
          if(S[x] == -1) d = min(d, e_delta(G[slack[x]][x]));
          else if(S[x] == 0) d = min(d, e_delta(G[slack[x]][x]) / 2);
        }
      for(int x = 1; x <= n; ++x) {
        if(S[st[x]] == 0) {
          if(lab[x] <= d) return 0;
          lab[x] -= d;
        }
        else if(S[st[x]] == 1) lab[x] += d;
      }
      for(int b = n + 1; b <= m; ++b)
        if(st[b] == b) {
          if(S[st[b]] == 0) lab[b] += d * 2;
          else if(S[st[b]] == 1) lab[b] -= d * 2;
        }
      Q = queue<int>();
      for(int x = 1; x <= m; ++x)
        if(st[x] == x && slack[x] && st[slack[x]] != x && e_delta(G[slack[x]][x]) == 0)
          if(on_found_edge(G[slack[x]][x])) return true;
      for(int b = n + 1; b <= m; ++b)
        if(st[b] == b && S[b] == 1 && lab[b] == 0)
          expand_blossom(b);
    }
    return false;
  }
  pair<int, int> solve(vector<pair<int, int>> &ans) {
    fill(match + 1, match + n + 1, 0);
    m = n;
    int cnt = 0; int sum = 0;
    for(int u = 0; u <= n; ++u) st[u] = u, flo[u].clear();
    int mx = 0;
    for(int x = 1; x <= n; ++x)
      for(int y = 1; y <= n; ++y){
        flo_from[x][y] = (x == y ? x : 0);
        mx = max(mx, G[x][y].w);
      }
    for(int x = 1; x <= n; ++x) lab[x] = mx;
    while(matching()) ++cnt;
    for(int x = 1; x <= n; ++x)
      if(match[x] && match[x] < x) {
        sum += G[x][match[x]].w;
        ans.push_back({x, G[x][match[x]].y});
      }
    return {sum, cnt};
  }
}
\end{lstlisting}

\subsection{Венгерский алгоритм}
\begin{lstlisting}[language=C++]
pair<int, vector<int>> venger(vector<vector<int>> a) {
// ищет минимальное по стоимости
// работает только при n <= m
// a - массив весов $(n+1) \times (m+1)$
// a[0][..] = a[..][0] = 0
// возвращает ans[i] = j если взяли ребро a[i][j]
  int n = (int) a.size() - 1;
  int m = (int) a[0].size() - 1;
  vector<int> u(n + 1), v(m + 1), p(m + 1), way(m + 1);
  for (int i = 1; i <= n; ++i) {
    p[0] = i;
    int j0 = 0;
    vector<int> minv(m + 1, INF);
    vector<char> used(m + 1, false);
    do {
      used[j0] = true;
      int i0 = p[j0], delta = INF, j1;
      for (int j = 1; j <= m; ++j)
        if (!used[j]) {
          int cur = a[i0][j] - u[i0] - v[j];
          if (cur < minv[j])
            minv[j] = cur, way[j] = j0;
          if (minv[j] < delta)
            delta = minv[j], j1 = j;
        }
      for (int j = 0; j <= m; ++j)
        if (used[j])
          u[p[j]] += delta, v[j] -= delta;
        else
          minv[j] -= delta;
      j0 = j1;
    } while (p[j0] != 0);
    do {
      int j1 = way[j0];
      p[j0] = p[j1];
      j0 = j1;
    } while (j0);
  }
  int cost = -v[0];
  vector<int> ans(n + 1);
  for (int j = 1; j <= m; ++j)
    ans[p[j]] = j;
  return {cost, ans};
}
\end{lstlisting}

\subsection{Вершинная двусвязность}
\begin{lstlisting}[language=C++]
vector<pair<int, int>> graph[MAX_V];
bitset<MAX_V> vis;
int st[MAX_E], col[MAX_E], tin[MAX_V], up[MAX_V];
int sti = 0, cc = 0, tt = 0;

void dfs(int v, int pei) {
  vis[v] = true;
  int upv = tin[v] = tt++;
  for (auto [u, ei] : graph[v]) {
    if (ei == pei) continue;
    if (!vis[u]) {
      int pt = sti;
      st[sti++] = ei;
      dfs(u, ei);
      upv = min(upv, up[u]);
      if (up[u] >= tin[v]) {
        while (sti > pt)
          col[st[--sti]] = cc;
        cc++;
      }
    } else if (tin[u] <= tin[v]) {
      st[sti++] = ei;
      upv = min(upv, tin[u]);
    }
  }
  up[v] = upv;
}

// graph[v].emplace_back(u, i);
// graph[u].emplace_back(v, i);
fill(col, col + m, -1);
for (int v = 0; v < n; ++v) {
  if (!vis[v])
    dfs(v, -1);
}
// col[i] - компонента i-го ребра
// cc - итоговое кол-во компонент
\end{lstlisting}

\subsection{Диниц}
\begin{lstlisting}[language=C++]
const int LOG = 29; // масштабирование, =0 если не нужно
struct Edge { int u, f, c, r; };

struct Dinic {
  vector<Edge> graph[MAXN];
  bitset<MAXN> vis;
  int inds[MAXN], dist[MAXN], Q[MAXN];
  int ql, qr, S, T, BIT;
  Dinic() {}

  bool bfs() {
    vis.reset();
    ql = 0, qr = 0;
    dist[S] = 0, vis[S] = true;
    Q[qr++] = S;
    while (ql < qr) {
      int v = Q[ql++];
      for (auto &e : graph[v]) {
        int u = e.u;
        if (vis[u] || e.c - e.f < BIT)
          continue;
        vis[u] = true;
        dist[u] = dist[v] + 1;
        Q[qr++] = u;
        if (u == T) return true;
      }
    }
    return false;
  }

  int dfs(int v, int maxF) {
    if (v == T) return maxF;
    int ans = 0;
    for (int &i = inds[v]; i < graph[v].size(); ++i) {
      auto &e = graph[v][i];
      auto cc = min(maxF - ans, e.c - e.f);
      if (dist[e.u] <= dist[v] || !vis[e.u] || inds[e.u] == graph[e.u].size() || cc < BIT)
        continue;
      auto f = dfs(e.u, cc);
      if (f != 0) {
        e.f += f, ans += f;
        graph[e.u][e.r].f -= f;
      }
      // иногда быстрее один иф, иногда другой
      if (maxF - ans < 1) break;
      // if (maxF - ans < BIT) break;
    }
    return ans;
  }

  void run(int s, int t) {
    S = s, T = t;
    assert(S != T);
    for (BIT = (1ll << LOG); BIT > 0; BIT >>= 1) {
      while (bfs()) {
        memset(inds, 0, sizeof inds);
        for (auto &e : graph[S]) {
          if (inds[e.u] == graph[e.u].size() || e.c - e.f < BIT)
            continue;
          int f = dfs(e.u, e.c - e.f);
          e.f += f, graph[e.u][e.r].f -= f;
        }
      }
    }
  }

  void addedge(int v, int u, int c) {
    graph[v].push_back({u, 0, c, (int)graph[u].size()});
    // если ориентированно, то обратная capacity = 0
    graph[u].push_back({v, 0, c, (int)graph[v].size() - 1});
  }
};

void use_example() {
  Dinic dinic;
  for (int i = 0; i < m; ++i) {
    int v, u, c;
    cin >> v >> u >> c;
    v--, u--;
    dinic.addedge(v, u, c);
  }
  dinic.run(s, t);

  long long maxflow = 0;
  for (auto &e : dinic.graph[s])
    maxflow += e.f;

  vector<int> cut;
  for (int i = 0; i < m; i++) {
    auto &e = edges[i];
    if (dinic.vis[e.v] != dinic.vis[e.u])
      cut.push_back(i);
  }
}
\end{lstlisting}

\subsection{КСС}
\begin{lstlisting}[language=C++]
void dfs1(int v, vector<char> &used, vector<int> &topsort) {
  used[v] = 1;
  for (auto u : g[v]) {
    if (!used[u])
      dfs1(u, used, topsort);
  }
  topsort.push_back(v);
}

void dfs2(int v, int col, vector<int> &comp) {
  comp[v] = col;
  for (auto u : rg[v]) {
    if (comp[u] == -1)
      dfs2(u, col, comp);
  }
}

signed main() {
  vector<int> topsort;
  topsort.reserve(n);
  vector<char> used(n, 0);
  for (int v = 0; v < n; ++v) {
    if (!used[v])
      dfs1(v, used, topsort);
  }
  reverse(all(topsort));
  int cc = 0;
  vector<int> comp(n, -1);
  for (int i = 0; i < n; ++i) {
    if (comp[topsort[i]] == -1)
      dfs2(topsort[i], cc++, comp);
  }
}
\end{lstlisting}

\subsection{Минкост (Джонсон)}
\begin{lstlisting}[language=C++]

using cost_t = ll;
using flow_t = int;

const int MAXN = 10000;
const int MAXM = 25000 * 2;
const cost_t INFw = 1e12;
const flow_t INFf = 10;

struct Edge {
  int v, u;
  flow_t f, c;
  cost_t w;
};

Edge edg[MAXM];
int esz = 0;
vector<int> graph[MAXN];
ll dist[MAXN];
ll pot[MAXN];
int S, T;
int NUMV;
int pre[MAXN];
bitset<MAXN> inQ;

flow_t get_flow() {
  int v = T;
  if (pre[v] == -1)
    return 0;
  flow_t f = INFf;
  do {
    int ei = pre[v];
    Edge &e = edg[ei];
    f = min(f, e.c - e.f);
    if (f == 0)
      return 0;
    v = e.v;
  } while (v != S);
  v = T;
  do {
    int ei = pre[v];
    edg[ei].f += f;
    edg[ei ^ 1].f -= f;
    v = edg[ei].v;
  } while (v != S);
  return f;
}

void spfa() {
  fill(dist, dist + NUMV, INFw);
  dist[S] = 0;
  deque<int> Q = {S};
  inQ[S] = true;
  while (!Q.empty()) {
    int v = Q.front();
    Q.pop_front();
    inQ[v] = false;
    cost_t d = dist[v];
    for (int ei : graph[v]) {
      Edge &e = edg[ei];
      if (e.f == e.c)
        continue;
      cost_t w = e.w + pot[v] - pot[e.u];
      if (dist[e.u] <= d + w)
        continue;
      pre[e.u] = ei;
      dist[e.u] = d + w;
      if (!inQ[e.u]) {
        inQ[e.u] = true;
        Q.push_back(e.u);
      }
    }
  }
  for (int i = 0; i < NUMV; ++i)
    pot[i] += dist[i];
}

cost_t mincost() {
  spfa(); // pot[i] = 0 // or ford_bellman
  flow_t f = 0;
  while (true) {
    flow_t ff = get_flow();
    if (ff == 0)
      break;
    f += ff;
    spfa(); // or dijkstra
  }
  cost_t res = 0;
  for (int i = 0; i < esz; ++i)
    res += edg[i].f * edg[i].w;
  res /= 2;
  return res;
}

void add_edge(int v, int u, int c, int w) {
  edg[esz] = {v, u, 0, c, w};
  edg[esz + 1] = {u, v, 0, 0, -w};
  graph[v].push_back(esz);
  graph[u].push_back(esz + 1);
  esz += 2;
}

signed main() {
  ios_base::sync_with_stdio(false);
  cin.tie(nullptr);
  int n, m;
  cin >> n >> m;
  S = 0;
  T = n - 1;
  NUMV = n;
  for (int i = 0; i < m; ++i) {
    int v, u, c, w;
    cin >> v >> u >> c >> w;
    v--, u--;
    add_edge(v, u, c, w);
  }
  cost_t ans = mincost();
  cout << ans;
}


\end{lstlisting}

\subsection{Мосты}
\begin{lstlisting}[language=C++]
// graph[v][i] = {u, edge_i}
void dfs(int v, int pi = -1) {
  vis[v] = 1;
  up[v] = tin[v] = timer++;
  for (auto [u, ei] : g[v]) {
    if (!vis[u]) {
      dfs(u, ei);
      up[v] = min(up[v], up[u]);
    } else if (ei != pi)
      up[v] = min(up[v], tin[u]);
    if (up[u] > tin[v]) {
      bridges.emplace_back(v, u);
      is_bridge[ei] = 1;
    }
  }
}
\end{lstlisting}

\subsection{Паросочетания}
\begin{lstlisting}[language=C++]
int dfs(int v, int c) {
  if (used[v] == c) return 0;
  used[v] = c;
  for (auto u : g[v]) {
    if (res[u] == -1) {
      res[u] = v;
      return 1;
    }
  }
  for (auto u : g[v]) {
    if (dfs(res[u], c)) {
      res[u] = v;
      return 1;
    }
  }
  return 0;
}

signed main() {
  // n - в левой доле, m - в правой
  fill(res, res + m, -1);
  for (int i = 0; i < n; ++i) {
    ans += dfs(i, i + 1);
  }
}
\end{lstlisting}

\subsection{Точки сочленения}
\begin{lstlisting}[language=C++]
void dfs(int v, int par) {
  vis[v] = 1;
  up[v] = tin[v] = timer++;
  int child = 0;
  for (auto u : g[v]) {
    if (!vis[u]) {
      dfs(u, v);
      up[v] = min(up[v], up[u]);
      if (up[u] >= tin[v] && par != -1) {
        points.insert(v);
      }
      child++;
    } else if (u != par) {
      up[v] = min(up[v], tin[u]);
    }
  }
  if (par == -1 && child >= 2) {
    points.insert(v);
  }
}

\end{lstlisting}

\subsection{Эдмондс-Карп}
\begin{lstlisting}[language=C++]
struct edge {
  int v, f, c, ind;
};

vector<edge> g[MAXN];

bool bfs(int start, int final, int W) {
  vector<int> d(MAXN, INF);
  vector<pair<int, int>> pred(MAXN);
  d[start] = 0;
  deque<int> q = {start};
  while (!q.empty()) {
    int v = q.front();
    q.pop_front();
    for (int i = 0; i < (int) g[v].size(); i++) {
      auto e = g[v][i];
      if (e.f + W <= e.c && d[e.v] > d[v] + 1) {
        d[e.v] = d[v] + 1;
        pred[e.v] = {v, i};
        q.push_back(e.v);
      }
    }
  }
  if (d[final] == INF) {
    return false;
  }
  int v = final;
  int x = INF;
  while (v != start) {
    int ind = pred[v].second;
    v = pred[v].first;
    x = min(x, g[v][ind].c - g[v][ind].f);
  }
  v = final;
  while (v != start) {
    int ind = pred[v].second;
    v = pred[v].first;
    g[v][ind].f += x;
    g[g[v][ind].v][g[v][ind].ind].f -= x;
  }
  return true;
}

signed main() {
  int n, m;
  for (int i = 0; i < m; i++) {
    int u, v, c;
    cin >> u >> v >> c;
    g[u].push_back({v, 0, c, (int) g[v].size()});
    g[v].push_back({u, 0, 0, (int) g[u].size() - 1});
  }
  int start = 1, final = n;
  int W = (1 << 30);
  do {
    while (bfs(start, final, W));
    W /= 2;
  } while (W >= 1);
  int res = 0;
  for (auto e : g[start]) {
    res += e.f;
  }
}
\end{lstlisting}

\subsection{Эйлеров цикл}
\begin{lstlisting}[language=C++]
// Эйлеров путь/цикл в компоненте связности s. Возвращает индексы рёбер. Если пути/цикла нет, то алгос найдёт фигню.
// Если неориентированный граф, то edges[ei] и edges[ei ^ 1] - обратные друг к другу рёбра.
// edges[graph[v][i]] = {v, u}
vector<int> eulerpath1(int s, vector<vector<int>> &graph, vector<pair<int, int>> &edges, vector<char> &used, vector<int> &start) {
  vector<pair<int, int>> st = {{-1, s}};
  vector<int> res;
  while (!st.empty()) {
    auto [ei, v] = st.back();
    while (start[v] < graph[v].size() && used[graph[v][start[v]]])
      start[v]++;
    if (start[v] == graph[v].size()) {
      if (ei != -1) res.push_back(ei);
      st.pop_back();
    } else {
      int ej = graph[v][start[v]++];
      used[ej] = true;
      used[ej ^ 1] = true; // Удалить если ориент. граф
      st.emplace_back(ej, edges[ej].second);
    }
  }
  reverse(all(res));
  return res;
}

vector<char> used(edges.size(), false);
vector<int> start(graph.size(), 0);
for (int v = 0; v < graph.size(); ++v) {
  // Если ориентированный граф, второе условие заменить на cnt_in[v] >= cnt_out[v]
  if (start[v]==graph[v].size() || graph[v].size()%2==0)
    continue;
  auto path = eulerpath1(v, graph, edges, used, start);
}
for (int v = 0; v < graph.size(); ++v) {
  if (start[v] == graph[v].size())
    continue;
  auto cycle = eulerpath1(v, graph, edges, used, start);
}

\end{lstlisting}

\section{ДП}

\subsection{CHT}
\begin{lstlisting}[language=C++]
struct line {
    int k, b;
    int eval(int x) {
      return k * x + b;
    }
};
struct part {
    line a;
    double x;
};
double intersection(line a, line b) {
  return (a.b - b.b) / (double) (b.k - a.k);
}
struct ConvexHull {
    // for min: k decreasing (non-increasing)
    // for max: k increasing (non-decreasing)
    vector<part> st;

    void add(line a) {
      if (!st.empty() && st.back().a.k == a.k) {
        if (st.back().a.b < a.b) st.pop_back(); // for max
//        if (st.back().a.b > a.b) st.pop_back(); // for min
        else return;
      }
      while (st.size() > 1 &&
             intersection(st[st.size() - 2].a, a) <= st[st.size() - 2].x)
        st.pop_back();
      if (!st.empty()) st.back().x = intersection(st.back().a, a);
      st.push_back({a, INFINITY}); // C++ define
    }

    int get_val(int x) {
      if (st.empty()) {
        return -INF; // min possible value, for max
//        return INF; // max possible value, for min
      }
      int l = -1, r = (int) st.size() - 1;
      while (r - l > 1) {
        int m = (l + r) / 2;
        if (st[m].x < x) l = m;
        else r = m;
      }
      return st[r].a.eval(x);
    }
};
\end{lstlisting}

\subsection{Li Chao}
\begin{lstlisting}[language=C++]
// MAXIMUM
struct Line {
  int k, b;

  int f(int x) {
    return k * x + b;
  }
};

struct ST {
  vector<Line> st;

  ST(int n) {
    Line ln = {0LL, -INF};
    st.resize(4 * n, ln);
  }

  void upd(int i, int l, int r, Line ln) {
    int child = 1;
    Line ln1 = ln;
    int m = (l + r) / 2;
    if (ln.f(m) > st[i].f(m)) {
      if (ln.k < st[i].k) {
        child = 2;
      }
      ln1 = st[i];
      st[i] = ln;
    } else {
      if (st[i].k < ln.k) {
        child = 2;
      }
    }
    if (l + 1 < r) {
      if (child == 1) {
        upd(i * 2 + 1, l, m, ln1);
      } else {
        upd(i * 2 + 2, m, r, ln1);
      }
    }
  }

  int res(int i, int l, int r, int x) {
    if (l + 1 == r) {
      return st[i].f(x);
    }
    int m = (l + r) / 2;
    int val = st[i].f(x);
    if (x < m) {
      val = max(val, res(i * 2 + 1, l, m, x));
    } else {
      val = max(val, res(i * 2 + 2, m, r, x));
    }
    return val;
  }
};
\end{lstlisting}

\subsection{SOS-dp}
\begin{lstlisting}[language=C++]
// dp initial fill, a[] is given array, mb extra zeros
for (int i = 0; i < (1 << N); i++) {
  dp[i] = a[i];
}

// Classic SOS-dp, goal: dp[mask] = \sum a[submasks of mask]
for (int i = 0; i < N; i++) {
  for (int mask = 0; mask < (1 << N); mask++) {
    if ((mask >> i) & 1) {
      dp[mask] += dp[mask ^ (1 << i)];
    }
  }
}

// Overmasks SOS-dp, goal: dp[mask] = \sum a[overmasks of mask]
for (int i = 0; i < N; i++) {
  for (int mask = (1 << N) - 1; mask >= 0; mask--) {
    if (((mask >> i) & 1) == 0) {
      dp[mask] += dp[mask ^ (1 << i)];
    }
  }
}

// to inverse SOS-dp (restore original array by SOS-dp array):
// use same code, but -= instead of += in dp transitions
\end{lstlisting}

\subsection{НВП}
\begin{lstlisting}[language=C++]
// 0-indexation ({$a_0, ..., a_{n-1}$})
vector<int> lis(vector<int> a) {
  int n = (int) a.size();
  vector<int> dp(n + 1, INF), ind(n + 1), par(n + 1); // INF > all a[i] required
  ind[0] = -INF;
  dp[0] = -INF;
  for (int i = 0; i < n; i++) {
    int l = upper_bound(dp.begin(), dp.end(), a[i]) - dp.begin();
    if (dp[l - 1] < a[i] && a[i] < dp[l]) {
      dp[l] = a[i];
      ind[l] = i;
      par[i] = ind[l - 1];
    }
  }
  vector<int> ans; // exact values
  for (int l = n; l >= 0; l--) {
    if (dp[l] < INF) {
      int pi = ind[l];
      ans.resize(l);
      for (int i = 0; i < l; i++) {
        ans[i] = a[pi]; // =pi if need indices
        pi = par[pi];
      }
      reverse(ans.begin(), ans.end());
      return ans;
    }
  }
  return {};
}
\end{lstlisting}

\subsection{НОВП}
\begin{lstlisting}[language=C++]
// 1-indexation ({$0, a_1, ..., a_n$}, {$0, b_1, ..., b_m$})
vector<int> lcis(vector<int> a, vector<int> b) {
  int n = (int) a.size() - 1, m = (int) b.size() - 1;
  vector<int> dp(m + 1), dp2(m + 1), par(m + 1);
  for (int i = 1; i <= n; i++) {
    int best = 0, best_idx = 0;
    for (int j = 1; j <= m; j++) {
      dp2[j] = dp[j];
      if (a[i] == b[j]) {
        dp2[j] = max(dp2[j], best + 1);
        par[j] = best_idx;
      }
      if (a[i] > b[j] && best < dp[j]) {
        best = dp[j];
        best_idx = j;
      }
    }
    swap(dp, dp2);
  }
  int pj = 0;
  for (int j = 1; j <= m; j++) {
    if (dp[pj] < dp[j]) {
      pj = j;
    }
  }
  vector<int> ans; // exact values
  while (pj > 0) {
    ans.push_back(b[pj]);
    pj = par[pj];
  }
  reverse(ans.begin(), ans.end());
  return ans;
}
\end{lstlisting}

\section{Деревья}

\subsection{Centroid}
\begin{lstlisting}[language=C++]
int levels[MAXN];
int szs[MAXN];
int cent_par[MAXN];

int calcsizes(int v, int p) {
  int sz = 1;
  for (int u : graph[v]) {
    if (u != p && levels[u] == 0)
      sz += calcsizes(u, v);
  }
  return szs[v] = sz;
}

void centroid(int v, int lvl=1, int p=-1) {
  int sz = calcsizes(v, -1);
  int nxt = v, prv;
  while (nxt != -1) {
    prv = v, v = nxt, nxt = -1;
    for (int u : graph[v]) {
      if (u != prv && levels[u] == 0 && szs[u] * 2 >= sz)
        nxt = u;
    }
  }
  levels[v] = lvl;
  cent_par[v] = p;
  for (int u : graph[v]) {
    if (levels[u] == 0)
      centroid(u, lvl + 1, v);
  }
  // calc smth for centroid v
}
\end{lstlisting}

\subsection{HLD}
\begin{lstlisting}[language=C++]
int par[MAXN], sizes[MAXN];
int pathup[MAXN];
int tin[MAXN], tout[MAXN];
int timer;

int dfs1_hld(int v, int p) {
  par[v] = p;
  int sz = 1;
  for (int i = 0; i < graph[v].size(); ++i) {
    int u = graph[v][i];
    if (u == p) {
      swap(graph[v][i--], graph[v].back());
      graph[v].pop_back();
      continue;
    }
    sz += dfs1_hld(u, v);
  }
  return sizes[v] = sz;
}

void dfs2_hld(int v, int up) {
  tin[v] = timer++;
  pathup[v] = up;
  if (graph[v].empty()) {
    tout[v] = timer;
    return;
  }
  for (int i = 1; i < graph[v].size(); ++i) {
    if (sizes[graph[v][i]] > sizes[graph[v][0]])
      swap(graph[v][i], graph[v][0]);
  }
  dfs2_hld(graph[v][0], up);
  for (int i = 1; i < graph[v].size(); ++i)
    dfs2_hld(graph[v][i], graph[v][i]);
  tout[v] = timer;
}

bool is_ancestor(int v, int p) {
  return tin[p] <= tin[v] && tout[v] <= tout[p];
}

// get_hld полностью аналогичный
void update_hld(int v, int u, int ARG) {
  for (int _ = 0; _ < 2; ++_) {
    while (!is_ancestor(u, pathup[v])) {
      int vup = pathup[v];
      ST.update(0, 0, timer, tin[vup], tin[v] + 1, ARG);
      v = par[vup];
    }
    swap(v, u);
  }
  if (tin[v] > tin[u])
    swap(v, u);
  // v = lca
  ST.update(0, 0, timer, tin[v], tin[u] + 1, ARG);
}

signed main() {
  dfs1_hld(0, -1);
  dfs2_hld(0, 0);
  ST.build();
  // your code here
}
\end{lstlisting}

\subsection{Link-cut}
\begin{lstlisting}[language=C++]
struct Node {
  Node *ch[2];
  Node *p;
  bool rev;
  int sz;

  Node() {
    ch[0] = nullptr;
    ch[1] = nullptr;
    p = nullptr;
    rev = false;
    sz = 1;
  }
};

int size(Node *v) {
  return (v ? v->sz : 0);
}

int chnum(Node *v) {
  return v->p->ch[1] == v;
}

bool isroot(Node *v) {
  return v->p == nullptr || v->p->ch[chnum(v)] != v;
}

void push(Node *v) {
  if (v->rev) {
    if (v->ch[0])
      v->ch[0]->rev ^= 1;
    if (v->ch[1])
      v->ch[1]->rev ^= 1;
    swap(v->ch[0], v->ch[1]);
    v->rev = false;
  }
}

void pull(Node *v) {
  v->sz = size(v->ch[1]) + size(v->ch[0]) + 1;
}

void attach(Node *v, Node *p, int num) {
  if (p)
    p->ch[num] = v;
  if (v)
    v->p = p;
}

void rotate(Node *v) {
  Node *p = v->p;
  push(p);
  push(v);
  int num = chnum(v);
  Node *u = v->ch[1 - num];
  if (!isroot(v->p))
    attach(v, p->p, chnum(p));
  else
    v->p = p->p;
  attach(u, p, num);
  attach(p, v, 1 - num);
  pull(p);
  pull(v);
}

void splay(Node *v) {
  push(v);
  while (!isroot(v)) {
    if (!isroot(v->p)) {
      if (chnum(v) == chnum(v->p))
        rotate(v->p);
      else
        rotate(v);
    }
    rotate(v);
  }
}

void expose(Node *v) {
  splay(v);
  v->ch[1] = nullptr;
  pull(v);
  while (v->p != nullptr) {
    Node *p = v->p;
    splay(p);
    attach(v, p, 1);
    pull(p);
    splay(v);
  }
}

void makeroot(Node *v) {
  expose(v);
  v->rev ^= 1;
  push(v);
}

void link(Node *v, Node *u) {
  makeroot(v);
  makeroot(u);
  u->p = v;
}

void cut(Node *v, Node *u) {
  makeroot(u);
  makeroot(v);
  v->ch[1] = nullptr;
  u->p = nullptr;
}

int get(Node *v, Node *u) {
  makeroot(u);
  makeroot(v);
  Node *w = u;
  while (!isroot(w))
    w = w->p;
  return (w == v ? size(v) - 1 : -1);
}

const int MAXN = 100010;
Node *nodes[MAXN];

int main() {
  int n, q;
  cin >> n >> q;
  for (int i = 0; i < n; ++i)
    nodes[i] = new Node();
  while (q--) {
    string s;
    int a, b;
    cin >> s >> a >> b;
    a--, b--;
    if (s[0] == 'g')
      cout << get(nodes[a], nodes[b]) << '\n';
    else if (s[0] == 'l')
      link(nodes[a], nodes[b]);
    else
      cut(nodes[a], nodes[b]);
  }
}

\end{lstlisting}

\section{Другое}

\subsection{Fast mod}
\begin{lstlisting}[language=C++]
// Быстрое взятие по НЕ константному модулю (в 2-4 раза быстрее)
struct FastMod {
  ull b, m;
  FastMod(ull b) : b(b), m(-1ULL / b) {}
  ull mod(ull a) const {
      ull r = a - (ull)((__uint128_t(m) * a) >> 64) * b;
      return r;  // r in [0, 2b) // $\approx$ x3.5 speed
      return r >= b ? r - b : r; // $\approx$ x3 speed
  }
}; // Usage:
// FastMod F(m);
// ull x_mod_m = F.mod(x);
\end{lstlisting}

\subsection{Slope trick}
\begin{lstlisting}[language=C++]
// Дан массив $a_n$. Сделать минимальное кол-во $\pm 1$, чтобы $a_n$ стал неубывающим.

void solve() {
  int n;
  cin >> n;
  vector<int> a(n);
  for (int i = 0; i < n; i++) {
    cin >> a[i];
  }
  int ans = 0;
  multiset<int> now;
  for (int i = 0; i < n; i++) {
    now.insert(a[i]);
    ans += (*now.rbegin() - a[i]);
    now.erase(now.find(*now.rbegin()));
    now.insert(a[i]);
  }
  cout << ans << '\n';
}
\end{lstlisting}

\subsection{attribute\_packed}
\begin{lstlisting}[language=C++]
struct Kek {
  int a;
  char b;
  // char[3]
  int c;
} __attribute__((packed));
// sizeof = 9 (instead of 12)
\end{lstlisting}

\subsection{custom\_bitset}
\begin{lstlisting}[language=C++]
// __builtin_ctz = Count Trailing Zeroes
// __builtin_clz = Count Leading Zeroes
// both are UB in gcc when pass 0
struct custom_bitset {
  vector<uint64_t> bits;
  int b, n;

  custom_bitset(int _b = 0) {
    init(_b);
  }
  void init(int _b) {
    b = _b, n = (b + 63) / 64;
    bits.assign(n, 0);
  }
  void clear() {
    b = n = 0;
    bits.clear();
  }
  void reset() {
    bits.assign(n, 0);
  }
  void _clean() {
    // Reset all bits after `b`.
    if (b != 64 * n)
      bits.back() &= (1LLU << (b - 64 * (n - 1))) - 1;
  }
  bool get(int i) const {
    return bits[i / 64] >> (i % 64) & 1;
  }
  void set(int i, bool value) {
    // assert(0 <= i && i < b);
    bits[i / 64] &= ~(1LLU << (i % 64));
    bits[i / 64] |= uint64_t(value) << (i % 64);
  }

  // Simulates `bs |= bs << shift;`
  // `|=` can be replaced with `^=`, `&=`, `=`
  void or_shift_left(int shift) {
    int div = shift / 64, mod = shift % 64;
    if (mod == 0) {
      for (int i = n - 1; i >= div; i--)
        bits[i] |= bits[i - div];
    } else {
      for (int i = n - 1; i >= div + 1; i--)
        bits[i] |= bits[i - (div + 1)] >> (64 - mod) | bits[i - div] << mod;
      if (div < n)
        bits[div] |= bits[0] << mod;
    }
    // if `&=`, `=`
    //fill(bits.begin(), bits.begin() + min(div, n), 0);
    _clean();
  }

  // Simulates `bs |= bs >> shift;`
  // `|=` can be replaced with `^=`, `&=`, `=`
  void or_shift_right(int shift) {
    int div = shift / 64, mod = shift % 64;
    if (mod == 0) {
      for (int i = div; i < n; i++)
        bits[i - div] |= bits[i];
    } else {
      for (int i = 0; i < n - (div + 1); i++)
        bits[i] |= bits[i + (div + 1)] << (64 - mod) | bits[i + div] >> mod;
      if (div < n)
        bits[n - div - 1] |= bits[n - 1] >> mod;
    }
    // if `&=`, `=`
    //fill(bits.end() - min(div, n), bits.end(), 0);
    _clean();
  }

  // find min j, that j >= i and bs[j] = 1;
  int find_next(int i) {
    if (i >= b) return b;
    int div = i / 64, mod = i % 64;
    auto x = bits[div] >> mod;
    if (x != 0)
      return i + __builtin_ctzll(x);
    for (auto k = div + 1; k < n; ++k) {
      if (bits[k] != 0)
        return 64 * k + __builtin_ctzll(bits[k]);
    }
    return b;
  }

  // `|=` can be replaced with `&=`, `^=`
  custom_bitset &operator|=(const custom_bitset &other){
    // assert(b == other.b);
    for (int i = 0; i < n; i++)
      bits[i] |= other.bits[i];
    return *this;
  }
};
\end{lstlisting}

\subsection{ordered\_set}
\begin{lstlisting}[language=C++]
#include <ext/pb_ds/assoc_container.hpp>
#include <ext/pb_ds/tree_policy.hpp>

using namespace __gnu_pbds;

typedef tree<int, null_type, less<>, rb_tree_tag, tree_order_statistics_node_update> ordered_set;

//st.find_by_order(index);
//st.order_of_key(key);

\end{lstlisting}

\subsection{pragma}
\begin{lstlisting}[language=C++]
#pragma GCC optimize("Ofast,fast-math,unroll-loops,no-stack-protector,inline")
#pragma GCC target("sse,sse2,sse3,ssse3,sse4,sse4.1,sse4.2,avx,avx2,abm,mmx,popcnt")

\end{lstlisting}

\subsection{Аллокатор Копелиовича}
\begin{lstlisting}[language=C++]
// Код вставить до инклюдов

#include <cassert>

const int MAX_MEM = 1e8; // ~100mb
int mpos = 0;
char mem[MAX_MEM];

inline void *operator new(std::size_t n) {
  mpos += n;
//  assert(mpos <= MAX_MEM);
  return (void *)(mem + mpos - n);
}

inline void operator delete(void *) noexcept {} // must have!
inline void operator delete(void *, std::size_t) noexcept {} // fix!!
\end{lstlisting}

\subsection{Альфа-бета отсечение}
\begin{lstlisting}[language=C++]
int alphabeta(int player, int alpha, int beta, int depth) {
  if (depth == 0) {
    // return current position score
  }
  if (player == 0) { // maximization player
    int val = -INF;
    for (auto move : possible_moves) {
      val = max(val, alphabeta(1, alpha, beta, depth - 1));
      if (val > beta) break;
      alpha = max(alpha, val);
    }
    return val;
  } else {
    int val = INF;
    for (auto move : possible_moves) {
      val = min(val, alphabeta(0, alpha, beta, depth - 1));
      if (val < alpha) break;
      beta = min(beta, val);
    }
    return val;
  }
}
\end{lstlisting}

\subsection{Отжиг}
\begin{lstlisting}[language=C++]
const double lambda = 0.999;
double temprature = 1;
mt19937 rnd(777);

double gen_rand_01() {
  return rnd() / (double) UINT32_MAX;
}
bool f(int delta) {
  return exp(-delta / temprature) > gen_rand_01();
}
void make_change() {
  temprature *= lambda;
  // calc change score
  if (change_score <= 0 || f(change_score)) {
    score += change_score;
    // make change
  }
}
\end{lstlisting}

\section{Математика}

\subsection{AdivB cmp CdivD}
\begin{lstlisting}[language=C++]
char sign(ll x) {
  return x < 0 ? -1 : x > 0;
}

// -1 = less, 0 = equal, 1 = greater
char compare(ll a, ll b, ll c, ll d) {
  if (a / b != c / d)
    return sign(a / b - c / d);
  a = a % b;
  c = c % d;
  if (a == 0)
    return -sign(c) * sign(d);
  if (c == 0)
    return sign(a) * sign(b);
  return compare(d, c, b, a) * sign(a) * sign(b) * sign(c) * sign(d);
}

\end{lstlisting}

\subsection{Berlecamp}
\begin{lstlisting}[language=C++]
int getkfps(vector<ll> p, vector<ll> q, ll k) {
  // assert(q[0] != 0);
  while (k) {
    auto f = q;
    for (int i = 1; i < (int) f.size(); i += 2) {
      f[i] = (MOD - f[i] % MOD) % MOD;
    }
    auto p2 = convMod(p, f);
    auto q2 = convMod(q, f);
    p.clear(), q.clear();
    for (int i = k % 2; i < (int) p2.size(); i += 2) {
      p.pb(p2[i]);
    }
    for (int i = 0; i < (int) q2.size(); i += 2) {
      q.pb(q2[i]);
    }
    k >>= 1;
  }
  return (int) ((p[0] * inverse(q[0])) % MOD);
}

// a - initials values of sequence, s - result of berlekamp on a
int kth_term(vector<ll> &a, vector<ll> s, ll k) {
  int d = ssize(s) - 1;
  s[0] = MOD - 1;
  while (s.back() == 0) {
    s.pop_back();
  }
  for (auto &el: s) {
    el = (MOD - el % MOD) % MOD;
  }
  vector<ll> p(d);
  copy(a.begin(), a.begin() + d, p.begin());
  p = convMod(p, s);
  p.resize(d);
  return getkfps(p, s, k);
}

vector<ll> berlekamp_massey(vector<ll> a) {
  // given a[0]...a[n], returns sequence s[1]..s[k] s.t a[i] = a[i-1] \cdot s[1] + \ldots + a[i-k] \cdot s[k]
  vector <ll> ls, s;
  int lf = 0, d = 0;
  for (int i = 0; i < a.size(); ++i) {
    ll t = 0;
    for (int j = 0; j < s.size(); ++j) {
      t = (t + 1ll * a[i - j - 1] * s[j]) % MOD;
    }
    if ((t - a[i]) % MOD == 0)continue;
    if (s.empty()) {
      s.resize(i + 1);
      lf = i;
      d = (t - a[i]) % MOD;
      continue;
    }
    ll k = -(a[i] - t) * inverse(d) % MOD;
    vector<ll> c(i - lf - 1);
    c.push_back(k);
    for (auto &j: ls)
      c.push_back(-j * k % MOD);
    if (c.size() < s.size())
      c.resize(s.size());
    for (int j = 0; j < s.size(); ++j) {
      c[j] = (c[j] + s[j]) % MOD;
    }
    if (i - lf + (int) ls.size() >= (int) s.size()) {
      tie(ls, lf, d) = make_tuple(s, i, (t - a[i]) % MOD);
    }
    s = c;
  }
  s.insert(s.begin(), 0); // fictive s[0] = 0
  for (auto &i: s)
    i = (i % MOD + MOD) % MOD;
  return s;
}
\end{lstlisting}

\subsection{FFT mod}
\begin{lstlisting}[language=C++]
const int MOD = 998244353; // $7 \cdot 17 \cdot 2^{23} + 1$
const int G = 3;
//const int MOD = 7340033; // $7 \cdot 2^{20} + 1$
//const int G = 5;
//const int MOD = 469762049; // $7 \cdot  2^{26} + 1$
//const int G = 30;
const int MAXLOG = 23;
int W[(1 << MAXLOG) + 10];
bool nttinit = false;
vector<int> pws;

// int add(), int sub(), int mul(),
// int binpow(), int inv()

void initNTT() {
  if (nttinit) return;
  nttinit = true;
  assert((MOD - 1) % (1 << MAXLOG) == 0);
  pws.push_back(binpow(G, (MOD - 1) / (1 << MAXLOG)));
  for (int i = 0; i < MAXLOG - 1; ++i)
    pws.push_back(mul(pws.back(), pws.back()));
  assert(pws.back() == MOD - 1);
  W[0] = 1;
  for (int i = 1; i < (1 << MAXLOG); ++i)
    W[i] = mul(W[i - 1], pws[0]);
}

void ntt(int n, vector<int> &a, bool rev) {
  initNTT();
  int lg = 31 - __builtin_clz(n);
  vector<int> rv(n);
  for (int i = 1; i < n; ++i) {
    rv[i] = (rv[i >> 1] >> 1) ^ ((i & 1) << (lg - 1));
    if (rv[i] > i) swap(a[i], a[rv[i]]);
  }
  int num = MAXLOG - 1;
  for (int len = 1; len < n; len *= 2, --num) {
    for (int i = 0; i < n; i += 2 * len) {
      for (int j = 0; j < len; ++j) {
        int u = a[i + j];
        int v = mul(W[j << num], a[i + j + len]);
        a[i + j] = add(u, v);
        a[i + j + len] = sub(u, v);
      }
    }
  }
  if (rev) {
    int invn = binpow(n, MOD - 2);
    for (int i = 0; i < n; ++i) a[i] = mul(a[i], invn);
    reverse(a.begin() + 1, a.end());
  }
}

vector<int> conv(vector<int> a, vector<int> b) {
  if (a.empty() || b.empty())
    return {};
  int lg = 32 - __builtin_clz(a.size() + b.size() - 1);
  int n = 1 << lg;
  a.resize(n);
  b.resize(n);
  ntt(n, a, false);
  ntt(n, b, false);
  for (int i = 0; i < n; ++i)
    a[i] = mul(a[i], b[i]);
  ntt(n, a, true);
  while (a.size() > 1 && a.back() == 0)
    a.pop_back();
  return a;
}

vector<int> add(vector<int> a, vector<int> b) {
  a.resize(max(a.size(), b.size()));
  for (int i = 0; i < (int) b.size(); ++i)
    a[i] = add(a[i], b[i]);
  return a;
}

vector<int> sub(vector<int> a, vector<int> b) {
  a.resize(max(a.size(), b.size()));
  for (int i = 0; i < (int) b.size(); ++i)
    a[i] = sub(a[i], b[i]);
  return a;
}

vector<int> inv(const vector<int> &a, int need) {
  vector<int> b = {inv(a[0])};
  while ((int) b.size() < need) {
    vector<int> a1 = a;
    int m = b.size();
    a1.resize(min((int) a1.size(), 2 * m));
    b = conv(b, sub({2}, conv(a1, b)));
    b.resize(2 * m);
  }
  b.resize(need);
  return b;
}

vector<int> div(vector<int> a, vector<int> b) {
  if (count(all(a), 0) == a.size())
    return {0};
  assert(a.back() != 0 && b.back() != 0);
  int n = a.size() - 1;
  int m = b.size() - 1;
  if (n < m)
    return {0};
  reverse(all(a));
  reverse(all(b));
  a.resize(n - m + 1);
  b.resize(n - m + 1);
  vector<int> c = inv(b, b.size());
  vector<int> q = conv(a, c);
  q.resize(n - m + 1);
  reverse(all(q));
  return q;
}

vector<int> mod(vector<int> a, vector<int> b) {
  auto res = sub(a, conv(b, div(a, b)));
  while (res.size() > 1 && res.back() == 0)
    res.pop_back();
  return res;
}

vector<int> multipoint(vector<int> a, vector<int> x) {
  int n = x.size();
  vector<vector<int>> tree(2 * n);
  for (int i = 0; i < n; ++i)
    tree[i + n] = {x[i], MOD - 1};
  for (int i = n - 1; i; --i)
    tree[i] = conv(tree[2 * i], tree[2 * i + 1]);
  tree[1] = mod(a, tree[1]);
  for (int i = 2; i < 2 * n; ++i)
    tree[i] = mod(tree[i >> 1], tree[i]);
  vector<int> res(n);
  for (int i = 0; i < n; ++i)
    res[i] = tree[i + n][0];
  return res;
}

vector<int> deriv(vector<int> a) {
  for (int i = 1; i < (int) a.size(); ++i)
    a[i - 1] = mul(i, a[i]);
  a.back() = 0;
  if (a.size() > 1)
    a.pop_back();
  return a;
}

vector<int> integ(vector<int> a) {
  a.push_back(0);
  for (int i = (int) a.size() - 1; i; --i)
    a[i] = mul(a[i - 1], inv(i));
  a[0] = 0;
  return a;
}

vector<int> log(vector<int> a, int n) {
  assert(a[0] == 1);
  auto res = integ(conv(deriv(a), inv(a, n)));
  res.resize(n);
  return res;
}

vector<int> exp(vector<int> a, int need) {
  assert(a[0] == 0);
  vector<int> b = {1};
  while ((int) b.size() < need) {
    vector<int> a1 = a;
    int m = b.size();
    a1.resize(min((int) a1.size(), 2 * m));
    a1[0] = add(a1[0], 1);
    b = conv(b, sub(a1, log(b, 2 * m)));
    b.resize(2 * m);
  }
  b.resize(need);
  return b;
}
\end{lstlisting}

\subsection{FFT}
\begin{lstlisting}[language=C++]
const double PI = acos(-1);
const int LOG = 20;
const int MAXN = 1 << LOG;

//using comp = complex<double>;
struct comp {
  double x, y;
  comp() : x(0), y(0) {}
  comp(double x, double y) : x(x), y(y) {}
  comp(int x) : x(x), y(0) {}
  comp operator+(const comp &o) const { return {x + o.x, y + o.y}; }
  comp operator-(const comp &o) const { return {x - o.x, y - o.y}; }
  comp operator*(const comp &o) const { return {x * o.x - y * o.y, x * o.y + y * o.x}; }
  comp operator/(const int k) const { return {x / k, y / k}; }
  comp conj() const { return {x, -y}; }
};

comp OMEGA[MAXN + 10];
int tail[MAXN + 10];

comp omega(int n, int k) {
  return OMEGA[MAXN / n * k];
}

int gettail(int x, int lg) {
  return tail[x] >> (LOG - lg);
}

void calcomega() {
  for (int i = 0; i < MAXN; ++i) {
    double x = 2 * PI * i / MAXN;
    OMEGA[i] = {cos(x), sin(x)};
  }
}

void calctail() {
  tail[0] = 0;
  for (int i = 1; i < MAXN; ++i)
    tail[i] = (tail[i >> 1] >> 1) | ((i & 1) << (LOG - 1));
}

void fft(vector<comp> &A, int lg) {
  int n = A.size();
  for (int i = 0; i < n; ++i) {
    int j = gettail(i, lg);
    if (i < j)
      swap(A[i], A[j]);
  }
  for (int len = 2; len <= n; len *= 2) {
    for (int i = 0; i < n; i += len) {
      for (int j = 0; j < len / 2; ++j) {
        auto v = A[i + j];
        auto u = A[i + j + len / 2] * omega(len, j);
        A[i + j] = v + u;
        A[i + j + len / 2] = v - u;
      }
    }
  }
}

void fft2(vector<comp> &A, vector<comp> &B, int lg) {
  int n = A.size();
  vector<comp> C(n);
  for (int i = 0; i < n; ++i) {
    C[i].x = A[i].x;
    C[i].y = B[i].x;
  }
  fft(C, lg);
  C.push_back(C[0]);
  for (int i = 0; i < n; ++i) {
    A[i] = (C[i] + C[n - i].conj()) / 2;
    B[i] = (C[i] - C[n - i].conj()) / 2 * comp(0, -1);
  }
}

void invfft(vector<comp> &A, int lg) {
  int n = 1 << lg;
  fft(A, lg);
  for (auto &el : A)
    el = el / n;
  reverse(A.begin() + 1, A.end());
}

vector<int> mul(vector<int> &a, vector<int> &b) {
  if (a.empty() || b.empty())
    return {};
  int lg = 32 - __builtin_clz(a.size() + b.size() - 1);
  int n = 1 << lg;
  vector<comp> A(n, 0), B(n, 0);
  for (int i = 0; i < a.size(); ++i)
    A[i] = a[i];
  for (int i = 0; i < b.size(); ++i)
    B[i] = b[i];
//  fft2(A, B, lg);
  fft(A, lg);
  fft(B, lg);
  for (int i = 0; i < n; ++i)
    A[i] = A[i] * B[i];
  invfft(A, lg);
  vector<int> c(n);
  for (int i = 0; i < n; ++i)
    c[i] = round(A[i].x);
  while (!c.empty() && c.back() == 0)
    c.pop_back();
  return c;
}

signed main() {
  calcomega(); // НЕ ЗАБЫТЬ
  calctail(); // НЕ ЗАБЫТЬ
  // your code here
}

\end{lstlisting}

\subsection{Floor Sum}
\begin{lstlisting}[language=C++]
int floor_sum(int n, int d, int m, int a) {
  // sum_{i=0}^{n-1} floor((a + i*m)/d), only non-negative integers!
  int ans = 0;
  ans += (n * (n - 1) / 2) * (m / d);
  m %= d;
  ans += n * (a / d);
  a %= d;
  int l = m * n + a;
  if (l >= d)
    ans += floor_sum(l / d, m, d, l % d);
  return ans;
}
\end{lstlisting}

\subsection{GCD LCM свёртки}
\begin{lstlisting}[language=C++]
vector<int> gcd_convolution(vector<int> a, vector<int> b) {
  // a,b is 1-indexed array; a[0],b[0] doesnt matter
  int n = ssize(a) - 1;
  for (int p = 2; p <= n; ++p) {
    if (!isprime[p]) continue; // alt: for (auto p : primes) if (p>n)break;
    for (int i = n / p; i > 0; --i) {
      a[i] += a[i * p];
      if (a[i] >= mod) a[i] -= mod;
      b[i] += b[i * p];
      if (b[i] >= mod) b[i] -= mod;
    }
  }
  for (int i = 1; i <= n; ++i) a[i] = (a[i] * b[i]) % mod;
  for (int p = 2; p <= n; ++p) {
    if (!isprime[p]) continue;
    for (int i = 1; i * p <= n; ++i) {
      a[i] += mod - a[i * p];
      if (a[i] >= mod) a[i] -= mod;
    }
  }
  return a;
}

vector<int> lcm_convolution(vector<int> a, vector<int> b) {
  int n = ssize(a) - 1;
  for (int p = 2; p <= n; ++p) {
    if (!isprime[p]) continue;
    for (int i = 1; i * p <= n; ++i) {
      a[i * p] += a[i];
      if (a[i * p] >= mod) a[i * p] -= mod;
      b[i * p] += b[i];
      if (b[i * p] >= mod) b[i * p] -= mod;
    }
  }
  for (int i = 1; i <= n; ++i) a[i] = (a[i] * b[i]) % mod;
  for (int p = 2; p <= n; ++p) {
    if (!isprime[p]) continue;
    for (int i = n / p; i > 0; --i) {
      a[i * p] += mod - a[i];
      if (a[i * p] >= mod) a[i * p] -= mod;
    }
  }
  return a;
}
\end{lstlisting}

\subsection{OR XOR AND Свёртки}
\begin{lstlisting}[language=C++]
vector<int> or_conv(int n, vector<int> a, vector<int> b) { // $|a| = |b| = 2^n$
  for (int i = 0; i < n; i++) {
    for (int j = 0; j < (1 << n); j++) {
      if ((j >> i) & 1) {
        a[j] = (a[j] + a[j ^ (1 << i)]) % MOD;
        b[j] = (b[j] + b[j ^ (1 << i)]) % MOD;
      }
    }
  }
  vector<int> c(1 << n);
  for (int i = 0; i < (1 << n); i++) {
    c[i] = (a[i] * b[i]) % MOD;
  }
  for (int i = n - 1; i >= 0; i--) {
    for (int j = (1 << n) - 1; j >= 0; j--) {
      if ((j >> i) & 1) {
        c[j] = (c[j] - c[j ^ (1 << i)] + MOD) % MOD;
      }
    }
  }
  return c;
}

vector<int> and_conv(int n, vector<int> a, vector<int> b) { // $|a| = |b| = 2^n$
  for (int i = 1; i < (1 << n); i *= 2) {
    for (int j = 0; j < (1 << n); j += i * 2) {
      for (int k = 0; k < i; k++) {
        a[j + k] = (a[j + k] + a[i + j + k]) % MOD;
        b[j + k] = (b[j + k] + b[i + j + k]) % MOD;
      }
    }
  }
  vector<int> c(1 << n);
  for (int i = 0; i < (1 << n); i++)
    c[i] = (a[i] * b[i]) % MOD;
  for (int i = 1; i < (1 << n); i *= 2) {
    for (int j = 0; j < (1 << n); j += i * 2) {
      for (int k = 0; k < i; k++) {
        c[j + k] = (c[j + k] - c[i + j + k] + MOD) % MOD;
      }
    }
  }
  return c;
}

const int inv2 = (MOD + 1) / 2;
vector<int> xor_conv(int n, vector<int> a, vector<int> b) { // $|a| = |b| = 2^n$
  for (int i = 1; i < (1 << n); i *= 2) {
    for (int j = 0; j < (1 << n); j++) {
      if ((j & i) == 0) {
        int x = a[j], y = a[j | i];
        a[j] = (x + y) % MOD, a[j | i] = (x - y + MOD) % MOD;
        x = b[j], y = b[j | i];
        b[j] = (x + y) % MOD, b[j | i] = (x - y + MOD) % MOD;
      }
    }
  }
  vector<int> c(1 << n);
  for (int i = 0; i < (1 << n); i++)
    c[i] = (a[i] * b[i]) % MOD;
  for (int i = 1; i < (1 << n); i *= 2) {
    for (int j = 0; j < (1 << n); j++) {
      if ((j & i) == 0) {
        int x = c[j], y = c[j | i];
        c[j] = (inv2 * (x + y)) % MOD, c[j | i] = (inv2 * (x - y + MOD)) % MOD;
      }
    }
  }
  return c;
}
\end{lstlisting}

\subsection{convMod}
\begin{lstlisting}[language=C++]
vector<ll> convMod(const vector<ll> &a, const vector<ll> &b) {
  if (a.empty() || b.empty()) return {};
  vector <ll> res((int) a.size() + (int) b.size() - 1);
  int lg = 32 - __builtin_clz((int) res.size()), n = 1 << lg, cut = (int) (sqrt(MOD));
  vector <comp> L(n), R(n), outs(n), outl(n);
  for (int i = 0; i < a.size(); i++) L[i] = comp((int) a[i] / cut, (int) a[i] % cut);
  for (int i = 0; i < b.size(); i++) R[i] = comp((int) b[i] / cut, (int) b[i] % cut);
  fft(L, lg), fft(R, lg);
  for (int i = 0; i < n; i++) {
    int j = -i & (n - 1);
    outl[j] = (L[i] + L[j].conj()) * R[i] / (2.0 * n);
    outs[j] = (L[i] - L[j].conj()) * R[i] / (2.0 * n) * comp(0, 1) * -1;
  }
  fft(outl, lg), fft(outs, lg);
  for (int i = 0; i < res.size(); i++) {
    ll av = (ll)((outl[i]).x + .5), cv = (ll)((outs[i]).y + .5);
    ll bv = (ll)((outl[i]).y + .5) + (ll)((outs[i]).x + .5);
    res[i] = ((av % MOD * cut + bv) % MOD * cut + cv) % MOD;
  }
  return res;
}
\end{lstlisting}

\subsection{min25 sieve}
\begin{lstlisting}[language=C++]
ll min25_sieve(ll n) {
  // given n, calculate prefix sums of some multiplicative function f
  // at all points of type floor(n/k) in O(n^{3/4}/log(n)), n up to 1e11 is ok
  // in particular you can find f(1) + ... + f(n)
  // also, calculation can be done for primes only, i.e prefix sum of f(i)*I{i is prime}
  // to do that, do not run last stage of algorithm

  vector<ll> v;
  v.reserve((int) sqrt(n) * 2 + 7);
  ll sq = 0;
  {
    ll k = 1;
    while (k * k <= n) {
      v.push_back(k);
      ++k;
    }
    --k;
    sq = k;
    if (k * k == n)--k;
    while (k >= 1) {
      v.push_back(n / k);
      --k;
    }
  }
  auto geti = [&](ll x) {
    // returns i, such that v[i] = x
    if (x <= sq) return x - 1;
    return (int) v.size() - (n / x);
  };
  // OP1: f(ab) = f(a)f(b) for coprime a, b; f(p) = p^T; f(p^k) can be calculated in O(1); we denote f(p^k) = g(p, k) (p is prime) for all k
  // OP2: f also can be any fully multiplicative function, f(ab) = f(a)f(b) for all a,b; you need to calc pref sum of f fast, so only prime case is useful

  auto g = [&](ll p, int k) {
    if (k == 1) {
      return p - 1; // polynomial, for primes-only can be any fully multiplicative function
    }
    return p + k; // any function, g(p^k)
  };

  auto f = [&](ll x) {
    return g(x, 1);
  };

  auto pref = [&](ll x) {
    // return sum_{i=1..x} g(i, 1), i.e 1^T + 2^T + ... + x^T
    return x * (x + 1) / 2;
  };

  vector<ll> s0(v.size()), s1(v.size()); // for all degrees separately
  for (int i = 0; i < (int) v.size(); i++) {
    s0[i] = v[i] % M;
    s1[i] = (((v[i] % M) * ((v[i] + 1) % M) % M) * (((M + 1) / 2) % M)) % M; // pref for g(p,1), degrees separately
    // s[i] = pref(v[i]) - 1  for primes
  }

  vector<ll> used_primes;
  used_primes.reserve((int) sqrt(n) + 7);
  for (ll p = 2; p * p <= n; ++p) {
    if (s0[p - 1] == s0[p - 2]) continue;
    // p is prime
    used_primes.push_back(p);
    for (int i = (int) v.size() - 1; i >= 0; --i) {
      if (v[i] < p * p) break; // very important, dont remove!
      s0[i] += M - ((s0[geti(v[i] / p)] + M - s0[p - 2]) % M * (1)) % M; // p^0
      s0[i] %= M;
      s1[i] += M - ((s1[geti(v[i] / p)] + M - s1[p - 2]) % M * (p)) % M; // p^1
      s1[i] %= M;
      // s[i] += M - ((s[geti(v[i] / p)] + M - s[p-2]) % M * f(p)) % M;
    }
  }

  // PRIMES ONLY calculation is done
  // desired answer for v[i] is in s[i]
  // in particular \sum_{i=1}^n f(i)*I{i is prime} is in s.back()
  // now last stage for default calculation

  vector<ll> s(v.size());
  for (int i = 0; i < v.size(); i++) {
    s[i] = (M - s0[i] % M + s1[i]) % M; // combine polynomial by degrees with needed coeffs
  }

  vector<ll> r = s;

  for (int ui = (int) used_primes.size() - 1; ui >= 0; --ui) { // ui >= 1, sum for odd numbers only
    ll p = used_primes[ui];
    for (int i = (int) v.size() - 1; i >= 0; --i) {
      if (v[i] < p * p) break; // very important, dont remove!
      for (ll c = 1, pc = p; pc * p <= v[i]; c++, pc *= p) { // pc = p^c
        r[i] += g(p, c + 1) % M + ((g(p, c) % M) * ((M + r[geti(v[i] / pc)] - s[geti(p)]) % M)) % M;
        r[i] %= M;
      }
    }
  }

  // done, answer for v[i] is r[i]+1 (f(1)=1)
  // in particular \sum_{i=2}^n f(i) is in r.back()
  // therefore \sum_{i=1}^n f(i) is r.back() + 1
  return r.back() + 1 - g(1, 1); // since f(1)=1 for real, not g(1,1): 1 is not prime
}
\end{lstlisting}

\subsection{sqrt mod}
\begin{lstlisting}[language=C++]
// p is prime
// -1 if no solution
// $x=\text{sqrt}(a, p) \implies x^2=a$ and $(-x)^2=a$
// $O(\log n)$ if $p \equiv 3 \mod 4$ else $O(\log^2 n)$
// should be changed if const p
ll sqrt(ll a, ll p) {
  a %= p;
  if (a < 0) a += p;
  if (a == 0) return 0;
  if (binpow(a, (p - 1) / 2, p) != 1)
    return -1; // no solution
  if (p % 4 == 3) return binpow(a, (p + 1) / 4, p);
  ll s = p - 1, n = 2;
  int r = 0, m;
  while (s % 2 == 0) ++r, s /= 2;
  while (binpow(n, (p - 1) / 2, p) != p - 1) ++n;
  ll x = binpow(a, (s + 1) / 2, p);
  ll b = binpow(a, s, p), g = binpow(n, s, p);
  for (;; r = m) {
    ll t = b;
    for (m = 0; m < r && t != 1; ++m) t = t * t % p;
    if (m == 0) return x;
    ll gs = binpow(g, 1LL << (r - m - 1), p);
    g = gs * gs % p;
    x = x * gs % p;
    b = b * g % p;
  }
}

\end{lstlisting}

\subsection{Гаусс}
\begin{lstlisting}[language=C++]
vector<vector<int>> gauss(vector<vector<int>> &a) {
  int n = a.size();
  int m = a[0].size();
//  int det = 1;
  for (int col = 0, row = 0; col < m && row < n; ++col) {
    for (int i = row; i < n; ++i) {
      if (a[i][col]) {
        swap(a[i], a[row]);
        if (i != row) {
//          det *= -1;
        }
        break;
      }
    }
    if (!a[row][col])
      continue;
    for (int i = 0; i < n; ++i) {
      if (i != row && a[i][col]) {
        int val = a[i][col] * inv(a[row][col]) % mod;
        for (int j = col; j < m; ++j) {
          a[i][j] -= val * a[row][j];
          a[i][j] %= mod;
        }
      }
    }
    ++row;
  }
//  for (int i = 0; i < n; ++i) det = (det * a[i][i]) % mod;
//  det = (det % mod + mod) % mod;
// result in (-mod, mod)
  return a;
}

pair<int, vector<int>> sle(vector<vector<int>> a, vector<int> b) {
  int n = a.size();
  int m = a[0].size();
  assert(n == b.size());
  for (int i = 0; i < n; ++i) {
    a[i].push_back(b[i]);
  }
  a = gauss(a);
  vector<int> x(m, 0);
  for (int i = n - 1; i >= 0; --i) {
    int leftmost = m;
    for (int j = 0; j < m; ++j) {
      if (a[i][j] != 0) {
        leftmost = j;
        break;
      }
    }
    if (leftmost == m && a[i].back() != 0) return {-1, {}};
    if (leftmost == m) continue;
    int val = a[i].back();
    for (int j = m - 1; j > leftmost; --j) {
      val -= a[i][j] * x[j];
      val %= mod;
    }
    x[leftmost] = (val * inv(a[i][leftmost]) % mod + mod) % mod;
  }
  return {1, x};
}

vector<bitset<N>> gauss_bit(vector<bitset<N>> a, int m) {
  int n = a.size();
  for (int col = 0, row = 0; col < m && row < n; ++col) {
    for (int i = row; i < n; ++i) {
      if (a[i][col]) {
        swap(a[i], a[row]);
        break;
      }
    }
    if (!a[row][col])
      continue;
    for (int i = 0; i < n; ++i)
      if (i != row && a[i][col])
        a[i] ^= a[row];
    ++row;
  }
  return a;
}
\end{lstlisting}

\subsection{Диофантовы уравнения}
\begin{lstlisting}[language=C++]
// $ax+by = \pm gcd$ if $a<0$ or $b<0$
pair<int, int> ext_gcd(int a, int b) {
  int x1 = 1, y1 = 0, x2 = 0, y2 = 1;
  while (b) {
    int k = a / b;
    x1 -= x2 * k;
    y1 -= y2 * k;
    a %= b;
    swap(x1, x2), swap(y1, y2), swap(a, b);
  }
  return {x1, y1};
}

// solve $ax+by=c$ with minimum $x \geq 0$
bool cool_ext_gcd(int a, int b, int c, int &x, int &y) {
    if (b == 0) {
        y = 0;
        if (a == 0)
            return x = 0, c == 0;
        return x = c / a, c % a == 0;
    }
    auto [x0, y0] = ext_gcd(a, b);
    int g = (ll)x0 * a + (ll)y0 * b;
    if (c % g != 0) return false;
    x = (ll)x0 * (c / g) % (b / g);
    if (x < 0) x += abs(b / g);
    y = (c - (ll)a * x) / b;
    return true;
}
\end{lstlisting}

\subsection{КТО}
\begin{lstlisting}[language=C++]
// ans % p_i = a_i
vector<vector<int>> r(k, vector<int>(k));
for (int i = 0; i < k; ++i)
  for (int j = 0; j < k; ++j)
    if (i != j)
      r[i][j] = binpow(p[i] % p[j], p[j] - 2, p[j]); // [phi(p[j]) - 1] для не простого модуля
vector<int> x(k);
for (int i = 0; i < k; ++i) {
  x[i] = a[i];
  for (int j = 0; j < i; ++j) {
    x[i] = r[j][i] * (x[i] - x[j]);
    x[i] = x[i] % p[i];
    if (x[i] < 0) x[i] += p[i];
  }
}
int ans = 0;
for (int i = 0; i < k; ++i) {
  int val = x[i];
  for (int j = 0; j < i; ++j) val *= p[j];
  ans += val;
}
\end{lstlisting}

\subsection{Код Грея}
\begin{lstlisting}[language=C++]
for (int i = 0; i < (1 << n); i++) {
  gray[i] = i ^ (i >> 1);
}
\end{lstlisting}

\subsection{Линейное решето}
\begin{lstlisting}[language=C++]
const int N = 10000000;
int lp[N + 1];
vector<int> pr;
for (int i = 2; i <= N; ++i) {
  if (lp[i] == 0) {
    lp[i] = i;
    pr.push_back(i);
  }
  for (int j = 0; j < (int) pr.size() && pr[j] <= lp[i] && i * pr[j] <= N; ++j)
    lp[i * pr[j]] = pr[j];
}
\end{lstlisting}

\subsection{Миллер Рабин}
\begin{lstlisting}[language=C++]
// works for all n < 2^64
const ll MAGIC[7] = {2, 325, 9375, 28178, 450775, 9780504, 1795265022};

bool is_prime(ll n) {
  if (n == 1) return false;
  if (n <= 3) return true;
  if (n % 2 == 0 || n % 3 == 0) return false;
  ll s = __builtin_ctzll(n - 1), d = n >> s; // $n-1 = 2^s \cdot d$
  for (auto a : MAGIC) {
    if (a % n == 0) {
      continue;
    }
    ll x = binpow(a, d, n); // a -> __int128 in binpow
    for (int _ = 0; _ < s; _++) {
      ll y = binpow(x, 2, n);  // x -> __int128 in binpow
      if (y == 1 && x != 1 && x != n - 1) {
        return false;
      }
      x = y;
    }
    if (x != 1) {
      return false;
    }
  }
  return true;
}
\end{lstlisting}

\subsection{Мёбиус}
\begin{lstlisting}[language=C++]
vector<int> mu(n + 1);
mu[1] = 1;
for (int x = 1; x <= n; x++) {
  for (int y = x + x; y <= n; y += x) mu[y] -= mu[x];
}
\end{lstlisting}

\subsection{Подсчёт прогулок}
\begin{lstlisting}[language=C++]
int count_walks_1(int b1, int b2, int p, int q) {
  // counting walks from (0, 0) to (p, q)
  // each turn x += 1 or y += 1
  // without touching y = x + b1 and y = x + b2
  // b1 < 0 < b2 must hold
  // O((p + q) / (b2 - b1))
  if (min(p, q) < 0) return 0;
  ll ans = C(p, p + q);
  ar(2) F = {p, q}, S = {p, q};
  int cf = mod - 1;
  while (true) {
    F[1] -= b1;
    swap(F[0], F[1]);
    F[1] += b1;
    S[1] -= b2;
    swap(S[0], S[1]);
    S[1] += b2;
    swap(F, S);
    int wf = C(F[0], F[0] + F[1]);
    int ws = C(S[0], S[0] + S[1]);
    ans += (cf * (ll) ((wf + ws) % mod)) % mod;
    if (wf == 0 && ws == 0) break;
    cf = mod - cf;
  }
  ans %= mod;
  return (int) ans;
}

int count_walks_2(int b1, int b2, int p, int q) {
  // counting walks from (0, 0) to (p, q)
  // each turn x += 1 and (y -= 1 or y += 1)
  // without touching y = b1 and y = b2
  // b1 < 0 < b2 must hold
  // O(p / (b2 - b1))
  if (abs(p) % 2 != abs(q) % 2) return 0;
  int p0 = (p - q) / 2, q0 = (p + q) / 2;
  return count_walks_1(b1, b2, p0, q0);
}
\end{lstlisting}

\subsection{Ро-Поллард}
\begin{lstlisting}[language=C++]
typedef long long ll;

ll mult(ll a, ll b, ll mod) {
  return (__int128)a * b % mod;
}

ll f(ll x, ll c, ll mod) {
  return (mult(x, x, mod) + c) % mod;
}

ll rho(ll n, ll x0=2, ll c=1) {
  ll x = x0;
  ll y = x0;
  ll g = 1;
  while (g == 1) {
    x = f(x, c, n);
    y = f(y, c, n);
    y = f(y, c, n);
    g = gcd(abs(x - y), n);
  }
  return g;
}

mt19937_64 rnd(time(nullptr));

void factor(int n, vector<int> &pr) {
  if (n == 4) {
    factor(2, pr);
    factor(2, pr);
    return;
  }
  if (n == 1) {
    return;
  }
  if (is_prime(n)) {
    pr.push_back(n);
    return;
  }
  int d = rho(n, rnd() % (n - 2) + 2, rnd() % 3 + 1);
  factor(n / d, pr);
  factor(d, pr);
}
\end{lstlisting}

\subsection{Чудо Формулы}
\noindent\rule{\linewidth}{0.15mm}

\begingroup
\small

\begin{enumerate}
\item % Число путей без соседних шагов:
$\sum_{0\le k\le n}\binom{n-k}{k}=\mathrm{Fib}_{n+1}$

\item % Альтернирующая сумма биномиалов:
$\sum_{i=0}^{k}(-1)^i\binom{n}{i}=(-1)^k\binom{n-1}{k}$

\item % Сумма верхних диагоналей треугольника Паскаля:
$\sum_{i=0}^{k}\binom{n+i}{i}=\binom{n+k+1}{k}$

\item % (Вандермонд) Свёртка биномиалов:
$\sum_{k=0}^{r}\binom{m}{k}\binom{n}{r-k}=\binom{m+n}{r}$

\item % «Хоккейная клюшка» (сумма биномиалов по строке):
$\sum_{i=r}^{n}\binom{i}{r}=\binom{n+1}{r+1}$

\item % (Куммер) Показатель простого $p$ в биномиале:
Степень $p$ в $\binom{n}{m}$ равна числу переносов при сложении $m$ и $n-m$ в $p$-ичной системе.

\item % Сумма отношений биномиалов:
$\sum_{i=0}^{n}\frac{\binom{k}{i}}{\binom{n}{i}}
=\frac{\binom{n+1}{\,n-k+1\,}}{\binom{n}{k}}$

\item % Рекурсия для дерранжментов (фиксированная позиция):
$d(n)=(n-1)\bigl(d(n-1)+d(n-2)\bigr),\  d(0)=1,\ d(1)=0$

\item % Теорема Лукаса (по модулю простого $p$):
$\binom{m}{n}\equiv\prod_i\binom{m_i}{n_i}\pmod p,$
\quad где $m_i,n_i$ — цифры $m,n$ в $p$-ичной записи.

\item % Стирлинг II рода в разложении степеней:
$\sum_{i=0}^{n}\binom{n}{i}i^k
=\sum_{j=0}^{k}S_2(k,j)\,\frac{n!}{(n-j)!}\,2^{\,n-j}$

\item % Частичная сумма биномиалов с весом $x^i$:
$\sum_{i=0}^{n-1}\binom{i}{j}x^i
= x^j(1-x)^{-j-1}\!\Bigl(1-x^n\sum_{i=0}^j\binom{n}{i}x^{\,j-i}(1-x)^i\Bigr)$

\item % Биномиальная инверсия (следствие):
$P(n)=\sum_{k=0}^{n}\binom{n}{k}Q(k) \implies Q(n)=\sum_{k=0}^{n}(-1)^{\,n-k}\binom{n}{k}P(k)$

\item % Инверсия со знаком (следствие):
$P(n)=\sum_{k=0}^{n}(-1)^k\binom{n}{k}Q(k) \implies Q(n)=\sum_{k=0}^{n}(-1)^k\binom{n}{k}P(k)$

\item % Стирлинговые числа первого рода (рекурсия):
$S_1(n,k)$ — число перестановок длины $n$ с $k$ циклами,
$S_1(n,k)=(n-1)S_1(n-1,k)+S_1(n-1,k-1),\ S_1(0,0)=1$

\item % Стирлинговые числа второго рода (рекурсия и частные случаи):
$S_2(n,k)$ — число разбиений $n$-множества на $k$ непустых блоков,
$S_2(n,k)=kS_2(n-1,k)+S_2(n-1,k-1),\ S_2(0,0)=1$

\item % Частные формулы для $S_2$:
$S_2(n,2)=2^{\,n-1}-1$

$S_2(n,3)=\tfrac{1}{2}\bigl(3^{\,n-1}-2^{\,n-1}\bigr)-\tfrac{1}{2}\bigl(3^{\,n-1}-1\bigr)$

$S_2(n,4)=\tfrac{1}{2!}\Bigl[(4^{\,n-1}-3^{\,n-1})-(4^{\,n-1}-2^{\,n-1})+\tfrac{1}{3}(4^{\,n-1}-1)\Bigr]$

$S_2(n,5)=\tfrac{1}{24}\,5^{\,n-1}-\tfrac{1}{6}\,4^{\,n-1}
+\tfrac{1}{4}\,3^{\,n-1}-\tfrac{1}{6}\,2^{\,n-1}+\tfrac{1}{24}$

\item % Пониженные числа Стирлинга:
$S_2^d(n,k)$ — число разбиений с расстоянием $\ge d$ внутри блока,
$S_2^d(n,k)=S_2(n-d+1,k-d+1)$

\item % Сумма $i a^i$:
$\sum_{i=1}^n i a^i=\frac{a\bigl(n a^{n+1}-(n+1)a^n+1\bigr)}{(a-1)^2}$

\item % Формула Эйлера—Пентовой (пентагональные числа):
$\prod_{n=1}^\infty(1-x^n)=1+\sum_{k=1}^{\infty}(-1)^k\bigl(x^{\,k(3k-1)/2}+x^{\,k(3k+1)/2}\bigr)$

\item % Критерий принадлежности к последовательности Фибоначчи:
$n$ — член Фибоначчи $\iff$ хотя бы одно из чисел $5n^2\pm4$ — точный квадрат.

\item % Периодичность Фибоначчи по модулю $n$:
Период последовательности Фибоначчи по модулю $n$ не превышает $6n$.

\item % Пифагоровы тройки (все целые решения $a^2+b^2=c^2$):
Пифагоровы тройки: пусть $m>n>0,\ \gcd(m,n)=1,$ и не оба нечётные. Тогда
$\;a=k(m^2-n^2),\ b=k(2mn),\ c=k(m^2+n^2)$


\item % Число решений $a^2+b^2=n^2$ с фиксированной гипотенузой:
$\displaystyle \#\{(a,b):a^2+b^2=n^2,a>0,b>0\}
=\Big(\prod_{p^\alpha\parallel n,\ p = 4k+1}2\alpha+1\Big)-1$

\item % Определения функций представления суммой квадратов:
$r_4(n)=\#\{x_1^2+\cdots+x_4^2=n\},\ 
r_8(n)=\#\{x_1^2+\cdots+x_8^2=n\}$

\item % Формулы для $r_4$ и $r_8$:
$r_4(n)=8\sum_{d\mid n, 4\nmid d}d,\ 
r_8(n)=16\sum_{d\mid n}(-1)^{\,n+d}d^3$

\item % Количество неотрицательных решений $ax+by=n$ и обратные:
$\#\{ax+by=n,\ x,y\ge0\}=\frac{n}{ab}-\Bigl\{\tfrac{b'n}{a}\Bigr\}-\Bigl\{\tfrac{a'n}{b}\Bigr\}+1$,

где $a'$ и $b'$ — обратные: $aa'\equiv1\pmod b,\; bb'\equiv1\pmod a$

\item %Формула для $\varphi(mn)$ через $\gcd$:
$\varphi(mn)=\varphi(m)\varphi(n)\frac{d}{\varphi(d)},\  d=\gcd(m,n)$

\item % Сокращение показателя по $\varphi(m)$ при $x\ge\log_2 m$:
$n^x\bmod m = n^{\varphi(m)+\bigl(x\bmod\varphi(m)\bigr)}\bmod m,\ x\ge\log_2 m$

\item % Подсчёт квадратсвободных $\le N$ через функцию Мёбиуса:
$\sum_{n=1}^N\mu^2(n)=\sum_{k=1}^{\lfloor\sqrt N\rfloor}\mu(k)\bigl\lfloor N/k^2\bigr\rfloor$

\item % Инверсия Мёбиуса (аддитивная):
$g(n)=\sum_{d\mid n}f(d) \implies f(n)=\sum_{d\mid n}\mu(d)\,g(n/d)$

\item % Инверсия Мёбиуса (мультипликативная):
$F(n)=\prod_{d\mid n}f(d) \implies f(n)=\prod_{d\mid n}F(n/d)^{\mu(d)}$

\item % Связь $\gcd/\operatorname{lcm}$ №1:
$\gcd\bigl(a,\operatorname{lcm}(b,c)\bigr)=\operatorname{lcm}\bigl(\gcd(a,b),\gcd(a,c)\bigr)$

\item % Связь $\gcd/\operatorname{lcm}$ №2:
$\operatorname{lcm}\bigl(a,\gcd(b,c)\bigr)=\gcd\bigl(\operatorname{lcm}(a,b),\operatorname{lcm}(a,c)\bigr)$

\item % Симметрия для тройки:
$\gcd\bigl(\operatorname{lcm}(a,b),\operatorname{lcm}(b,c),\operatorname{lcm}(a,c)\bigr)
 =\\= \operatorname{lcm}\bigl(\gcd(a,b),\gcd(b,c),\gcd(a,c)\bigr)$

\item % Сумма $\operatorname{lcm}$ по сетке $1\ldots n$:
$\sum_{i,j=1}^n\operatorname{lcm}(i,j)
=\sum_{l=1}^n\Bigl(\frac{(1+\lfloor n/l\rfloor)\lfloor n/l\rfloor}{2}\Bigr)^2\sum_{d\mid l}\mu(d)ld$

\item % Сумма $\operatorname{lcm}(k,n)$ по $k$:
$\sum_{i=1}^n\operatorname{lcm}(i,n)=\frac{n}{2}\Bigl(\sum_{d\mid n}\varphi(d)d+1\Bigr)$

\item % Закон квадратичной взаимности (Лежандр):
$\left(\tfrac{p}{q}\right)\left(\tfrac{q}{p}\right)=(-1)^{\frac{(p-1)(q-1)}{4}}$
 для нечётных простых $p,q$

\end{enumerate}

\endgroup


\noindent\rule{\linewidth}{0.15mm}

\section{Строки}

\subsection{Z-функция}
\begin{lstlisting}[language=C++]
vector<int> z_func(string s) {
  int n = s.size();
  vector<int> z(n, 0);
  z[0] = n;
  int l = 0, r = 0;
  for (int i = 1; i < n; i++) {
    if (i < r) {
      z[i] = min(z[i - l], r - i);
    }
    while (i + z[i] < n && s[z[i]] == s[i + z[i]]) {
      z[i]++;
    }
    if (i + z[i] > r) {
      l = i;
      r = i + z[i];
    }
  }
  return z;
}
\end{lstlisting}

\subsection{eertree}
\begin{lstlisting}[language=C++]
int len[MAXN], suf[MAXN];
int go[MAXN][ALPH];
char s[MAXN];

int n, last, sz;

void init() {
  n = 0, last = 0;
  s[n++] = -1;
  suf[0] = 1; // root of suflink tree = 1
  len[1] = -1;
  sz = 2;
}

int get_link(int v) {
  while (s[n - len[v] - 2] != s[n - 1])
    v = suf[v];
  return v;
}

void add_char(char c) {
  c -= 'a';
  s[n++] = c;
  last = get_link(last);
  if (!go[last][c]) {
    len[sz] = len[last] + 2;
    suf[sz] = go[get_link(suf[last])][c];
    go[last][c] = sz++;
  }
  last = go[last][c]; // cur v = last
}
\end{lstlisting}

\subsection{Ахо-Корасик}
\begin{lstlisting}[language=C++]
int go[MAXN][ALPH];
vector<int> term[MAXN];
int par[MAXN], suf[MAXN];
char par_c[MAXN];
vector<int> g[MAXN];

int cntv = 1;

void add(string &s) {
  static int cnt_s = 1;
  int v = 0;
  for (char el: s) {
    if (go[v][el - 'a'] == 0) {
      go[v][el - 'a'] = cntv;
      par[cntv] = v;
      par_c[cntv] = el;
      cntv++;
    }
    v = go[v][el - 'a'];
  }
  term[v].push_back(cnt_s++);
}

void bfs() {
  deque<int> q = {0};
  while (!q.empty()) {
    int v = q.front();
    q.pop_front();
    if (v > 0) {
      if (par[v] == 0) {
        suf[v] = 0;
      } else {
        suf[v] = go[suf[par[v]]][par_c[v] - 'a'];
      }
      g[suf[v]].push_back(v);
    }
    for (int c = 0; c < ALPH; c++) {
      if (go[v][c] == 0) {
        go[v][c] = go[suf[v]][c];
      } else {
        q.push_back(go[v][c]);
      }
    }
  }
}
\end{lstlisting}

\subsection{Муффиксный Сассив}
\begin{lstlisting}[language=C++]
vector<int> build_suff_arr(string &s) {
  // Remove, if you want to sort cyclic shifts
  s += (char) (1);
  int n = s.size();
  vector<int> a(n);
  iota(all(a), 0);
  stable_sort(all(a), [&](int i, int j) {
      return s[i] < s[j];
  });
  vector<int> c(n);
  int cc = 0;
  for (int i = 0; i < n; i++) {
    if (i == 0 || s[a[i]] != s[a[i - 1]])
      c[a[i]] = cc++;
    else
      c[a[i]] = c[a[i - 1]];
  }
  for (int L = 1; L < n; L *= 2) {
    vector<int> cnt(n);
    for (auto i: c) cnt[i]++;
    if (*min_element(all(cnt)) > 0) break;
    vector<int> pref(n);
    for (int i = 1; i < n; i++)
      pref[i] = pref[i - 1] + cnt[i - 1];
    vector<int> na(n);
    for (int i = 0; i < n; i++) {
      int pos = (a[i] - L + n) % n;
      na[pref[c[pos]]++] = pos;
    }
    a = na;
    vector<int> nc(n);
    cc = 0;
    for (int i = 0; i < n; i++) {
      if (i == 0 || c[a[i]] != c[a[i - 1]] ||
          c[(a[i] + L) % n] != c[(a[i - 1] + L) % n])
        nc[a[i]] = cc++;
      else
        nc[a[i]] = nc[a[i - 1]];
    }
    c = nc;
  }
  // Remove, if you want to sort cyclic shifts
  a.erase(a.begin());
  s.pop_back();
  return a;
}

vector<int> kasai(string s, vector<int> sa) {
  // lcp[i] = lcp(sa[i], sa[i + 1])
  int n = s.size(), k = 0;
  vector<int> lcp(n, 0);
  vector<int> rank(n, 0);
  for (int i = 0; i < n; i++) rank[sa[i]] = i;
  for (int i = 0; i < n; i++, k ? k-- : 0) {
    if (rank[i] == n - 1) {
      k = 0;
      continue;
    }
    int j = sa[rank[i] + 1];
    while (i + k < n && j + k < n && s[i + k] == s[j + k]) k++;
    lcp[rank[i]] = k;
  }
  return lcp;
}
\end{lstlisting}

\subsection{Префикс-функция}
\begin{lstlisting}[language=C++]
vector<int> prefix_func(string s) {
  int n = s.size();
  vector<int> pref(n, 0);
  int ans = 0;
  for (int i = 1; i < n; i++) {
    while (ans > 0 && s[ans] != s[i]) {
      ans = pref[ans - 1];
    }
    if (s[i] == s[ans]) {
      ans++;
    }
    pref[i] = ans;
  }
  return pref;
}
\end{lstlisting}

\subsection{Суффиксный автомат}
\begin{lstlisting}[language=C++]
// Суфавтомат с подсчётом кол-ва различных подстрок

const int SIGMA = 26;
int ans = 0;

struct Node {
  int go[SIGMA];
  int s, p;
  int len;

  Node() {
    fill(go, go + SIGMA, -1);
    s = -1, p = -1;
    len = 0;
  }
};

int add(int A, int ch, vector<Node> &sa) {
  int B = sa.size();
  sa.emplace_back();
  sa[B].p = A;
  sa[B].s = 0;
  sa[B].len = sa[A].len + 1;
  for (; A != -1; A = sa[A].s) {
    if (sa[A].go[ch] == -1) {
      sa[A].go[ch] = B;
      continue;
    }
    int C = sa[A].go[ch];
    if (sa[C].p == A) {
      sa[B].s = C;
      break;
    }
    int D = sa.size();
    sa.emplace_back();
    sa[D].s = sa[C].s;
    sa[D].p = A;
    sa[D].len = sa[A].len + 1;
    sa[C].s = D;
    sa[B].s = D;
    copy(sa[C].go, sa[C].go + SIGMA, sa[D].go);
    for (; A != -1 && sa[A].go[ch] == C; A = sa[A].s)
      sa[A].go[ch] = D;
    break;
  }
  ans += sa[B].len - sa[sa[B].s].len;
  return B;
}

signed main() {
  string s;
  cin >> s;
  vector<Node> sa(1);
  int A = 0;
  for (char c : s)
    A = add(A, c - 'a', sa);
  cout << ans;
}
\end{lstlisting}

\section{Структуры данных}

\subsection{Disjoint Sparse Table}
\begin{lstlisting}[language=C++]
// MAXN дополнить до степени двойки (или n*2)
int tree[LOG][MAXN];
int floorlog2[MAXN]; // i ? (31 - __builtin_clz(i)) : 0

void build(vector<int> &a) {
  int n = a.size();
  copy(a.begin(), a.end(), tree[0]);
  for (int lg = 1; lg < LOG; ++lg) {
    int len = 1 << lg;
    auto &lvl = tree[lg];
    for (int m = len; m < n; m += len * 2) {
      lvl[m - 1] = a[m - 1];
      lvl[m] = a[m];
      for (int i = m - 2; i >= m - len; --i)
        lvl[i] = min(lvl[i + 1], a[i]);
      for (int i = m + 1; i < m + len && i < n; ++i)
        lvl[i] = min(lvl[i - 1], a[i]);
    }
  }
  for (int i = 2; i < min(MAXN, n * 2); ++i)
    floorlog2[i] = floorlog2[i / 2] + 1;
}

// a[l..r)
int get(int l, int r) {
  r--;
  int i = floorlog2[l ^ r];
  return min(tree[i][l], tree[i][r]);
}

\end{lstlisting}

\subsection{Segment Tree Beats}
\begin{lstlisting}[language=C++]
// %=, =, sum
// mx[v], all_equal[v]
// break: mx[v] < x
// tag: all_equal[v] == true, запрос становится =mx[v]%x

// min=, max=, =, +=, sum, mn, mx
// также как и для min=, sum
// для max= храним mn[v], sec_mn[v]

// +=, gcd
// храним gcd разностей какого-то остовного дерева
// храним any_value[v] = любое значение на отрезке
// gcd(l...r) = gcd(any_value[v], gcd[v])
// при сливании добавляем к gcd значение |a_v[l] - a_v[r]|

// min=, sum
struct ST {
  vector<int> st, mx, mx_cnt, sec_mx;

  ST(vector<int> &a) {
    int n = a.size();
    st.resize(n * 4), mx.resize(n * 4);
    mx_cnt.resize(n * 4, 0), sec_mx.resize(n * 4, 0);
    build(0, 0, n, a);
  }

  void upd_from_children(int v) {
    st[v] = st[v * 2 + 1] + st[v * 2 + 2];
    mx[v] = max(mx[v * 2 + 1], mx[v * 2 + 2]);
    mx_cnt[v] = 0;
    sec_mx[v] = max(sec_mx[v * 2 + 1], sec_mx[v * 2 + 2]);
    if (mx[v * 2 + 1] == mx[v]) {
      mx_cnt[v] += mx_cnt[v * 2 + 1];
    } else {
      sec_mx[v] = max(sec_mx[v], mx[v * 2 + 1]);
    }
    if (mx[v * 2 + 2] == mx[v]) {
      mx_cnt[v] += mx_cnt[v * 2 + 2];
    } else {
      sec_mx[v] = max(sec_mx[v], mx[v * 2 + 2]);
    }
  }

  void build(int i, int l, int r, vector<int> &a) {
    if (l + 1 == r) {
      st[i] = mx[i] = a[l];
      mx_cnt[i] = 1;
      sec_mx[i] = -INF;
      return;
    }
    int m = (r + l) / 2;
    build(i * 2 + 1, l, m, a);
    build(i * 2 + 2, m, r, a);
    upd_from_children(i);
  }

  void push_min_eq(int v, int val) {
    if (mx[v] > val) {
      st[v] -= (mx[v] - val) * mx_cnt[v];
      mx[v] = val;
    }
  }

  void push(int i, int l, int r) {
    if (l + 1 < r) {
      push_min_eq(i * 2 + 1, mx[i]);
      push_min_eq(i * 2 + 2, mx[i]);
    }
  }

  void update(int i, int l, int r, int ql, int qr, int val) {
    if (qr <= l || r <= ql || mx[i] <= val) {
      return;
    }
    if (ql <= l && r <= qr && sec_mx[i] < val) {
      push_min_eq(i, val);
      return;
    }
    push(i, l, r);
    int m = (l + r) / 2;
    update(i * 2 + 1, l, m, ql, qr, val);
    update(i * 2 + 2, m, r, ql, qr, val);
    upd_from_children(i);
  }

  int sum(int i, int l, int r, int ql, int qr) {
    if (qr <= l || r <= ql) {
      return 0;
    }
    push(i, l, r);
    if (ql <= l && r <= qr) {
      return st[i];
    }
    int m = (l + r) / 2;
    return sum(i * 2 + 1, l, m, ql, qr) + sum(i * 2 + 2, m, r, ql, qr);
  }
};
\end{lstlisting}

\subsection{ДД по неявному}
\begin{lstlisting}[language=C++]
// потому что nds[0].sz == 0 и sz не изменяется в push
int size(int t) { return nds[t].sz; }

pair<int, int> split(int t, int k) {
  if (!t) return {0, 0};
  push(t);
  int szl = size(nds[t].l);
  if (k <= szl) {
    auto [l, r] = split(nds[t].l, k);
    nds[t].l = r;
    pull(t);
    return {l, t};
  } else {
    auto [l, r] = split(nds[t].r, k - szl - 1);
    nds[t].r = l;
    pull(t);
    return {t, r};
  }
}

// всё остальное ровно как в обычном ДД
// не забыть обновлять sz в pull 
// инициализация sz=0 в Node() и sz=1 в Node(...)


\end{lstlisting}

\subsection{ДД}
\begin{lstlisting}[language=C++]
// insert $(key,val)$, erase $key$, $\max(val)$ for $key\in[l,r)$, $val$+= for $key\in[l,r)$
struct Node {
  int l, r;
  int x, y;
  int val, mx, mod;

  // value of empty set
  Node() : val(-INF), mx(-INF) {
    l = r = 0, mod = 0;
  }
  Node(int x, int val) : x(x), val(val), mx(val) {
    l = r = 0, mod = 0, y = rnd();
  }
};

Node nds[MAX];
int ndsz = 1; // nds[0] means empty

void push(int t) {
  if (!t || nds[t].mod == 0) return;
  nds[t].val += nds[t].mod;
  nds[t].mx += nds[t].mod;
  if (nds[t].l) nds[nds[t].l].mod += nds[t].mod;
  if (nds[t].r) nds[nds[t].r].mod += nds[t].mod;
  nds[t].mod = 0;
}

int getmx(int t) {
  push(t); // delete if sure (faster)
  return nds[t].mx;
}

void pull(int t) {
  if (!t) return;
  push(t), push(nds[t].l), push(nds[t].r); // must have
  nds[t].mx = max(nds[t].val, max(getmx(nds[t].l), getmx(nds[t].r)));
}

pair<int, int> split(int t, int x) {
  if (!t) return {0, 0};
  push(t);
  if (x <= nds[t].x) {
    auto [l, r] = split(nds[t].l, x);
    nds[t].l = r;
    pull(t);
    return {l, t};
  } else {
    auto [l, r] = split(nds[t].r, x);
    nds[t].r = l;
    pull(t);
    return {t, r};
  }
}

int merge(int l, int r) {
  push(l), push(r);
  if (!l) return r;
  if (!r) return l;
  if (nds[l].y < nds[r].y) {
    nds[l].r = merge(nds[l].r, r);
    pull(l);
    return l;
  } else {
    nds[r].l = merge(l, nds[r].l);
    pull(r);
    return r;
  }
}

void insert(int &root, int x, int val) {
  nds[ndsz++] = Node(x, val);
  auto [l, r] = split(root, x);
  root = merge(merge(l, ndsz - 1), r);
}

// erase all equal to x
void erase(int &root, int x) {
  auto [lm, r] = split(root, x + 1);
  auto [l, m] = split(lm, x);
  root = merge(l, r);
}

// query [l, r)
int query(int &root, int ql, int qr) {
  auto [lm, r] = split(root, qr);
  auto [l, m] = split(lm, ql);
  int res = getmx(m);
  root = merge(merge(l, m), r);
  return res;
}

// update [l, r)
void update(int &root, int ql, int qr, int qx) {
  auto [lm, r] = split(root, qr);
  auto [l, m] = split(lm, ql);
  if (m) nds[m].mod += qx;
  root = merge(merge(l, m), r);
}

\end{lstlisting}

\subsection{Персистентное ДД по неявному}
\begin{lstlisting}[language=C++]
struct Node;
int size(int);
int sum(int);

struct Node {
  int l, r;
  int val, sz, sm;
    
  Node() : val(0), sz(0), sm(0) {}
  Node(int val, int l, int r) : val(val), l(l), r(r) {
    sz = 1 + size(l) + size(r);
    sm = val + sum(l) + sum(r);
  }
};

Node nds[MAX];
int ndsz = 1;

int size(int t) { return nds[t].sz; }
int sum(int t) { return nds[t].sm; }

int newNode(int val, int l, int r) {
  nds[ndsz++] = newNode(val, l, r);
  return ndsz - 1;
}

pair<int, int> split(int t, int x) {
  if (!t) return {0, 0};
  int szl = size(nds[t].l);
  if (szl >= x) {
    auto [l, r] = split(nds[t].l, x);
    int v = newNode(nds[t].val, r, nds[t].r);
    return {l, v};
  } else {
    auto [l, r] = split(nds[t].r, x - szl - 1);
    int v = newNode(nds[t].val, nds[t].l, l);
    return {v, r};
  }
}

bool chooseleft(int szl, int szr) {
  return rnd() % (szl + szr) < szl;
}

int merge(int l, int r) {
  if (!l) return r;
  if (!r) return l;
  if (chooseleft(nds[l].sz, nds[r].sz)) {
    int rr = merge(nds[l].r, r);
    int v = newNode(nds[l].val, nds[l].l, rr);
    return v;
  } else {
    int ll = merge(l, nds[r].l);
    int v = newNode(nds[r].val, ll, nds[r].r);
    return v;
  }
}

int insert(int root, int ponds[t].s, int val) {
  int new_v = newNode(val, 0, 0);
  auto [l, r] = split(root, pos);
  return merge(merge(l, new_v), r);
}

int erase(int root, int pos) {
  auto [lm, r] = split(root, pos + 1);
  auto [l, m] = split(lm, pos);
  return merge(l, r);
}

// query [l, r)
pair<int, int> query(int root, int ql, int qr) {
  auto [lm, r] = split(root, qr);
  auto [l, m] = split(lm, ql);
  int res = sum(m);
  auto new_root = merge(merge(l, m), r);
  return {res, new_root};
}

\end{lstlisting}

\subsection{Персистентное ДО}
\begin{lstlisting}[language=C++]
Node nds[MAX];
int ndsz = 1;
// nds[0] is default (empty) value

int sum(int v) { return nds[v].sm; }

// returns new root of subtree
int update(int v, int l, int r, int qi, int qx) {
  if (qi < l || r <= qi) return v;
  if (l + 1 == r) {
    nds[ndsz++] = Node(qx);
    return ndsz - 1;
  }
  int m = (l + r) / 2;
  int u = ndsz++;
  nds[u].l = update(nds[v].l, l, m, qi, qx);
  nds[u].r = update(nds[v].r, m, r, qi, qx);
  nds[u].sm = sum(nds[u].l) + sum(nds[u].r);
  return u;
}

int get(int v, int l, int r, int ql, int qr) {
  if (!v || qr <= l || r <= ql) return 0;
  if (ql <= l && r <= qr) return nds[v].sm;
  int m = (l + r) / 2;
  auto a = get(nds[v].l, l, m, ql, qr);
  auto b = get(nds[v].r, m, r, ql, qr);
  return a + b;
}

\end{lstlisting}

\subsection{Спарсы}
\begin{lstlisting}[language=C++]
int tree[LOG][MAXN];
int floorlog2[MAXN]; // i ? (31 - __builtin_clz(i)) : 0

void build(vector<int> &a) {
  int n = a.size();
  copy(a.begin(), a.end(), tree[0]);
  for (int i = 1; i < LOG; ++i) {
    int len = 1 << (i - 1);
    for (int j = 0; j + len < n; ++j)
      tree[i][j] = min(tree[i - 1][j], tree[i - 1][j + len]);
  }
  for (int i = 2; i <= n; ++i)
    floorlog2[i] = floorlog2[i / 2] + 1;
}

// min a[l..r)
int get(int l, int r) {
  int i = floorlog2[r - l];
  return min(tree[i][l], tree[i][r - (1 << i)]);
}

\end{lstlisting}

\subsection{Фенвик (+ на отрезке)}
\begin{lstlisting}[language=C++]
// a[l..r) += x
void update(int l, int r, int x) {
  T1.add(l, x);
  T1.add(r, -x);
  T2.add(l, -x * l);
  T2.add(r, x * r);
}

// sum a[0..i)
int get(int i) {
  return T1.get(i) * i + T2.get(i);
}

\end{lstlisting}

\subsection{Фенвик}
\begin{lstlisting}[language=C++]
// Нумерация с 0

struct Fenwick {
  int n;
  vector<int> f;
 
  Fenwick(int n) : n(n) {
    f.resize(n + 1);
  }
 
  // a[i] += x
  void add(int i, int x) {
    for (++i; i <= n; i += i & -i)
      f[i] += x;
  }
 
  // sum a[0..i)
  int get(int i) {
    int ans = 0;
    for (; i > 0; i -= i & -i)
      ans += f[i];
    return ans;
  }
 
  // a[..] > 0; find max k: sum a[0..k) <= x
  int max_not_more(int x) {
    int cur = 0;
    for (int i = 20; i >= 0; --i) {
      int len = 1 << i;
      if (cur + len <= n && f[cur + len] <= x) {
        cur += len;
        x -= f[cur];
      }
    }
    return cur;
  }
};

// sum a[x1..x2)[y1..y2)[z1..x2)
int sum_3d(int x1, int x2, int y1, int y2, int z1, int z2) {
  int ans = get(x2, y2, z2);
  ans -= get(x1, y2, z2) + get(x2, y1, z2) + get(x2, y2, z1);
  ans += get(x1, y1, z2) + get(x1, y2, z1) + get(x2, y1, z1);
  ans -= get(x1, y1, z1);
  return ans;
}
\end{lstlisting}


\end{multicols*}
\end{document}


