\documentclass[10pt,a4paper,landscape,twosided]{extarticle}

\usepackage[T2A]{fontenc}
\usepackage[T1]{fontenc}
\usepackage[utf8]{inputenc}
\usepackage[english, russian]{babel}
\usepackage{listings}
\usepackage[usenames,dvipsnames]{color}
\usepackage{amsmath}
\usepackage{verbatim}
\usepackage{hyperref}
\usepackage{color}
\usepackage{geometry}
\usepackage{multicol}
\usepackage{graphicx}

\geometry{verbose,landscape,a4paper,tmargin=0.7cm,bmargin=0.7cm,lmargin=0.7cm,rmargin=0.7cm}

\usepackage{listings}
\usepackage{color}

\definecolor{dkgreen}{rgb}{0,0.6,0}
\definecolor{gray}{rgb}{0.5,0.5,0.5}
\definecolor{mauve}{rgb}{0.58,0,0.82}

%% Golang definition for listings
%% http://github.io/julienc91/lstlistings-golang
%%
\lstdefinelanguage{Golang}%
  {morekeywords=[1]{package,import,func,type,struct,return,defer,panic,%
     recover,select,var,const,iota,},%
   morekeywords=[2]{string,uint,uint8,uint16,uint32,uint64,int,int8,int16,%
     int32,int64,bool,float32,float64,complex64,complex128,byte,rune,uintptr,%
     error,interface},%
   morekeywords=[3]{map,slice,make,new,nil,len,cap,copy,close,true,false,%
     delete,append,real,imag,complex,chan,},%
   morekeywords=[4]{for,break,continue,range,go,goto,switch,case,fallthrough,if,%
     else,default,},%
   morekeywords=[5]{Println,Printf,Error,Print,},%
   sensitive=true,%
   morecomment=[l]{//},%
   morecomment=[s]{/*}{*/},%
   morestring=[b]',%
   morestring=[b]",%
   morestring=[s]{`}{`},%
   }


\lstset{frame=tb,
    language=C++,
    aboveskip=1mm,
    belowskip=1mm,
    showstringspaces=true,
    columns=flexible,
    keepspaces=true,
    basicstyle={\small\ttfamily},
    numbers=none,
    numberstyle=\tiny\color{gray},
    keywordstyle=\color{blue},
    commentstyle=\color{dkgreen},
    stringstyle=\color{mauve},
    escapebegin=\color{dkgreen},
    breaklines=true,
    breakatwhitespace=false,
    tabsize=2,
    inputencoding = utf8,  % Input encoding
    extendedchars = true,  % Extended ASCII
    mathescape    = true   % Mathematical expressions between $
    captionpos    = b,     % Caption position
    literate      =        % Support additional characters
        {á}{{\'a}}1  {é}{{\'e}}1  {í}{{\'i}}1 {ó}{{\'o}}1  {ú}{{\'u}}1
        {Á}{{\'A}}1  {É}{{\'E}}1  {Í}{{\'I}}1 {Ó}{{\'O}}1  {Ú}{{\'U}}1
        {à}{{\`a}}1  {è}{{\`e}}1  {ì}{{\`i}}1 {ò}{{\`o}}1  {ù}{{\`u}}1
        {À}{{\`A}}1  {È}{{\'E}}1  {Ì}{{\`I}}1 {Ò}{{\`O}}1  {Ù}{{\`U}}1
        {ä}{{\"a}}1  {ë}{{\"e}}1  {ï}{{\"i}}1 {ö}{{\"o}}1  {ü}{{\"u}}1
        {Ä}{{\"A}}1  {Ë}{{\"E}}1  {Ï}{{\"I}}1 {Ö}{{\"O}}1  {Ü}{{\"U}}1
        {â}{{\^a}}1  {ê}{{\^e}}1  {î}{{\^i}}1 {ô}{{\^o}}1  {û}{{\^u}}1
        {Â}{{\^A}}1  {Ê}{{\^E}}1  {Î}{{\^I}}1 {Ô}{{\^O}}1  {Û}{{\^U}}1
        {œ}{{\oe}}1  {Œ}{{\OE}}1  {æ}{{\ae}}1 {Æ}{{\AE}}1  {ß}{{\ss}}1
        {ç}{{\c c}}1 {Ç}{{\c C}}1 {ø}{{\o}}1  {å}{{\r a}}1 {Å}{{\r A}}1
        {ñ}{{\~n}}1  {Ñ}{{\~N}}1  {¿}{{?`}}1  {¡}{{!`}}1
        {а}{{\selectfont\char224}}1
        {б}{{\selectfont\char225}}1
        {в}{{\selectfont\char226}}1
        {г}{{\selectfont\char227}}1
        {д}{{\selectfont\char228}}1
        {е}{{\selectfont\char229}}1
        {ё}{{\"e}}1
        {ж}{{\selectfont\char230}}1
        {з}{{\selectfont\char231}}1
        {и}{{\selectfont\char232}}1
        {й}{{\selectfont\char233}}1
        {к}{{\selectfont\char234}}1
        {л}{{\selectfont\char235}}1
        {м}{{\selectfont\char236}}1
        {н}{{\selectfont\char237}}1
        {о}{{\selectfont\char238}}1
        {п}{{\selectfont\char239}}1
        {р}{{\selectfont\char240}}1
        {с}{{\selectfont\char241}}1
        {т}{{\selectfont\char242}}1
        {у}{{\selectfont\char243}}1
        {ф}{{\selectfont\char244}}1
        {х}{{\selectfont\char245}}1
        {ц}{{\selectfont\char246}}1
        {ч}{{\selectfont\char247}}1
        {ш}{{\selectfont\char248}}1
        {щ}{{\selectfont\char249}}1
        {ъ}{{\selectfont\char250}}1
        {ы}{{\selectfont\char251}}1
        {ь}{{\selectfont\char252}}1
        {э}{{\selectfont\char253}}1
        {ю}{{\selectfont\char254}}1
        {я}{{\selectfont\char255}}1
        {А}{{\selectfont\char192}}1
        {Б}{{\selectfont\char193}}1
        {В}{{\selectfont\char194}}1
        {Г}{{\selectfont\char195}}1
        {Д}{{\selectfont\char196}}1
        {Е}{{\selectfont\char197}}1
        {Ё}{{\"E}}1
        {Ж}{{\selectfont\char198}}1
        {З}{{\selectfont\char199}}1
        {И}{{\selectfont\char200}}1
        {Й}{{\selectfont\char201}}1
        {К}{{\selectfont\char202}}1
        {Л}{{\selectfont\char203}}1
        {М}{{\selectfont\char204}}1
        {Н}{{\selectfont\char205}}1
        {О}{{\selectfont\char206}}1
        {П}{{\selectfont\char207}}1
        {Р}{{\selectfont\char208}}1
        {С}{{\selectfont\char209}}1
        {Т}{{\selectfont\char210}}1
        {У}{{\selectfont\char211}}1
        {Ф}{{\selectfont\char212}}1
        {Х}{{\selectfont\char213}}1
        {Ц}{{\selectfont\char214}}1
        {Ч}{{\selectfont\char215}}1
        {Ш}{{\selectfont\char216}}1
        {Щ}{{\selectfont\char217}}1
        {Ъ}{{\selectfont\char218}}1
        {Ы}{{\selectfont\char219}}1
        {Ь}{{\selectfont\char220}}1
        {Э}{{\selectfont\char221}}1
        {Ю}{{\selectfont\char222}}1
        {Я}{{\selectfont\char223}}1
        {і}{{\selectfont\char105}}1
        {ї}{{\selectfont\char168}}1
        {є}{{\selectfont\char185}}1
        {ґ}{{\selectfont\char160}}1
        {І}{{\selectfont\char73}}1
        {Ї}{{\selectfont\char136}}1
        {Є}{{\selectfont\char153}}1
        {Ґ}{{\selectfont\char128}}1 
}

\newcommand{\dollar}{\mbox{\textdollar}}

\setlength{\columnsep}{0.1in}
\setlength{\columnseprule}{1px}

\usepackage{fancyhdr}
\pagestyle{fancyplain}
\fancyhf{}
\fancyhead[R]{\fontsize{12}{12}\selectfont\bf{\thepage}\hspace{-0.4cm}}
\fancyhead[L]{\hspace{-0.4cm}\fontsize{8}{8}\selectfont\bf{NRU HSE (Andrianov, Lepeshov, Shulyatev)}}
\renewcommand{\headrulewidth}{0pt}
\renewcommand\headrule{\vspace{-0.2cm}\hspace{-0.7cm}\rule{\paperwidth}{0.4pt}\vspace{-0.8cm}}

\begin{document}

\title{\bf{Muffix Sassif -- TRD}}
\author{Andrianov, Lepeshov, Shulyatev}
\date{}
\maketitle
\begin{multicols*}{3}
\begin{center}{\includegraphics[width=5cm]{/home/ilian/github/muffix-sassif-trd/notebook-generator/picture.jpg}\end{center}
\tableofcontents
\end{multicols*}
\pagebreak
\begin{multicols*}{3}
\lstloadlanguages{C++,Java}


\section{Геометрия}

\subsection{2D}
\begin{lstlisting}[language=Python]

class Pt:
    def dot(self, other):
        return self.x * other.x + self.y * other.y

    def cross(self, other):
        return self.x * other.y - self.y * other.x

    @staticmethod
    def get_straight(self, other):
        a = self.y - other.y
        b = other.x - self.x
        c = self.cross(other)
        return a, b, c


class Straight:

    def __eq__(self, other):
        if self.b != 0 or other.b != 0:
            return self.a * other.b == other.a * self.b and self.c * other.b == other.c * self.b
        val1 = math.sqrt(self.a ** 2 + self.b ** 2)
        val2 = math.sqrt(other.a ** 2 + other.b ** 2)
        a1, c1 = self.a / val1, self.c / val1
        a2, c2 = other.a / val2, other.c / val2
        if (a1 < 0) != (a2 < 0):
            a1, a2, c1, c2 = a1, -a2, c1, -c2
        return a1 == a2 and c1 == c2

    def perpendicular(self, point: Pt):
        return Straight(-self.b, self.a, self.b * point.x - self.a * point.y)

    def get_value(self, point):
        return self.a * point.x + self.b * point.y + self.c

    def intersection(self, other):
        d = Pt(self.a, self.b).cross(Pt(other.a, other.b))
        dx = Pt(self.c, self.b).cross(Pt(other.c, other.b))
        dy = Pt(self.a, self.c).cross(Pt(other.a, other.c))
        return Pt(-dx / d, -dy / d)

    def dist_from_point(self, point):
        val = math.sqrt(self.a ** 2 + self.b ** 2)
        return abs(Straight(self.a / val, self.b / val, self.c / val).get_value(point))

    def parallel(self, dist):
        val = math.sqrt(self.a ** 2 + self.b ** 2)
        return Straight(self.a, self.b, self.c - dist * val)

    def is_parallel(self, other):
        return self.a * other.b == self.b * other.a

    def is_perpendicular(self, other):
        per = Straight(-self.b, self.a, 0)
        return per.a * other.b == per.b * other.a


class Triangle:
    def intersection_medians(self):
        return (self.A + self.B + self.C) / 3

    def intersection_altitudes(self):
        st1 = Straight(self.A, self.B).perpendicular(self.C)
        st2 = Straight(self.A, self.C).perpendicular(self.B)
        return st1.intersection(st2)

    def intersection_middle_pers(self):
        st1 = Straight(self.A, self.B).perpendicular((self.A + self.B) / 2)
        st2 = Straight(self.A, self.C).perpendicular((self.A + self.C) / 2)
        return st1.intersection(st2)


class Circle:
    def intersect_straight(self, st):
        pt = st.get_point()
        A = st.a ** 2 + st.b ** 2
        B = 2 * (-st.b * (pt.x - self.center.x) + st.a * (pt.y - self.center.y))
        C = (pt.x - self.center.x) ** 2 + (pt.y - self.center.y) ** 2 - self.radius ** 2
        D = B ** 2 - 4 * A * C
        if D < 0:
            return []
        D = math.sqrt(D)
        vector = Pt(-st.b, st.a)
        if D == 0:
            t = -B / (2 * A)
            return [pt + t * vector]
        t1 = (-B - D) / (2 * A)
        t2 = (-B + D) / (2 * A)
        return [pt + t1 * vector, pt + t2 * vector]

    def intersect_circle(self, other):
        x1, x2 = self.center.x, other.center.x
        y1, y2 = self.center.y, other.center.y
        a = -2 * (x1 - x2)
        b = -2 * (y1 - y2)
        c = (x1 ** 2 - x2 ** 2) + (y1 ** 2 - y2 ** 2) - (self.radius ** 2 - other.radius ** 2)
        return self.intersect_straight(Straight(a, b, c))

    def is_own(self, pt):
        return (pt.x - self.center.x) ** 2 + (pt.y - self.center.y) ** 2 == self.radius ** 2

    def tangent_pts(self, pt):
        if self.is_own(pt):
            return [pt]
        cir = Circle(pt, math.sqrt(abs(pt - self.center) ** 2 - self.radius ** 2))
        return self.intersect_circle(cir)

    def dist_by_circle(self, pt1, pt2):
        pt1 -= self.center
        pt2 -= self.center
        ang = (pt1.polar_angle() - pt2.polar_angle()) % (2 * math.pi)
        ang = min(ang, 2 * math.pi - ang)
        return self.radius * ang


\end{lstlisting}

\subsection{3D}
\begin{lstlisting}[language=C++]
// TODO

\end{lstlisting}

\subsection{Выпуклая оболочка}
\begin{lstlisting}[language=C++]
Pt start(0, 0);

bool comp(Pt a, Pt b) {
    ll ang = (a - start).cross(b - start);
    if (ang < 0) {
        return false;
    } else if (ang > 0) {
        return true;
    }
    return abs(start - a) < abs(start - b);
}

vector<Pt> convex_hull(vector<Pt> points) {
    int n = points.size();
    start = points[0];
    for (auto x : points) {
        start = min(start, x);
    }
    sort(points.begin(), points.end(), comp);
    vector<Pt> s = {points[0], points[1]};
    for (int i = 2; i < n; i++) {
        int k = s.size();
        while (k > 1 && (s[k - 1] - s[k - 2]).cross(points[i] - s[k - 1]) <= 0) {
            s.pop_back();
            k = s.size();
        }
        s.push_back(points[i]);
    }
    return s;
}


\end{lstlisting}

\subsection{Касательные из точки}
\begin{lstlisting}[language=C++]
// TODO

\end{lstlisting}

\subsection{Касательные параллельные прямой}
\begin{lstlisting}[language=C++]
// TODO

\end{lstlisting}

\section{Графы}

\subsection{Венгерский алгоритм}
\begin{lstlisting}[language=C++]
vector<int> venger(vector<vector<int>> arr) {
    int n = (int) arr.size() - 1;
    vector<int> u(n + 1), v(n + 1), p(n + 1), way(n + 1);
    for (int i = 1; i <= n; i++) {
        p[0] = i;
        int ind = 0;
        vector<int> minv(n + 1, INF), used(n + 1);
        do {
            used[ind] = 1;
            int ind2 = p[ind], dlt = INF, ind3 = 0;
            for (int j = 1; j <= n; j++)
                if (!used[j]) {
                    int cur = arr[ind2][j] - u[ind2] - v[j];
                    if (cur < minv[j]) {
                        minv[j] = cur;
                        way[j] = ind;
                    }
                    if (minv[j] < dlt) {
                        dlt = minv[j], ind3 = j;
                    }
                }
            for (int j = 0; j <= n; j++)
                if (used[j]) {
                    u[p[j]] += dlt;
                    v[j] -= dlt;
                } else {
                    minv[j] -= dlt;
                }
            ind = ind3;
        } while (p[ind] != 0);
        do {
            int ind3 = way[ind];
            p[ind] = p[ind3];
            ind = ind3;
        } while (ind);
    }
    vector<int> ans(n + 1);
    for (int j = 1; j <= n; j++) {
        ans[p[j]] = j;
    }
    return ans;
}

\end{lstlisting}

\subsection{Дейкстра за квадрат}
\begin{lstlisting}[language=C++]
// TODO

\end{lstlisting}

\subsection{Диниц}
\begin{lstlisting}[language=C++]
vector<edge> g[MAXN];
pair<int, int> pred[MAXN];
int d[MAXN];
int inds[MAXN];

bool dfs(int v, int final, int W) {
    if (v == final) {
        return true;
    }
    for (int i = inds[v]; i < (int) g[v].size(); i++) {
        auto e = g[v][i];
        if (e.f + W <= e.c && d[v] + 1 == d[e.v]) {
            pred[e.v] = {v, i};
            bool flag = dfs(e.v, final, W);
            if (flag) {
                return true;
            }
            inds[v]++;
        } else {
            inds[v]++;
        }
    }
    return false;
}

bool bfs(int start, int final, int W) {
    fill(d, d + MAXN, INF);
    d[start] = 0;
    deque<int> q = {start};
    while (!q.empty()) {
        int v = q.front();
        q.pop_front();
        for (auto e : g[v]) {
            if (e.f + W <= e.c && d[e.v] > d[v] + 1) {
                d[e.v] = d[v] + 1;
                q.push_back(e.v);
            }
        }
    }
    if (d[final] == INF) {
        return false;
    }
    fill(inds, inds + MAXN, 0);
    while (dfs(start, final, W)) {
        int v = final;
        int x = INF;
        while (v != start) {
            int ind = pred[v].second;
            v = pred[v].first;
            x = min(x, g[v][ind].c - g[v][ind].f);
        }
        v = final;
        while (v != start) {
            int ind = pred[v].second;
            v = pred[v].first;
            g[v][ind].f += x;
            g[g[v][ind].v][g[v][ind].ind].f -= x;
        }
    }
    return true;
}

void Dinic(int start, int final) {
    int W = (1LL << 30);
    do {
        while (bfs(start, final, W));
        W /= 2;
    } while (W >= 1);
}

signed main() {
    vector<pair<int, int>> edges;
    for (int i = 0; i < m; i++) {
        int u, v, c;
        cin >> u >> v >> c;
        edges.emplace_back(u, v);
        g[u].push_back({v, 0, c, (int) g[v].size()});
        g[v].push_back({u, 0, c, (int) g[u].size() - 1});
    }
    Dinic(1, n);
    int res = 0;
    for (auto e : g[1]) {
        res += e.f;
    }
    vector<int> ans;
    for (int i = 0; i < m; i++) {
        int u = edges[i].first, v = edges[i].second;
        if ((d[u] != INF && d[v] == INF) || (d[u] == INF && d[v] != INF)) {
            ans.push_back(i + 1);
        }
    }
}

\end{lstlisting}

\subsection{КСС}
\begin{lstlisting}[language=C++]
void dfs1(int v) {
    vis[v] = 1;
    for (auto u : g[v]) {
        if (!vis[u]) {
            dfs1(u);
        }
    }
    topsort.push_back(v);
}

void dfs2(int v) {
    vis[v] = 1;
    for (auto u : rg[v]) {
        if (!vis[u]) {
            dfs2(u);
        }
    }
    comp.push_back(v);
}

signed main() {
    for (int i = 1; i <= n; i++)
        if (!vis[i])
            dfs1(i);
    reverse(topsort.begin(), topsort.end());
    for (int j = 1; j <= n; j++) {
        int vert = topsort[j - 1];
        if (!vis[vert])
            dfs2(vert);
    }
}


\end{lstlisting}

\subsection{Минкост (Джонсон)}
\begin{lstlisting}[language=C++]

using cost_t = ll;
using flow_t = int;

const int MAXN = 10000;
const int MAXM = 25000 * 2;
const cost_t INFw = 1e12;
const flow_t INFf = 10;

struct Edge {
  int v, u;
  flow_t f, c;
  cost_t w;
};

Edge edg[MAXM];
int esz = 0;
vector<int> graph[MAXN];
ll dist[MAXN];
ll pot[MAXN];
int S, T;
int NUMV;
int pre[MAXN];
bitset<MAXN> inQ;

flow_t get_flow() {
  int v = T;
  if (pre[v] == -1)
    return 0;
  flow_t f = INFf;
  do {
    int ei = pre[v];
    Edge &e = edg[ei];
    f = min(f, e.c - e.f);
    if (f == 0)
      return 0;
    v = e.v;
  } while (v != S);
  v = T;
  do {
    int ei = pre[v];
    edg[ei].f += f;
    edg[ei ^ 1].f -= f;
    v = edg[ei].v;
  } while (v != S);
  return f;
}

void spfa() {
  fill(dist, dist + NUMV, INFw);
  dist[S] = 0;
  deque<int> Q = {S};
  inQ[S] = true;
  while (!Q.empty()) {
    int v = Q.front();
    Q.pop_front();
    inQ[v] = false;
    cost_t d = dist[v];
    for (int ei : graph[v]) {
      Edge &e = edg[ei];
      if (e.f == e.c)
        continue;
      cost_t w = e.w + pot[v] - pot[e.u];
      if (dist[e.u] <= d + w)
        continue;
      pre[e.u] = ei;
      dist[e.u] = d + w;
      if (!inQ[e.u]) {
        inQ[e.u] = true;
        Q.push_back(e.u);
      }
    }
  }
  for (int i = 0; i < NUMV; ++i)
    pot[i] += dist[i];
}

cost_t mincost() {
  spfa(); // pot[i] = 0 // or ford_bellman
  flow_t f = 0;
  while (true) {
    flow_t ff = get_flow();
    if (ff == 0)
      break;
    f += ff;
    spfa(); // or dijkstra
  }
  cost_t res = 0;
  for (int i = 0; i < esz; ++i)
    res += edg[i].f * edg[i].w;
  res /= 2;
  return res;
}

void add_edge(int v, int u, int c, int w) {
  edg[esz] = {v, u, 0, c, w};
  edg[esz + 1] = {u, v, 0, 0, -w};
  graph[v].push_back(esz);
  graph[u].push_back(esz + 1);
  esz += 2;
}

signed main() {
  ios_base::sync_with_stdio(false);
  cin.tie(nullptr);
  int n, m;
  cin >> n >> m;
  S = 0;
  T = n - 1;
  NUMV = n;
  for (int i = 0; i < m; ++i) {
    int v, u, c, w;
    cin >> v >> u >> c >> w;
    v--, u--;
    add_edge(v, u, c, w);
  }
  cost_t ans = mincost();
  cout << ans;
}


\end{lstlisting}

\subsection{Мосты}
\begin{lstlisting}[language=C++]
void dfs(int v, int par) {
    vis[v] = 1;
    up[v] = tin[v] = timer++;
    for (auto u : g[v]) {
        if (!vis[u]) {
            dfs(u, v);
            up[v] = min(up[v], up[u]);
        } else if (u != par) {
            up[v] = min(up[v], tin[u]);
        }
        if (up[u] > tin[v]) {
            bridges.emplace_back(v, u);
        }
    }
}

\end{lstlisting}

\subsection{Паросочетания}
\begin{lstlisting}[language=C++]
bool dfs(int v, int c) {
    if (used[v] == c) return false;
    used[v] = c;
    for (auto u : g[v]) {
        if (res[u] == -1) {
            res[u] = v;
            return true;
        }
    }
    for (auto u : g[v]) {
        if (dfs(res[u], c)) {
            res[u] = v;
            return true;
        }
    }
    return false;
}

signed main() {
    for (int i = 0; i < s; ++i) {
        ans += dfs(i, i + 1);
    }
}

\end{lstlisting}

\subsection{Точки сочленения}
\begin{lstlisting}[language=C++]
void dfs(int v, int par) {
    vis[v] = 1;
    up[v] = tin[v] = timer++;
    int child = 0;
    for (auto u : g[v]) {
        if (!vis[u]) {
            dfs(u, v);
            up[v] = min(up[v], up[u]);
            if (up[u] >= tin[v] && par != -1) {
                points.insert(v);
            }
            child++;
        } else if (u != par) {
            up[v] = min(up[v], tin[u]);
        }
    }
    if (par == -1 && child >= 2) {
        points.insert(v);
    }
}

\end{lstlisting}

\subsection{Эдмондс-Карп}
\begin{lstlisting}[language=C++]
struct edge {
    int v, f, c, ind;
};

vector<edge> g[MAXN];

bool bfs(int start, int final, int W) {
    vector<int> d(MAXN, INF);
    vector<pair<int, int>> pred(MAXN);
    d[start] = 0;
    deque<int> q = {start};
    while (!q.empty()) {
        int v = q.front();
        q.pop_front();
        for (int i = 0; i < (int) g[v].size(); i++) {
            auto e = g[v][i];
            if (e.f + W <= e.c && d[e.v] > d[v] + 1) {
                d[e.v] = d[v] + 1;
                pred[e.v] = {v, i};
                q.push_back(e.v);
            }
        }
    }
    if (d[final] == INF) {
        return false;
    }
    int v = final;
    int x = INF;
    while (v != start) {
        int ind = pred[v].second;
        v = pred[v].first;
        x = min(x, g[v][ind].c - g[v][ind].f);
    }
    v = final;
    while (v != start) {
        int ind = pred[v].second;
        v = pred[v].first;
        g[v][ind].f += x;
        g[g[v][ind].v][g[v][ind].ind].f -= x;
    }
    return true;
}

signed main() {
    for (int i = 0; i < m; i++) {
        int u, v, c;
        cin >> u >> v >> c;
        g[u].push_back({v, 0, c, (int) g[v].size()});
        g[v].push_back({u, 0, 0, (int) g[u].size() - 1});
    }
    int W = (1 << 30);
    do {
        while (bfs(1, n, W));
        W /= 2;
    } while (W >= 1);
    int res = 0;
    for (auto e : g[1]) {
        res += e.f;
    }
}

\end{lstlisting}

\subsection{Эйлеров цикл}
\begin{lstlisting}[language=C++]
// TODO

\end{lstlisting}

\section{ДП}

\subsection{CHT}
\begin{lstlisting}[language=C++]
pair<ld, ld> inter(Line a, Line b) {
    ld x = (b.b - a.b) / (a.k - b.k);
    ld y = a.k * x + a.b;
    return {x, y};
}
 
void add_line(ld k, ld b, vector<Line> &s, vector<pair<ld, ld>> &pts) {
    while (s.size() >= 2) {
        pair<ld, ld> x1 = inter(s.back(), s[s.size() - 2]);
        pair<ld, ld> x2 = inter(s[s.size() - 2], {k, b});
        if (x1 > x2) {
            break;
        }
        pts.pop_back();
        s.pop_back();
    }
    if (!s.empty()) {
        pts.push_back(inter(s.back(), {k, b}));
    }
    s.push_back({k, b});
}
 
ld bin_search(vector<Line> &s, ld x) {
    int l = 0, r = s.size();
    while (l + 1 < r) {
        int m = (r + l) / 2;
        auto kek = inter(s[m - 1], s[m]);
        if (kek.first >= x) {
            l = m;
        } else {
            r = m;
        }
    }
    return s[l].k * x + s[l].b;
}

\end{lstlisting}

\subsection{Li Chao}
\begin{lstlisting}[language=C++]
// max

struct Line {
    int k, b;

    int f(int x) {
        return k * x + b;
    }
};

struct ST {
    vector<Line> st;

    ST(int n) {
        Line ln = {0LL, -INF};
        st.resize(4 * n, ln);
    }

    void upd(int i, int l, int r, Line ln) {
        int child = 1;
        Line ln1 = ln;
        int m = (l + r) / 2;
        if (ln.f(m) > st[i].f(m)) {
            if (ln.k < st[i].k) {
                child = 2;
            }
            ln1 = st[i];
            st[i] = ln;
        } else {
            if (st[i].k < ln.k) {
                child = 2;
            }
        }
        if (l + 1 < r) {
            if (child == 1) {
                upd(i * 2 + 1, l, m, ln1);
            } else {
                upd(i * 2 + 2, m, r, ln1);
            }
        }
    }

    int res(int i, int l, int r, int x) {
        if (l + 1 == r) {
            return st[i].f(x);
        }
        int m = (l + r) / 2;
        int val = st[i].f(x);
        if (x < m) {
            val = max(val, res(i * 2 + 1, l, m, x));
        } else {
            val = max(val, res(i * 2 + 2, m, r, x));
        }
        return val;
    }
};

\end{lstlisting}

\subsection{SOS-dp}
\begin{lstlisting}[language=C++]
// TODO

\end{lstlisting}

\subsection{НВП}
\begin{lstlisting}[language=C++]
// TODO

\end{lstlisting}

\subsection{НОВП}
\begin{lstlisting}[language=C++]
// TODO

\end{lstlisting}

\section{Деревья}

\subsection{Centroid}
\begin{lstlisting}[language=C++]
void sizes(int v, int p) {
    sz[v] = 1;
    for (auto u : g[v]) {
        if (u != p && !used[u]) {
            sizes(u, v);
            sz[v] += sz[u];
        }
    }
}
int centroid(int v, int p, int n) {
    for (int u : g[v]) {
        if (sz[u] > n / 2 && u != p && !used[u]) {
            return centroid(u, v, n);
        }
    }
    return v;
}

void dfs(int v, int p) {
    ........
    for (auto u : g[v]) {
        if (u != p && !used[u]) {
            dfs(u, v);
        }
    }
}

void solve(int v) {
    sizes(v, -1);
    .........
    for (auto u : g[v]) {
        if (!used[u]) {
            ...........
            dfs(u, v);   
            ..................     
        }
    }
    .....................
    used[v] = 1;
    for (int u : g[v]) {
        if (!used[u]) {
            solve(centroid(u, v, sz[u]));
        }
    }
}
int main() {
    sizes(0, -1);
    solve(centroid(0, -1, n));
}


\end{lstlisting}

\subsection{HLD}
\begin{lstlisting}[language=C++]
const int MAXN = 50500;
const int INF = (int) 1e15;
const int L = 20;
vector<int> g[MAXN];
int sz[MAXN];
int depth[MAXN];

vector<vector<int>> up(MAXN, vector<int>(L + 1));

void dfs(int v, int p) {
    up[v][0] = p;
    for (int i = 1; i <= L; i++) {
        up[v][i] = up[up[v][i - 1]][i - 1];
    }
    for (int u : g[v]) {
        if (u != p) {
            dfs(u, v);
        }
    }
}

int lca(int u, int v) {
    if (u == v) {
        return u;
    }
    int du = depth[u], dv = depth[v];
    if (du < dv) {
        swap(du, dv);
        swap(u, v);
    }
    for (int i = L; i >= 0; i--) {
        if (du - (int) pow(2, i) >= dv) {
            u = up[u][i];
            du -= (int) pow(2, i);
        }
    }
    if (u == v) {
        return u;
    }
    for (int i = L; i >= 0; i--) {
        if (up[u][i] != up[v][i]) {
            u = up[u][i];
            v = up[v][i];
        }
    }
    return up[u][0];
}

void dfs1(int v, int p) {
    sz[v] = 1;
    for (int u : g[v]) {
        if (u != p) {
            dfs1(u, v);
            sz[v] += sz[u];
        }
    }
}

int cnt = 0;
int nn[MAXN];
int pred[MAXN];
int rup[MAXN];

void dfs2(int v, int p, int root, int dep = 0) {
    depth[v] = dep;
    nn[v] = cnt++;
    pred[v] = p;
    rup[v] = root;
    int mx = 0;
    int vert = -1;
    for (int u : g[v]) {
        if (u != p) {
            if (mx < sz[u]) {
                mx = sz[u];
                vert = u;
            }
        }
    }
    if (vert != -1) {
        dfs2(vert, v, root, dep + 1);
    }
    for (int u : g[v]) {
        if (u != p && u != vert) {
            dfs2(u, v, u, dep + 1);
        }
    }
}

ST st({});
int n;

int mx_path_up(int u, int v) {
    if (depth[u] < depth[v]) {
        swap(u, v);
    }
    int res = -INF;
    while (true) {
        int root = rup[u];
        if (depth[root] <= depth[v]) {
            res = max(res, st.rmq(0, 0, n, nn[v], nn[u] + 1));
            break;
        }
        res = max(res, st.rmq(0, 0, n, nn[root], nn[u] + 1));
        u = pred[rup[u]];
    }
    return res;
}

int mx_path(int u, int v) {
    int vert = lca(u, v);
    return max(mx_path_up(u, vert), mx_path_up(v, vert));
}

void change(int u, int qd) {
    st.update(0, 0, n, nn[u], qd);
}

signed main() {
    cin >> n;
    vector<int> hs(n);
    for (auto &x : hs) {
        cin >> x;
    }
    for (int i = 0; i < n - 1; i++) {
        cin >> u1 >> v1;
        g[u1].push_back(v1);
        g[v1].push_back(u1);
    }
    dfs1(1, -1);
    dfs(1, 1);
    dfs2(1, -1, 1);
    vector<int> nhs(n);
    for (int i = 1; i <= n; i++) {
        nhs[nn[i]] = hs[i - 1];
    }
    st = *new ST(nhs);
    char op;
    int q;
    cin >> q;
    while (q--) {
        cin >> op >> v1 >> u1;
        if (op == '?') {
            cout << mx_path(u1, v1) << endl;
        } else {
            change(v1, u1);
        }
    }
}

\end{lstlisting}

\subsection{Link-cut}
\begin{lstlisting}[language=C++]
struct Node {
  Node *ch[2];
  Node *p;
  bool rev;
  int sz;

  Node () {
    ch[0] = nullptr;
    ch[1] = nullptr;
    p = nullptr;
    rev = false;
    sz = 1;
  }
};

int size(Node *v) {
  return (v ? v->sz : 0);
}

int chnum(Node *v) {
  return v->p->ch[1] == v;
}

bool isroot(Node *v) {
  return v->p == nullptr || v->p->ch[chnum(v)] != v;
}

void push(Node *v) {
  if (v->rev) {
    if (v->ch[0])
      v->ch[0]->rev ^= 1;
    if (v->ch[1])
      v->ch[1]->rev ^= 1;
    swap(v->ch[0], v->ch[1]);
    v->rev = false;
  }
}

void pull(Node *v) {
  v->sz = size(v->ch[1]) + size(v->ch[0]) + 1;
}

void attach(Node *v, Node *p, int num) {
  if (p)
    p->ch[num] = v;
  if (v)
    v->p = p;
}

void rotate(Node *v) {
  Node *p = v->p;
  push(p);
  push(v);
  int num = chnum(v);
  Node *u = v->ch[1 - num];
  if (!isroot(v->p))
    attach(v, p->p, chnum(p));
  else
    v->p = p->p;
  attach(u, p, num);
  attach(p, v, 1 - num);
  pull(p);
  pull(v);
}

void splay(Node *v) {
  push(v);
  while (!isroot(v)) {
    if (!isroot(v->p)) {
      if (chnum(v) == chnum(v->p))
        rotate(v->p);
      else
        rotate(v);
    }
    rotate(v);
  }
}

void expose(Node *v) {
  splay(v);
  v->ch[1] = nullptr;
  pull(v);
  while (v->p != nullptr) {
    Node *p = v->p;
    splay(p);
    attach(v, p, 1);
    pull(p);
    splay(v);
  }
}

void makeroot(Node *v) {
  expose(v);
  v->rev ^= 1;
  push(v);
}

void link(Node *v, Node *u) {
  makeroot(v);
  makeroot(u);
  u->p = v;
}

void cut(Node *v, Node *u) {
  makeroot(u);
  makeroot(v);
  v->ch[1] = nullptr;
  u->p = nullptr;
}

int get(Node *v, Node *u) {
  makeroot(u);
  makeroot(v);
  Node *w = u;
  while (!isroot(w))
    w = w->p;
  return (w == v ? size(v) - 1 : -1);
}

const int MAXN = 100010;
Node *nodes[MAXN];

int main() {
  ios_base::sync_with_stdio(false);
  cin.tie(nullptr);
  cout.tie(nullptr);
  int n, q;
  cin >> n >> q;
  for (int i = 0; i < n; ++i)
    nodes[i] = new Node();
  while (q--) {
    string s;
    Int a, b;
    cin >> s >> a >> b;
    a--, b--;
    if (s[0] == 'g')
      cout << get(nodes[a], nodes[b]) << '\n';
    else if (s[0] == 'l')
      link(nodes[a], nodes[b]);
    else
      cut(nodes[a], nodes[b]);
  }
}

\end{lstlisting}

\section{Другое}

\subsection{Slope trick}
\begin{lstlisting}[language=C++]
// Дан массив $a_n$. Сделать минимальное кол-во $\pm 1$, чтобы $a_n$ стал неубывающим.

void solve() {
    int n;
    cin >> n;
    vector<int> a(n);
    for (int i = 0; i < n; i++) {
        cin >> a[i];
    }
    int ans = 0;
    multiset<int> now;
    for (int i = 0; i < n; i++) {
        now.insert(a[i]);
        ans += (*now.rbegin() - a[i]);
        now.erase(now.find(*now.rbegin()));
        now.insert(a[i]);
    }
    cout << ans << '\n';
}
\end{lstlisting}

\subsection{attribute\_packed}
\begin{lstlisting}[language=C++]
// TODO

\end{lstlisting}

\subsection{ordered\_set}
\begin{lstlisting}[language=C++]
#include <ext/pb_ds/assoc_container.hpp>
#include <ext/pb_ds/tree_policy.hpp>

using namespace __gnu_pbds;

typedef tree<int, null_type, less<>, rb_tree_tag, tree_order_statistics_node_update> ordered_set;

//st.find_by_order(index);
//st.order_of_key(key);

\end{lstlisting}

\subsection{pragma}
\begin{lstlisting}[language=C++]
#pragma GCC optimize("Ofast,fast-math,unroll-loops,no-stack-protector,inline")
#pragma GCC target("sse,sse2,sse3,ssse3,sse4,sse4.1,sse4.2,avx,avx2,abm,mmx,popcnt")

\end{lstlisting}

\subsection{Аллокатор Копелиовича}
\begin{lstlisting}[language=C++]
// TODO

\end{lstlisting}

\section{Математика}

\subsection{2-SAT}
\begin{lstlisting}[language=C++]
// TODO

\end{lstlisting}

\subsection{FFT mod}
\begin{lstlisting}[language=C++]
#include <bits/stdc++.h>

using namespace std;

typedef long long ll;
#define int long long

const int MOD = 998244353; // 7*17 * 2^23 + 1
const int GEN = 3;
//const int MOD = 7340033; // 7 * 2^20 + 1
//const int GEN = 5;
//const int MOD = 469762049; // 7 * 2^26 + 1
//const int GEN = 30;

const int LOG = 20;
const int MAXN = 1 << LOG;
int tail[MAXN + 1];
int OMEGA[MAXN + 1];

int binpow(int x, int p) {
    int res = 1;
    while (p > 0) {
        if (p & 1)
            res = res * 1ll * x % MOD;
        x = x * 1ll * x % MOD;
        p >>= 1;
    }
    return res;
}

int omega(int n, int k) {
    return OMEGA[MAXN / n * k];
}

int gettail(int x, int lg) {
    return tail[x] >> (LOG - lg);
}

void calcomega() {
    ll one = binpow(GEN, (MOD - 1) / MAXN);
    OMEGA[0] = 1;
    for (int i = 1; i < MAXN; ++i) {
        OMEGA[i] = OMEGA[i - 1] * one % MOD;
    }
}

void calctail() {
    int n = MAXN;
    for (int x = 0; x < n; ++x) {
        int res = 0;
        for (int i = 0; i < LOG; ++i) {
            res += ((x >> i) & 1) << (LOG - i - 1);
        }
        tail[x] = res;
    }
}

// Without precalc, tail[], OMEGA[]
//
//ll omega(int n, int k) {
//    return binpow(GEN, (MOD - 1) / n * k);
//}
//
//int gettail(int x, int lg) {
//    int res = 0;
//    for (int i = 0; i < lg; ++i)
//        res += ((x >> i) & 1) << (lg - i - 1);
//    return res;
//}

void fft(vector<int> &A, int lg) {
    int n = 1 << lg;
    for (int i = 0; i < n; ++i) {
        int j = gettail(i, lg);
        if (i < j)
            swap(A[i], A[j]);
    }
    for (int len = 2; len <= n; len *= 2) {
        for (int i = 0; i < n; i += len) {
            for (int j = 0; j < len / 2; ++j) {
                auto v = A[i + j];
                auto u = A[i + j + len / 2] * 1ll * omega(len, j) % MOD;
                A[i + j] = (v + u) % MOD;
                A[i + j + len / 2] = (v - u + MOD) % MOD;
            }
        }
    }
}

int inverse(int x) {
    return binpow(x, MOD - 2);
}

void invfft(vector<int> &A, int lg) {
    int n = 1 << lg;
    fft(A, lg);
    for (auto &el : A)
        el = el * 1ll * inverse(n % MOD) % MOD;
    reverse(A.begin() + 1, A.end());
}

vector<int> mul(vector<int> A, vector<int> B) {
    int lg = 32 - __builtin_clz(A.size() + B.size() - 1);
    int n = 1 << lg;
    A.resize(n, 0);
    B.resize(n, 0);
    fft(A, lg);
    fft(B, lg);
    for (int i = 0; i < n; ++i)
        A[i] = A[i] * 1ll * B[i] % MOD;
    invfft(A, lg);
    return A;
}

signed main() {
    ios_base::sync_with_stdio(false);
    cin.tie(nullptr);
    calctail();
    calcomega();
    int n, m;
    cin >> n >> m;
    vector<int> A(n), B(m);
    for (int &el : A)
        cin >> el;
    for (int &el : B)
        cin >> el;
    auto C = mul(A, B);
    for (auto el : C)
        cout << el << ' ';
}

\end{lstlisting}

\subsection{FFT}
\begin{lstlisting}[language=C++]
const double PI = acos(-1);
const int LOG = 19;
const int MAXN = 1 << LOG;

struct comp {
    double x, y;
    comp() : x(0), y(0) {}
    comp(double x, double y) : x(x), y(y) {}
    comp(int x) : x(x), y(0) {}
    comp operator+(const comp &o) const {
        return comp(x + o.x, y + o.y);
    }
    comp operator-(const comp &o) const {
        return comp(x - o.x, y - o.y);
    }
    comp operator*(const comp &o) const {
        return comp(x * o.x - y * o.y, x * o.y + y * o.x);
    }
    comp operator/(const int k) const {
        return comp(x / k, y / k);
    }
    comp conj() const {
        return comp(x, -y);
    }
};

comp OMEGA[MAXN + 10];
int tail[MAXN + 10];

comp omega(int n, int k) {
    return OMEGA[MAXN / n * k];
}

void calcomega() {
    for (int i = 0; i < MAXN; ++i) {
        double x = 2 * PI * i / MAXN;
        OMEGA[i] = {cos(x), sin(x)};
    }
}

void calctail() {
    tail[0] = 0;
    for (int i = 1; i < MAXN; ++i) {
        tail[i] = (tail[i >> 1] >> 1) | ((i & 1) << (LOG - 1));
    }
}

void fft(vector<comp> &A) {
    int n = A.size();
    for (int i = 0; i < n; ++i) {
        if (i < tail[i])
            swap(A[i], A[tail[i]]);
    }
    for (int len = 2; len <= n; len *= 2) {
        for (int i = 0; i < n; i += len) {
            for (int j = 0; j < len / 2; ++j) {
                auto v = A[i + j];
                auto u = A[i + j + len / 2] * omega(len, j);
                A[i + j] = v + u;
                A[i + j + len / 2] = v - u;
            }
        }
    }
}

void fft2(vector<comp> &A, vector<comp> &B) {
    int n = A.size();
    vector<comp> C(n);
    for (int i = 0; i < n; ++i) {
        C[i].x = A[i].x;
        C[i].y = B[i].x;
    }
    fft(C);
    C.push_back(C[0]);
    for (int i = 0; i < n; ++i) {
        A[i] = (C[i] + C[n - i].conj()) / 2;
        B[i] = (C[i] - C[n - i].conj()) / 2 * comp(0, -1);
    }
}
void invfft(vector<comp> &A) {
    fft(A);
    for (auto &el : A)
        el = el / MAXN;
    reverse(A.begin() + 1, A.end());
}

vector<int> mul(vector<int> &a, vector<int> &b) {
    vector<comp> A(MAXN, 0), B(MAXN, 0);
    for (int i = 0; i < (int)a.size(); ++i)
        A[i] = a[i];
    for (int i = 0; i < (int)b.size(); ++i)
        B[i] = b[i];
    fft2(A, B);
    for (int i = 0; i < MAXN; ++i)
        A[i] = A[i] * B[i];
    invfft(A);
    vector<int> c(MAXN);
    for (int i = 0; i < MAXN; ++i) {
        int x = round(A[i].x);
        c[i] = x;
    }
    while (!c.empty() && c.back() == 0)
        c.pop_back();
    return c;
}

signed main() {
    ios_base::sync_with_stdio(false);
    cin.tie(nullptr);
    calcomega();
    calctail();
    string sa, sb;
    cin >> sa >> sb;
    reverse(sa.begin(), sa.end());
    reverse(sb.begin(), sb.end());
    vector<int> a(sa.size()), b(sb.size());
    bool minus = false;
    if (sa.back() == '-') {
        minus ^= true;
        sa.pop_back();
    }
    if (sb.back() == '-') {
        minus ^= true;
        sb.pop_back();
    }
    for (int i = 0; i < (int)sa.size(); ++i)
        a[i] = sa[i] - '0';
    for (int i = 0; i < (int)sb.size(); ++i)
        b[i] = sb[i] - '0';
    auto c = mul(a, b);
    int shift = 0;
    for (int i = 0; i < (int)c.size(); ++i) {
        int x = c[i] + shift;
        c[i] = x % 10;
        shift = x / 10;
    }
    while (shift > 0) {
        c.push_back(shift % 10);
        shift /= 10;
    }
    while (!c.empty() && c.back() == 0)
        c.pop_back();
    if (c.empty()) {
        cout << 0;
        return 0;
    }
    if (minus)
        cout << '-';
    for (int i = c.size() - 1; i >= 0; --i)
        cout << c[i];
}

\end{lstlisting}

\subsection{Гаусс}
\begin{lstlisting}[language=C++]
// TODO битовый
// TODO дабловый
// TODO по модулю


\end{lstlisting}

\subsection{Диофантовы уравнения}
\begin{lstlisting}[language=Python]
def bezout(a, b):
    x, xx, y, yy = 1, 0, 0, 1
    while b:
        q = a // b
        a, b = b, a % b
        x, xx = xx, x - xx*q
        y, yy = yy, y - yy*q
    return (a, x, y)

a, b, c = map(int, input().split())
d, k, l = bezout(a, b)
q = c // d
x, y = q*k, q*l
if c % d == 0:
    x -= x // (b//d) * (b//d)
    y = (c-a*x) // b

\end{lstlisting}

\subsection{КТО}
\begin{lstlisting}[language=C++]
// TODO

\end{lstlisting}

\subsection{Код Грея}
\begin{lstlisting}[language=C++]
// TODO

\end{lstlisting}

\subsection{Линейное решето}
\begin{lstlisting}[language=C++]
// TODO

\end{lstlisting}

\subsection{Ро-Поллард}
\begin{lstlisting}[language=C++]
// TODO

\end{lstlisting}

\subsection{Символ Якоби, Лежандра}
\begin{lstlisting}[language=C++]
// TODO

\end{lstlisting}

\section{Строки}

\subsection{Z-функция}
\begin{lstlisting}[language=C++]
int main() {
    vector<int> z(n, 0);
    z[0] = n;
    int l = 0, r = 0;
    for (int i = 1; i < n; i++) {
        if (i < r) {
            z[i] = min(z[i - l], r - i);
        }
        while (i + z[i] < n && s[z[i]] == s[i + z[i]]) {
            z[i]++;
        }
        if (i + z[i] > r) {
            l = i;
            r = i + z[i];
        }
    }
}

\end{lstlisting}

\subsection{Ахо-Корасик}
\begin{lstlisting}[language=C++]
int cntv = 1;

void add(string &s) {
    static int cnt_s = 1;
    int v = 0;
    for (char el : s) {
        if (go[v][el - 'a'] == 0) {
            go[v][el - 'a'] = cntv;
            par[cntv] = v;
            par_c[cntv] = el;
            cntv++;
        }
        v = go[v][el - 'a'];
    }
    term[v].push_back(cnt_s++);
}

void bfs() {
    deque<int> q = {0};
    while (!q.empty()) {
        int v = q.front();
        q.pop_front();
        if (v > 0) {
            if (par[v] == 0) {
                suf[v] = 0;
            } else {
                suf[v] = go[suf[par[v]]][par_c[v] - 'a'];
            }
            g[suf[v]].push_back(v);
        }
        for (int c = 0; c < 26; c++) {
            if (go[v][c] == 0) {
                go[v][c] = go[suf[v]][c];
            } else {
                q.push_back(go[v][c]);
            }
        }
    }
}

\end{lstlisting}

\subsection{Префикс-функция}
\begin{lstlisting}[language=C++]
int main() {
    vector<int> pref(n, 0);
    int ans = 0;
    for (int i = 1; i < n; i++) {
        while (ans > 0 && s[ans] != s[i]) {
            ans = pref[ans - 1];
        }
        if (s[i] == s[ans]) {
            ans++;
        }
        pref[i] = ans;
    }
}

\end{lstlisting}

\subsection{Суффиксный автомат}
\begin{lstlisting}[language=C++]
struct Node {
    int go[26];
    int suf;
    int len;
};
Node verts[MAXN];
int cnt_v = 1;
int max_v = 0;

void add(char c) {
    c -= 'a';
    int nv = cnt_v++;
    verts[max_v].go[c] = nv;
    verts[nv].len = verts[max_v].len + 1;
    int v = max_v;
    while (v != 0) {
        if (verts[verts[v].suf].go[c] == 0) {
            v = verts[v].suf;
            verts[v].go[c] = nv;
            verts[nv].len = max(verts[nv].len, verts[v].len + 1);
            continue;
        }
        int vv = verts[v].suf, uu = verts[vv].go[c];
        if (verts[vv].len + 1 == verts[uu].len) {
            verts[nv].suf = uu;
            break;
        }
        int v2 = cnt_v++;
        for (int c2 = 0; c2 < 26; c2++) {
            verts[v2].go[c2] = verts[uu].go[c2];
        }
        int to = verts[vv].go[c];
        do {
            if (verts[vv].go[c] == to) {
                verts[vv].go[c] = v2;
                verts[v2].len = max(verts[v2].len, verts[vv].len + 1);
            } else {
                break;
            }
            vv = verts[vv].suf;
        } while (vv != 0);
        if (verts[vv].go[c] == to) {
            verts[vv].go[c] = v2;
            verts[v2].len = max(verts[v2].len, verts[vv].len + 1);
        }
        verts[v2].suf = verts[uu].suf;
        verts[uu].suf = verts[nv].suf = v2;
        break;
    }
    max_v = nv;
}


\end{lstlisting}

\subsection{Суффиксный массив}
\begin{lstlisting}[language=C++]
vector<int> build_suff_arr(string s) {
    s.push_back('#');
    int n = s.size();
    vector<int> suf(n), c(n);
    vector<int> cnt(MAX);
    for (int i = 0; i < n; i++) {
        cnt[s[i] - '#']++;
    }
    vector<int> pos(MAX);
    for (int i = 1; i < MAX; i++) {
        pos[i] = pos[i - 1] + cnt[i - 1];
    }
    for (int i = 0; i < n; i++) {
        suf[pos[s[i] - '#']++] = i;
    }
    int cls = -1;
    for (int i = 0; i < n; i++) {
        if (i == 0 || s[suf[i]] != s[suf[i - 1]]) {
            cls++;
        }
        c[suf[i]] = cls;
    }
    for (int L = 1; L < n; L *= 2) {
        fill(cnt.begin(), cnt.end(), 0);
        for (int i = 0; i < n; i++) {
            cnt[c[i]]++;
        }
        pos[0] = 0;
        for (int i = 1; i < n; i++) {
            pos[i] = pos[i - 1] + cnt[i - 1];
        }
        for (int i = 0; i < n; i++) {
            suf[i] = (suf[i] - L + n) % n;
        }
        vector<int> new_suf(n), new_c(n);
        for (int i = 0; i < n; i++) {
            int where = pos[c[suf[i]]];
            new_suf[where] = suf[i];
            pos[c[suf[i]]]++;
        }
        cls = -1;
        for (int i = 0; i < n; i++) {
            if (i == 0) {
                cls++;
                new_c[new_suf[i]] = cls;
                continue;
            }
            pair<int, int> prev = {c[new_suf[i - 1]], c[(new_suf[i - 1] + L) % n]};
            pair<int, int> now = {c[new_suf[i]], c[(new_suf[i] + L) % n]};
            if (prev != now) {
                cls++;
            }
            new_c[new_suf[i]] = cls;
        }
        swap(c, new_c);
        swap(suf, new_suf);
    }
    vector<int> res;
    for (int i = 1; i < n; i++) {
        res.push_back(suf[i]);
    }
    return res;
}

vector<int> lcp_neighboring(string &s, vector<int> &suf) {
    int n = s.size();
    vector<int> lcp(n), where(n);
    for (int i = 0; i < n; i++) {
        where[suf[i]] = i;
    }
    int k = 0;
    for (int j = 0; j < n; j++) {
        int pos = where[j];
        if (pos == n - 1) {
            k = 0;
            lcp[pos] = 0;
        } else {
            k = max(0LL, k - 1);
            while (s[j + k] == s[suf[pos + 1] + k]) {
                k++;
                if (j + k >= n || suf[pos + 1] + k >= n) {
                    break;
                }
            }
            lcp[pos] = k;
        }
    }
    return lcp;
}

int sol(int k, string s) {
    int n = s.size();
    vector<int> suf = build_suff_arr(s);
    vector<int> lcp = lcp_neighboring(s, suf);
    vector<int> where(n);
    for (int i = 0; i < n; i++) {
        where[suf[i]] = i;
    }
    Sparse_Table st(lcp);
    int ans = 0;
    for (int i = 0; i < n - k; i++) {
        ans += st.rmq(where[i], where[i + k]);
    }
    return ans;
}

\end{lstlisting}

\section{Структуры данных}

\subsection{Disjoint Sparse Table}
\begin{lstlisting}[language=C++]
// TODO
\end{lstlisting}

\subsection{Segment Tree Beats}
\begin{lstlisting}[language=C++]
// min=, sum

struct ST {
    vector<int> st, mx, mx_cnt, sec_mx;

    ST(int n) {
        st.resize(n * 4, 0);
        mx.resize(n * 4, 0);
        mx_cnt.resize(n * 4, 0);
        sec_mx.resize(n * 4, 0);
        build(0, 0, n);
    }

    void upd_from_children(int v) {
        st[v] = st[v * 2 + 1] + st[v * 2 + 2];
        mx[v] = max(mx[v * 2 + 1], mx[v * 2 + 2]);
        mx_cnt[v] = 0;
        sec_mx[v] = max(sec_mx[v * 2 + 1], sec_mx[v * 2 + 2]);
        if (mx[v * 2 + 1] == mx[v]) {
            mx_cnt[v] += mx_cnt[v * 2 + 1];
        } else {
            sec_mx[v] = max(sec_mx[v], mx[v * 2 + 1]);
        }
        if (mx[v * 2 + 2] == mx[v]) {
            mx_cnt[v] += mx_cnt[v * 2 + 2];
        } else {
            sec_mx[v] = max(sec_mx[v], mx[v * 2 + 2]);
        }
    }

    void build(int i, int l, int r) {
        if (l + 1 == r) {
            st[i] = mx[i] = 0;
            mx_cnt[i] = 1;
            sec_mx[i] = -INF;
            return;
        }
        int m = (r + l) / 2;
        build(i * 2 + 1, l, m);
        build(i * 2 + 2, m, r);
        upd_from_children(i);
    }

    void push_min_eq(int v, int val) {
        if (mx[v] > val) {
            st[v] -= (mx[v] - val) * mx_cnt[v];
            mx[v] = val;
        }
    }

    void push(int i) {
        push_min_eq(i * 2 + 1, mx[i]);
        push_min_eq(i * 2 + 2, mx[i]);
    }

    void update(int i, int l, int r, int ql, int qr, int val) {
        if (mx[i] <= val) {
            return;
        }
        if (ql == l && qr == r && sec_mx[i] < val) {
            push_min_eq(i, val);
            return;
        }
        push(i);
        int m = (r + l) / 2;
        if (qr <= m) {
            update(i * 2 + 1, l, m, ql, qr, val);
        } else if (ql >= m) {
            update(i * 2 + 2, m, r, ql, qr, val);
        } else {
            update(i * 2 + 1, l, m, ql, m, val);
            update(i * 2 + 2, m, r, m, qr, val);
        }
        upd_from_children(i);
    }

    int sum(int i, int l, int r, int ql, int qr) {
        if (l == ql && r == qr) {
            return st[i];
        }
        push(i);
        int m = (r + l) / 2;
        if (qr <= m) {
            return sum(i * 2 + 1, l, m, ql, qr);
        }
        if (ql >= m) {
            return sum(i * 2 + 2, m, r, ql, qr);
        }
        return sum(i * 2 + 1, l, m, ql, m) + sum(i * 2 + 2, m, r, m, qr);
    }
};
\end{lstlisting}

\subsection{ДД по неявному}
\begin{lstlisting}[language=C++]
pair<Node *, Node *> split(Node *now, int k) {
    if (!now) {
        return {nullptr, nullptr};
    }
    if (size(now->l) + 1 <= k) {
        auto ans = split(now->r, k - 1 - size(now->l));
        now->r = ans.first;
        update_size(now);
        return {now, ans.second};
    }
    auto ans = split(now->l, k);
    now->l = ans.second;
    update_size(now);
    return {ans.first, now};
}

Node *merge(Node *l, Node *r) {
    if (!l) {
        return r;
    }
    if (!r) {
        return l;
    }
    if (l->y <= r->y) {
        auto ans = merge(l->r, r);
        l->r = ans;
        update_size(l);
        return l;
    }
    auto ans = merge(l, r->l);
    r->l = ans;
    update_size(r);
    return r;
}

Node *insert(Node *root, int pos) {
    auto r = split(root, pos);
    Node *nn = new Node(pos);
    root = merge(r.first, nn);
    root = merge(root, r.second);
    return root;
}

Node *to_begin(Node *root, int l, int r) {
    auto a = split(root, l);
    auto b = split(a.second, r - l);
    return merge(b.first, merge(a.first, b.second));
}

\end{lstlisting}

\subsection{ДД}
\begin{lstlisting}[language=C++]
pair<Node *, Node *> split(Node *now, ll x) {
    if (!now) {
        return {nullptr, nullptr};
    }
    if (now->x <= x) {
        auto ans = split(now->r, x);
        now->r = ans.first;
        update_sum(now);
        return {now, ans.second};
    }
    auto ans = split(now->l, x);
    now->l = ans.second;
    update_sum(now);
    return {ans.first, now};
}

Node *merge(Node *l, Node *r) {
    if (!l) {
        return r;
    }
    if (!r) {
        return l;
    }
    if (l->y <= r->y) {
        auto ans = merge(l->r, r);
        l->r = ans;
        update_sum(l);
        return l;
    }
    auto ans = merge(l, r->l);
    r->l = ans;
    update_sum(r);
    return r;
}

Node *insert(Node *root, ll val) {
    Node *new_v = new Node(val);
    auto ans = split(root, val);
    return merge(merge(ans.first, new_v), ans.second);
}

Node *del(Node *root, ll val) {
    auto ans = split(root, val);
    auto ans1 = split(ans.first, val - 1);
    return merge(ans1.first, ans.second);
}

ll get_sum(Node *root, ll l, ll r) {
    if (!root) {
        return 0;
    }
    auto ans = split(root, l - 1);
    auto ans2 = split(ans.second, r);
    if (!ans2.first) {
        merge(merge(ans.first, ans2.first), ans2.second);
        return 0;
    }
    ll res = ans2.first->sum_;
    merge(merge(ans.first, ans2.first), ans2.second);
    return res;
}

\end{lstlisting}

\subsection{Персистентное ДД}
\begin{lstlisting}[language=C++]
// TODO

\end{lstlisting}

\subsection{Персистентное ДО}
\begin{lstlisting}[language=C++]
struct ST {
    vector<Node *> roots;
    int n;

    ST(vector<int> &a) {
        start = a;
        n = a.size();
        roots.push_back(nullptr);
        build(roots[0], 0, n);
    }

    void update(Node *&now, Node *old, int l, int r, int pos, int qd) {
        if (l + 1 == r) {
            now = new_node(qd);
            return;
        }
        now = new_node();
        int m = (l + r) / 2;
        if (pos < m) {
            now->r = old->r;
            update(now->l, old->l, l, m, pos, qd);
        } else {
            now->l = old->l;
            update(now->r, old->r, m, r, pos, qd);
        }
        now->sm = now->l->sm + now->r->sm;
    }
};

\end{lstlisting}

\subsection{Спарсы}
\begin{lstlisting}[language=C++]
struct Sparse_Table {
    vector<vector<int>> st;
    vector<int> max2;

    Sparse_Table(vector<int> &a) {
        int n = a.size();
        st.emplace_back();
        for (int i = 0; i < n; i++) {
            st[0].push_back(a[i]);
        }
        for (int i = 1; (1 << i) <= n; i++) {
            st.emplace_back();
            for (int p = 0; p + (1 << i) <= n; p++) {
                st[i].push_back(min(st[i - 1][p], st[i - 1][p + (1 << (i - 1))]));
            }
        }
        max2.resize(n + 1);
        max2[1] = 0;
        for (int i = 2; i <= n; i++) {
            max2[i] = max2[i / 2] + 1;
        }
    }

    int rmq(int l, int r) {
        r++;
        int i = max2[r - l];
        return min(st[i][l], st[i][r - (1 << i)]);
    }
};

\end{lstlisting}

\subsection{Фенвик (pref += x)}
\begin{lstlisting}[language=C++]
// a[left..right] += delta;  get_sum a[1..pos]
//
//void update(left, right, delta)
//  T1.add(left, delta)
//  T1.add(right + 1, -delta);
//  T2.add(left, delta * (left - 1))
//  T2.add(right + 1, -delta * right);
//
//int getSum(pos)
//  return T1.sum(pos) * pos - T2.sum(pos)

\end{lstlisting}

\subsection{Фенвик}
\begin{lstlisting}[language=C++]
// Нумерация с 1

struct Fenwick_tree {
    vector<vector<vector<int>>> ft;

    Fenwick_tree(int n) {
        ft.resize(n + 1, vector<vector<int>>(n + 1, vector<int>(n + 1)));
    }

    void upd(int x, int y, int z, int d) {
        for (int x1 = x; x1 < ft.size(); x1 += x1 & -x1) {
            for (int y1 = y; y1 < ft[x1].size(); y1 += y1 & -y1) {
                for (int z1 = z; z1 < ft[x1][y1].size(); z1 += z1 & -z1) {
                    ft[x1][y1][z1] += d;
                }
            }
        }
    }

    int rsq(int x, int y, int z) {
        int ans = 0;
        for (int x1 = x; x1 > 0; x1 -= x1 & -x1) {
            for (int y1 = y; y1 > 0; y1 -= y1 & -y1) {
                for (int z1 = z; z1 > 0; z1 -= z1 & -z1) {
                    ans += ft[x1][y1][z1];
                }
            }
        }
        return ans;
    }

    int sum_3d(int x1, int x2, int y1, int y2, int z1, int z2) {
        int ans = rsq(x2, y2, z1 - 1) + rsq(x1 - 1, y2, z2) - rsq(x1 - 1, y2, z1 - 1);
        ans += rsq(x2, y1 - 1, z2);
        ans -= rsq(x2, y1 - 1, z1 - 1) + rsq(x1 - 1, y1 - 1, z2) - rsq(x1 - 1, y1 - 1, z1 - 1);
        return rsq(x2, y2, z2) - ans;
    }
};

\end{lstlisting}


\end{multicols*}
\end{document}


